\documentclass{beamer}
\usepackage[slovene]{babel}

\newcommand{\ttitle}{Algoritmi za reševanje problema matričnih napolnitev}
\newcommand{\ttitleEn}{Algorithms for solving the matrix completion problem}
\newcommand{\tsubject}{\ttitle}
\newcommand{\tsubjectEn}{\ttitleEn}
\newcommand{\tauthor}{Matej Klančar}
\newcommand{\tkeywords}{matrične napolnitve, minimizacija ranga, rekonstrukcija slik, priporočilni sistemi}
\newcommand{\tkeywordsEn}{matrix completion, rank minimization, image reconstruction,
recommendation systems}
\newcommand{\nnorm}[1]{\lVert#1\rVert_*}
\newcommand{\fnorm}[1]{\lVert#1\rVert_F}
\newcommand{\norm}[1]{\lVert#1\rVert}
\newcommand{\CR}[1]{\begin{color}{red}#1\end{color}}
\newcommand{\CB}[1]{\begin{color}{blue}#1\end{color}}
\newcommand{\CG}[1]{\begin{color}{green}#1\end{color}}

\newcommand{\proj}{\mathcal{P}_\Omega}
\newcommand{\shrink}{\mathcal{D}}
%\newcommand{\tr}{\text{Tr}}
\newcommand{\numberthis}{\addtocounter{equation}{1}\tag{\theequation}}
\newcommand{\trOp}[2]{\langle #1, #2 \rangle}
\newcommand{\mapa}{Poglavja}

\DeclareMathOperator{\diag}{diag}
\DeclareMathOperator{\tr}{Tr}
\DeclareMathOperator*{\argmin}{arg\,min}

%Information to be included in the title page:
\title{Algoritmi za reševanje problema matričnih napolnitev}
\author[Matej Klančar]{Matej Klančar \\ \vspace{0.2cm} Mentor: doc.\ dr.\ Aljaž Zalar}
\usetheme{Madrid}
\date{2023}

\definecolor{colorTheme}{rgb}{0.19, 0.73, 0.56} % UBC Blue (primary)
\usecolortheme[named=colorTheme]{structure}
\setbeamercolor{alerted text}{fg=colorTheme}


\makeatletter
\setbeamertemplate{footline}
{
  \leavevmode%
  \hbox{%
  \begin{beamercolorbox}[wd=.25\paperwidth,ht=2.25ex,dp=1ex,center]{author in head/foot}%
    \usebeamerfont{author in head/foot}\insertshortauthor\expandafter\beamer@ifempty\expandafter{\beamer@shortinstitute}{}{~~(\insertshortinstitute)}
  \end{beamercolorbox}%
  \begin{beamercolorbox}[wd=.5\paperwidth,ht=2.25ex,dp=1ex,center]{title in head/foot}%
    \usebeamerfont{title in head/foot}\insertshorttitle
  \end{beamercolorbox}%
  \begin{beamercolorbox}[wd=.25\paperwidth,ht=2.25ex,dp=1ex,right]{date in head/foot}%
    \usebeamerfont{date in head/foot}\insertshortdate{}\hspace*{2em}
    \insertframenumber{} / \inserttotalframenumber\hspace*{2ex} 
  \end{beamercolorbox}}%
  \vskip0pt%
}
\makeatother

\AtBeginSection[]
{
    \begin{frame}
        \frametitle{Kazalo}
        \tableofcontents[currentsection]
    \end{frame}
}

\begin{document}
\selectlanguage{slovene}
\frame{\titlepage}
\section{Predstavitev problema}

\section{Algoritmi}
\begin{frame}
  \frametitle{Algoritem NNM}
  \begin{itemize}
    \item Minimizira \alert{nuklearno normo} \[
      \nnorm{A} = \sum_{i = 1}^{n} \sigma_i(A)
    \]
    \item Rešuje problem semidefinitega programiranja
    \item Lahko rešujemo s primernimi orodji za reševanje takih problemov (npr. Sedumi)
  \end{itemize}
\end{frame}

\begin{frame}
  \frametitle{Algoritem SVT}
  \begin{itemize}
    \item Problem definira z Lagrangeovo funkcijo \[
      \mathcal{L}(X, Y) = \tau \nnorm{X} + \frac{1}{2}\fnorm{X}^2 + \trOp{Y}{\proj(M-X)} 
    \]
    \item Uporabimo t.i. Uzawa algoritem, rešujemo iterativno
    \begin{align*}
              X^{(k)} &= \shrink_\tau(Y^{(k-1)}), \\
              Y^{(k)} &= Y^{(k-1)} + \delta_k \proj(M - X^{(k)}), 
      \end{align*}
      \item \alert{Operator praga} $\shrink_\tau: \mathbb{R}^{n_1 \times n_2} \rightarrow \mathbb{R}^{n_1 \times n_2}$ je definiran kot
      \begin{align*}
          \shrink_\tau(A) &:= U \shrink_\tau(\Sigma) V^T, \\ \shrink_\tau(\Sigma) &= \diag\big(\max(\sigma_1 - \tau, 0),
          \max(\sigma_2-\tau,0),\ldots,\\
          &\hspace{1.5cm}
          \max(\sigma_{\min(n_1,n_2)}-\tau,0)\big)
      \end{align*}
      \item Uporabi idejo, da ima matrika majhnega ranga le nekaj velikih singularnih vrednosti.
  \end{itemize}
\end{frame}

\begin{frame}
  \frametitle{Algoritem TNNM}
  \begin{itemize}
    \item Minimizira \alert{prirezano nuklearno normo} \[
      \norm{X}_r = \sum_{i = r + 1}^{\min(n_1, n_2)} \sigma_i(X)
    \]
    \item Uporabi $A_l, B_l$, sestavljeni iz lastnih vektorjev $X$ 
    \item Uporablja iterativni algoritem ADMM, definira pomožni matriki $W, Y$ 
    \begin{align*}
        X^{(k+1)} &= \shrink_\frac{1}{\beta}(W^{(k)} - \frac{1}{\beta} Y^{(k)})\\
        W^{(k+1)} &= X^{(k+1)} + \frac{1}{\beta}(A_l^T B_l + Y^{(k)})\\
        Y^{(k+1)} &= Y^{(k)} + \beta(X^{(k+1)}- W^{(k+1)})
      \end{align*}
  \end{itemize}
\end{frame}

\begin{frame}
  \frametitle{Algoritem ASD}
  \begin{itemize}
    \item Išče matriki $X, Y$ katerih produkt bo enak rekonstrukciji
    \item Uporablja \alert{gradientni spust} nad funkcijo \[
      f(X, Y) = \frac{1}{2}\, \fnorm{\proj(M) - \proj(XY)}^2  
    \]
    \item Z uporabo gradienta in optimalnega koraka izvaja premik 
    \begin{align*}
      X^{(k+1)} = X^{(k)} - t_{x^{(k)}} \nabla f_{Y^{(k)}}(X^{(k)}) \\
      Y^{(k+1)} = Y^{(k)} - t_{y^{(k)}} \nabla f_{X^{(k+1)}}(Y^{(k)})
  \end{align*}
  \end{itemize}
\end{frame}

\begin{frame}
  \frametitle{Algoritem LMaFit}
  \begin{itemize}
    \item Uporablja Moore-Penrose inverz
    \item Množenje rešuje po metodi najmanjših kvadratov
    \begin{align*}
      X^{(k+1)} &= Z^{(k)}(Y^{(k)})^\dagger \\
      Y^{(k+1)} &= (X^{(k+1)})^\dagger Z^{(k)} \\
      Z^{(k+1)} &= X^{(k+1)}Y^{(k+1)} + \proj(M - X^{(k+1)}Y^{(k+1)})
    \end{align*}
  \end{itemize}
\end{frame}


\section{Rezultati}
\section{Zaključek}
\end{document}