%%%%%%%%%%%%%%%%%%%%%%%%%%%%%%%%%%%%%%%%
% datoteka diploma-FRI-vzorec.tex
%
% vzorčna datoteka za pisanje diplomskega dela v formatu LaTeX
% na UL Fakulteti za računalništvo in informatiko
%
% na osnovi starejših verzij vkup spravil Franc Solina, maj 2021
% prvo verzijo je leta 2010 pripravil Gašper Fijavž
%
% za upravljanje z literaturo ta vezija uporablja BibLaTeX
%
% svetujemo uporabo Overleaf.com - na tej spletni implementaciji LaTeXa ta vzorec zagotovo pravilno deluje
%

\documentclass[a4paper,12pt,openright]{book}
%\documentclass[a4paper, 12pt, openright, draft]{book}  Nalogo preverite tudi z opcijo draft, ki pokaže, katere vrstice so predolge! Pozor, v draft opciji, se slike ne pokažejo!
 
\usepackage[utf8]{inputenc}   % omogoča uporabo slovenskih črk kodiranih v formatu UTF-8
\usepackage[slovene,english]{babel}    % naloži, med drugim, slovenske delilne vzorce
\usepackage[pdftex]{graphicx}  % omogoča vlaganje slik različnih formatov
\usepackage{fancyhdr}          % poskrbi, na primer, za glave strani
\usepackage{amssymb}           % dodatni matematični simboli
\usepackage{amsmath}           % eqref, npr.
\usepackage{hyperxmp}
\usepackage{mathtools}
\usepackage{todonotes}
\usepackage{subcaption} %  for subfigures environments 
\usepackage{placeins}
\usepackage{diagbox}
\usepackage[hyphens]{url}
\usepackage{csquotes}
\usepackage{float}
\usepackage{amsthm}
\usepackage[pdftex, colorlinks=true,
						citecolor=black, filecolor=black, 
						linkcolor=black, urlcolor=black,
						pdfproducer={LaTeX}, pdfcreator={LaTeX}]{hyperref}

\usepackage{color}
\usepackage{soul}

\usepackage[
backend=biber,
style=numeric,
sorting=nty,
]{biblatex}


\addbibresource{literatura.bib} %Imports bibliography file


%%%%%%%%%%%%%%%%%%%%%%%%%%%%%%%%%%%%%%%%
%	DIPLOMA INFO
%%%%%%%%%%%%%%%%%%%%%%%%%%%%%%%%%%%%%%%%
\newcommand{\ttitle}{Algoritmi za reševanje problema matričnih napolnitev}
\newcommand{\ttitleEn}{Algorithms for solving the matrix completion problem}
\newcommand{\tsubject}{\ttitle}
\newcommand{\tsubjectEn}{\ttitleEn}
\newcommand{\tauthor}{Matej Klančar}
\newcommand{\tkeywords}{matrične napolnitve, minimizacija ranga, rekonstrukcija slik, priporočilni sistemi}
\newcommand{\tkeywordsEn}{matrix completion, rank minimization, image reconstruction,
recommendation systems}
\newcommand{\nnorm}[1]{\lVert#1\rVert_*}
\newcommand{\fnorm}[1]{\lVert#1\rVert_F}
\newcommand{\norm}[1]{\lVert#1\rVert}
\DeclareMathOperator{\diag}{diag}
\DeclareMathOperator{\tr}{Tr}
\DeclareMathOperator*{\argmin}{arg\,min}


%%%%%%%%%%%%%%%%%%%%%%%%%%%%%%%%%%%%%%%%
%	HYPERREF SETUP
%%%%%%%%%%%%%%%%%%%%%%%%%%%%%%%%%%%%%%%%
\hypersetup{pdftitle={\ttitle}}
\hypersetup{pdfsubject=\ttitleEn}
\hypersetup{pdfauthor={\tauthor}}
\hypersetup{pdfkeywords=\tkeywordsEn}

%%%%%%%%%%%%%%%%%%%%%%%%%%%%%%%%%%%%%%%%
% postavitev strani
%%%%%%%%%%%%%%%%%%%%%%%%%%%%%%%%%%%%%%%%  

\addtolength{\marginparwidth}{-20pt} % robovi za tisk
\addtolength{\oddsidemargin}{40pt}
\addtolength{\evensidemargin}{-40pt}

\renewcommand{\baselinestretch}{1.3} % ustrezen razmik med vrsticami
\setlength{\headheight}{15pt}        % potreben prostor na vrhu
\renewcommand{\chaptermark}[1]%
{\markboth{\MakeUppercase{\thechapter.\ #1}}{}} \renewcommand{\sectionmark}[1]%
{\markright{\MakeUppercase{\thesection.\ #1}}} \renewcommand{\headrulewidth}{0.5pt} \renewcommand{\footrulewidth}{0pt}
\fancyhf{}
\fancyhead[LE,RO]{\sl \thepage} 
%\fancyhead[LO]{\sl \rightmark} \fancyhead[RE]{\sl \leftmark}
\fancyhead[RE]{\sc \tauthor}              % dodal Solina
\fancyhead[LO]{\sc Diplomska naloga}     % dodal Solina


\newcommand{\BibLaTeX}{{\sc Bib}\LaTeX}
\newcommand{\BibTeX}{{\sc Bib}\TeX}

%%%%%%%%%%%%%%%%%%%%%%%%%%%%%%%%%%%%%%%%
% naslovi
%%%%%%%%%%%%%%%%%%%%%%%%%%%%%%%%%%%%%%%%  

\newcommand{\autfont}{\Large}
\newcommand{\titfont}{\LARGE\bf}
\newcommand{\clearemptydoublepage}{\newpage{\pagestyle{empty}\cleardoublepage}}
\setcounter{tocdepth}{1}	      % globina kazala

%%%%%%%%%%%%%%%%%%%%%%%%%%%%%%%%%%%%%%%%
% konstrukti
%%%%%%%%%%%%%%%%%%%%%%%%%%%%%%%%%%%%%%%%  
\newtheorem{izrek}{Izrek}[chapter]
\newtheorem{theorem}{Izrek}[chapter]
\newtheorem{trditev}[theorem]{Trditev}
\newenvironment{dokaz}{\emph{Dokaz.}\ }{\hspace{\fill}{$\Box$}}

\newcommand{\CR}[1]{\begin{color}{red}#1\end{color}}
\newcommand{\CB}[1]{\begin{color}{blue}#1\end{color}}
\newcommand{\CG}[1]{\begin{color}{green}#1\end{color}}

\newcommand{\proj}{\mathcal{P}_\Omega}
\newcommand{\shrink}{\mathcal{D}}
%\newcommand{\tr}{\text{Tr}}
\newcommand{\numberthis}{\addtocounter{equation}{1}\tag{\theequation}}
\newcommand{\trOp}[2]{\langle #1, #2 \rangle}
\newcommand{\mapa}{Poglavja}


%%%%%%%%%%%%%%%%%%%%%%%%%%%%%%%%%%%%%%%%%%%%%%%%%%%%%%%%%%%%%%%%%%%%%%%%%%%%%%%
%% PDF-A
%%%%%%%%%%%%%%%%%%%%%%%%%%%%%%%%%%%%%%%%%%%%%%%%%%%%%%%%%%%%%%%%%%%%%%%%%%%%%%%

%%%%%%%%%%%%%%%%%%%%%%%%%%%%%%%%%%%%%%%% 
% define medatata
%%%%%%%%%%%%%%%%%%%%%%%%%%%%%%%%%%%%%%%% 
\def\Title{\ttitle}
\def\Author{\tauthor, matjaz.kralj@fri.uni-lj.si}
\def\Subject{\ttitleEn}
\def\Keywords{\tkeywordsEn}

%%%%%%%%%%%%%%%%%%%%%%%%%%%%%%%%%%%%%%%% 
% \convertDate converts D:20080419103507+02'00' to 2008-04-19T10:35:07+02:00
%%%%%%%%%%%%%%%%%%%%%%%%%%%%%%%%%%%%%%%% 
\def\convertDate{%
    \getYear
}

{\catcode`\D=12
 \gdef\getYear D:#1#2#3#4{\edef\xYear{#1#2#3#4}\getMonth}
}
\def\getMonth#1#2{\edef\xMonth{#1#2}\getDay}
\def\getDay#1#2{\edef\xDay{#1#2}\getHour}
\def\getHour#1#2{\edef\xHour{#1#2}\getMin}
\def\getMin#1#2{\edef\xMin{#1#2}\getSec}
\def\getSec#1#2{\edef\xSec{#1#2}\getTZh}
\def\getTZh +#1#2{\edef\xTZh{#1#2}\getTZm}
\def\getTZm '#1#2'{%
    \edef\xTZm{#1#2}%
    \edef\convDate{\xYear-\xMonth-\xDay T\xHour:\xMin:\xSec+\xTZh:\xTZm}%
}

%\expandafter\convertDate\pdfcreationdate 

%%%%%%%%%%%%%%%%%%%%%%%%%%%%%%%%%%%%%%%%
% get pdftex version string
%%%%%%%%%%%%%%%%%%%%%%%%%%%%%%%%%%%%%%%% 
\newcount\countA
\countA=\pdftexversion
\advance \countA by -100
\def\pdftexVersionStr{pdfTeX-1.\the\countA.\pdftexrevision}


%%%%%%%%%%%%%%%%%%%%%%%%%%%%%%%%%%%%%%%%
% XMP data
%%%%%%%%%%%%%%%%%%%%%%%%%%%%%%%%%%%%%%%%  
\usepackage{xmpincl}
%\includexmp{pdfa-1b}

%%%%%%%%%%%%%%%%%%%%%%%%%%%%%%%%%%%%%%%%
% pdfInfo
%%%%%%%%%%%%%%%%%%%%%%%%%%%%%%%%%%%%%%%%  
\pdfinfo{%
    /Title    (\ttitle)
    /Author   (\tauthor, damjan@cvetan.si)
    /Subject  (\ttitleEn)
    /Keywords (\tkeywordsEn)
    /ModDate  (\pdfcreationdate)
    /Trapped  /False
}

%%%%%%%%%%%%%%%%%%%%%%%%%%%%%%%%%%%%%%%%
% znaki za copyright stran
%%%%%%%%%%%%%%%%%%%%%%%%%%%%%%%%%%%%%%%%  

\newcommand{\CcImageCc}[1]{%
	\includegraphics[scale=#1]{cc_cc_30.pdf}%
}
\newcommand{\CcImageBy}[1]{%
	\includegraphics[scale=#1]{cc_by_30.pdf}%
}
\newcommand{\CcImageSa}[1]{%
	\includegraphics[scale=#1]{cc_sa_30.pdf}%
}

%%%%%%%%%%%%%%%%%%%%%%%%%%%%%%%%%%%%%%%%%%%%%%%%%%%%%%%%%%%%%%%%%%%%%%%%%%%%%%%
%%%%%%%%%%%%%%%%%%%%%%%%%%%%%%%%%%%%%%%%%%%%%%%%%%%%%%%%%%%%%%%%%%%%%%%%%%%%%%%

\begin{document}
\selectlanguage{slovene}
\frontmatter
\setcounter{page}{1} %
\renewcommand{\thepage}{}       % preprečimo težave s številkami strani v kazalu

%%%%%%%%%%%%%%%%%%%%%%%%%%%%%%%%%%%%%%%%
%naslovnica
\thispagestyle{empty}%
\begin{center}
    {\large\sc Univerza v Ljubljani\\%
        %      Fakulteta za elektrotehniko\\% za študijski program Multimedija
        %      Fakulteta za upravo\\% za študijski program Upravna informatika
        Fakulteta za računalništvo in informatiko\\%
        %      Fakulteta za matematiko in fiziko\\% za študijski program Računalništvo in matematika
    }
    \vskip 10em%
        {\autfont \tauthor\par}%
        {\titfont \ttitle \par}%
        {\vskip 3em \textsc{DIPLOMSKO DELO\\[5mm]         % dodal Solina za ostale študijske programe
                %    VISOKOŠOLSKI STROKOVNI ŠTUDIJSKI PROGRAM\\ PRVE STOPNJE\\ RAČUNALNIŠTVO IN INFORMATIKA}\par}%
                UNIVERZITETNI  ŠTUDIJSKI PROGRAM\\ PRVE STOPNJE\\ RAČUNALNIŠTVO IN INFORMATIKA}\par}%
    %    INTERDISCIPLINARNI UNIVERZITETNI\\ ŠTUDIJSKI PROGRAM PRVE STOPNJE\\ MULTIMEDIJA}\par}%
    %    INTERDISCIPLINARNI UNIVERZITETNI\\ ŠTUDIJSKI PROGRAM PRVE STOPNJE\\ UPRAVNA INFORMATIKA}\par}%
    %    INTERDISCIPLINARNI UNIVERZITETNI\\ ŠTUDIJSKI PROGRAM PRVE STOPNJE\\ RAČUNALNIŠTVO IN MATEMATIKA}\par}%
    \vfill\null%
    % izberite pravi habilitacijski naziv mentorja!
    {\large \textsc{Mentor}: doc.\ dr.\
        Aljaž Zalar\par}%
    {\vskip 2em \large Ljubljana, \the\year \par}%
\end{center}
% prazna stran
%\clearemptydoublepage      
% izjava o licencah itd. se izpiše na hrbtni strani naslovnice

%%%%%%%%%%%%%%%%%%%%%%%%%%%%%%%%%%%%%%%%
%copyright stran
%%%%%%%%%%%%%%%%%%%%%%%%%%%%%%%%%%%%%%%%
\newpage
\thispagestyle{empty}

\vspace*{5cm}
{\small \noindent
    To delo je ponujeno pod licenco \textit{Creative Commons Priznanje avtorstva-Deljenje pod enakimi pogoji 2.5 Slovenija} (ali novej\v so razli\v cico).
    To pomeni, da se tako besedilo, slike, grafi in druge sestavine dela kot tudi rezultati diplomskega dela lahko prosto distribuirajo,
    reproducirajo, uporabljajo, priobčujejo javnosti in predelujejo, pod pogojem, da se jasno in vidno navede avtorja in naslov tega
    dela in da se v primeru spremembe, preoblikovanja ali uporabe tega dela v svojem delu, lahko distribuira predelava le pod
    licenco, ki je enaka tej.
    Podrobnosti licence so dostopne na spletni strani \href{http://creativecommons.si}{creativecommons.si} ali na Inštitutu za
    intelektualno lastnino, Streliška 1, 1000 Ljubljana.

    \vspace*{1cm}
    \begin{center}% 0.66 / 0.89 = 0.741573033707865
        \CcImageCc{0.741573033707865}\hspace*{1ex}\CcImageBy{1}\hspace*{1ex}\CcImageSa{1}%
    \end{center}
}

\vspace*{1cm}
{\small \noindent
    Izvorna koda diplomskega dela, njeni rezultati in v ta namen razvita programska oprema je ponujena pod licenco GNU General Public License,
    različica 3 (ali novejša). To pomeni, da se lahko prosto distribuira in/ali predeluje pod njenimi pogoji.
    Podrobnosti licence so dostopne na spletni strani \url{http://www.gnu.org/licenses/}.
}

\vfill
\begin{center}
    \ \\ \vfill
    {\em
        Besedilo je oblikovano z urejevalnikom besedil \LaTeX.}
\end{center}

% prazna stran
\clearemptydoublepage

%%%%%%%%%%%%%%%%%%%%%%%%%%%%%%%%%%%%%%%%
% stran 3 med uvodnimi listi
\thispagestyle{empty}
\
\vfill

\bigskip
\noindent\textbf{Kandidat:} Matej Klančar\\
\noindent\textbf{Naslov:} Naslov diplomskega dela\\
% vstavite ustrezen naziv študijskega programa!
\noindent\textbf{Vrsta naloge:} Diplomska naloga na univerzitetnem programu prve stopnje Računalništvo in informatika \\
% izberite pravi habilitacijski naziv mentorja!
\noindent\textbf{Mentor:} doc.\ dr.\ Aljaž Zalar\\

\bigskip
\noindent\textbf{Opis:}\\
V diplomskem delu 
predstavite nekaj različnih algoritmov za reševanje problema matričnih napolnitev. Seznanite se z uporabljenimi matematičnimi orodji. Algoritme implementirajte in testirajte na izbranem problemu uporabe.
Primerjajte algoritme glede na kakovost napolnitve, časovno zahtevnost in pojasnite ugotovitve z navezavo na teoretično podlago algoritmov. Na koncu predlagajte možne nadaljnje izboljšave.

\bigskip
\noindent\textbf{Title:} Algorithms for solving matrix completion problem

\bigskip
\noindent\textbf{Description:}\\
In the thesis, 
present a few different algorithms for solving the matrix completion problem.
Study the underlying mathematical methods. Implement the algorithms and test them on a selected applied problem. Compare the algorithms in terms of quality of the completions, computational cost, and explain the results referring to the theoretical background of the algorithms. Finally, suggest possible improvements to the methods presented.
\vfill



\vspace{2cm}

% prazna stran
\clearemptydoublepage

% zahvala
\thispagestyle{empty}\mbox{}\vfill\null\it%
\noindent
Velika zahvala za nastanek diplomskega dela gre mentorju doc.\ dr.\ Aljažu Zalarju, ki mi je pri pisanju diplomske naloge skrbno pomagal s številnimi nasveti, idejami in popravki.
Zahvaljujem se tudi družini in prijateljem, ki so me vzpodbujali in mi tekom študija dajali podporo in motivacijo.

\rm\normalfont

% prazna stran
\clearemptydoublepage

%%%%%%%%%%%%%%%%%%%%%%%%%%%%%%%%%%%%%%%%
% % posvetilo, če sama zahvala ne zadošča :-)
% \thispagestyle{empty}\mbox{}{\vskip0.20\textheight}\mbox{}\hfill\begin{minipage}{0.55\textwidth}%
%     Svoji dragi Alenčici.
%     \normalfont\end{minipage}

% prazna stran
\clearemptydoublepage


%%%%%%%%%%%%%%%%%%%%%%%%%%%%%%%%%%%%%%%%
% kazalo
\pagestyle{empty}
\def\thepage{}% preprečimo težave s številkami strani v kazalu
\tableofcontents{}


% prazna stran
\clearemptydoublepage

%%%%%%%%%%%%%%%%%%%%%%%%%%%%%%%%%%%%%%%%
% seznam kratic

\chapter*{Seznam uporabljenih kratic}

\noindent\begin{tabular}{p{0.11\textwidth}|p{.39\textwidth}|p{.39\textwidth}}    % po potrebi razširi prvo kolono tabele na račun drugih dveh!
    {\bf kratica} & {\bf angleško}                      & {\bf slovensko}                        \\ \hline
    {\bf SVT }    & Singular Value Thresholding         & Prag singularnih vrednosti             \\
    {\bf NNM}     & Nuclear norm minimization           & Minimizacija nuklearne norme           \\
    {\bf TNNM}    & Truncated nuclear norm minimization & Minimizacija prirezane nuklearne norme \\
    {\bf ASD}     & Alternating Steepest Descent        & Izmenjujoč gradientni spust            \\
    {\bf NP}     & Nondeterministic polynomial time       & Nedeterministični polinomski čas            \\
    {\bf SDP}     & Semidefinite programming        & Semidefinitno programiranje            
    %  \dots & \dots & \dots \\
\end{tabular}


% prazna stran
\clearemptydoublepage

%%%%%%%%%%%%%%%%%%%%%%%%%%%%%%%%%%%%%%%%
% povzetek
\phantomsection
\addcontentsline{toc}{chapter}{Povzetek}
\chapter*{Povzetek}

\noindent\textbf{Naslov:} \ttitle
\bigskip

\noindent\textbf{Avtor:} \tauthor
\bigskip

%\noindent\textbf{Povzetek:} 
\noindent Problem matričnih napolnitev sprejme matriko, ki nima določenih vrednosti vseh elementov, cilj pa je določiti vrednosti teh elementov tako, da bo rang napolnjene matrike najmanjši možen. V diplomskem delu predstavimo teoretično ozadje petih različnih algoritmov, ki rešujejo ta problem (NNM, SVT, TNNM, ASD, LMaFit), in jih testiramo. Pri testiranju se osredotočimo na problem rekonstrukcije slik, kjer vrednosti nekaterih pikslov ne poznamo. Analiziramo različne vidike rekonstrukcij, rezultate
pa interpretiramo prek matematičnega ozadja algoritmov. Rezultate primerjamo tudi z uveljavljeno metodo rekonstrukcije slik, ki temelji na reševanju Laplaceove diferencialne enačbe.

\bigskip

\noindent\textbf{Ključne besede:} \tkeywords.
% prazna stran
\clearemptydoublepage

%%%%%%%%%%%%%%%%%%%%%%%%%%%%%%%%%%%%%%%%
% abstract
\phantomsection
\selectlanguage{english}
\addcontentsline{toc}{chapter}{Abstract}
\chapter*{Abstract}

\noindent\textbf{Title:} \ttitleEn
\bigskip

\noindent\textbf{Author:} \tauthor
\bigskip

%\noindent\textbf{Abstract:} 
\noindent 
    The matrix completion problem considers a matrix in which some elements are unknown. The goal is to determine the elements, such that the rank of the filled matrix is minimal. In this thesis, we present the theoretical background of five different algorithms used to solve this problem (NNM, SVT, TNNM, ASD, LMaFit) and test them. In testing, we focus on the reconstruction of images where the values of some pixels are unknown.
    We analyze different aspects of the reconstructions and interpret the results 
    referring to the mathematical background of the algorithms. We also compare the results with a more common method of image reconstruction, based on solving the Laplace differential equations.
\bigskip

\noindent\textbf{Keywords:} \tkeywordsEn.
\selectlanguage{slovene}
% prazna stran
\clearemptydoublepage

%%%%%%%%%%%%%%%%%%%%%%%%%%%%%%%%%%%%%%%%
\mainmatter
\setcounter{page}{1}
\pagestyle{fancy}


% Optional TOC
% \tableofcontents
% \pagebreak
\chapter{Uvod}
\todo{spisi in popravi vse}
\section{Motivacija}
Problem matričnih napolnitev sprejme matriko, največkat označeno z $M$, pri kateri so nekateri elementi označeni kot neznani. Problem nato sprašuje po vrednostih, ki jih lahko vstavimo v neznane vrednosti, tako da bo rang matrike najmanjši možen. Gre za NP-poln problem, zato ga poskušamo poenostaviti, ter reševati lažje probleme, ki vrnejo dovolj dobre, a ne optimalne rešitve. 

Problem je v zadnjih letih zelo popularen, z njim pa se ukvarjajo tako številni matematiki kot računalničarji. Njegova splošnost naredi reševanje problema na številnih področjih, sam pa se v diplomski nalogi osredotočim na razreševanje neznanih pikslov v slikah. Prav tako omenjam in preizkusim algoritem na priporočilnih sistemih. Rezultate teh predstavim v poglavju X.

V tej diplomski nalogi bom predstavil par  algoritmov, ki rešujejo omenjen problem. Algoritmi so bili izbrani glede na njihovo popularnost in priznanost v literaturi. Prav tako sem poskrbel, da so algoritmi primerno različni in temeljijo na drugačnih principih. Algoritme sem tudi implementiral, nato pa še napravil analizo ter opisal ugotovitve v poglavju X.

\section{Cilji}

Nekaj o tem, kaj želimo narediti.

Glavni prispevki tega diplomskega dela so ...

\section{Struktura diplomskega dela}

V poglavju 2 ,,,, v poglavju 3 ....
\chapter{Pregled področja} \label{1407-1010}
Področje matričnih napolnitev je trenutno zelo aktivno, s številnimi raziskovalci, ki na danem področju raziskujejo in iščejo nove načine reševanja problema. Algoritmi, ki problem rešujejo, so zaradi splošnosti problema zelo uporabni, kar pojasnjuje motivacijo po iskanju učinkovitih algoritmov.

Eno prvih del, ki sam problem opisuje je \cite{MCPAS}. Delo opisuje sam problem in njegovo naravo, vendar se ne osredotoča zgolj na napolnjevanje s ciljem minimalnega ranga. Delo opisuje tudi ideje za napolnitve matrik tako, da je napolnjena matrika pozitivno semidefinitna, kot tudi maksimizacijo determinante napolnjene matrike.

Doktorska disertacija \cite{NNM-PHD} podaja pomembne opise pretvorbe problema minimizacije ranga v problem iz področja semidefinitnega programiranja. Ideje in dokazi tega dela služijo kot temelj za algoritme, ki so se razvili v zadnjih letih. 

Članki \cite{CCS,TNNM-HZYLH12,AST-TK15,LMaFit-WY12} opisujejo določene algoritme, ter predstavijo nadgradnjo del in idej pred njimi. Glavne ugotovitve člankov so v tem delu povzete v njihovih razdelkih poglavja \ref{1407-1011}. Ta dela so bila ključna za samo implementacijo algoritmov.

Vodilna literatura tekom pisanja diplomske naloge je bil članek \cite{Survey-NKS19}, ki opisuje problem, ter mnoge algoritme. Medtem ko opisi pogosto niso bili dovolj podrobni, da bi lahko začel algoritme implementirati, je članek ponujal dobro razumljive opise algoritmov, kot tudi navedel vire, ki so pomagali pri implementaciji. Prav tako je članek podal pomembno primerjavo rezultatov različnih algoritmov.
\chapter{Algoritmi}

\section{Pomembe definicije}
Nekatere definicije so uporabljene čez več algoritmov. Z namenom preglednosti, te opisujem v tem poglavju
\begin{enumerate}
  \item $\Omega$ je definirana množica znanih vrednosti
  \item \[ [\proj]_{i,j} = \begin{cases}
            a_{ij} & (i, j) \in \Omega \\
            0      & \textrm{drugače}
          \end{cases}
        \]
  \item Operator $\shrink_\tau$ kot \[
            \shrink_\tau(A) := U \shrink_\tau(\Sigma) V^T, \hspace{0.3cm} \shrink_\tau(\Sigma) = diag(max(\sigma_i - \tau, 0))
        \] \cite{CCS}
  \item Z oznako $M \in \mathbb{R}^{n_1 \times n_2}$ označujemo vhodno matriko, torej tisto, ki ima nekatere podatke neznane. {\vspace{-10cm}\todo{V programu nato uporabljamo oznako M za bitno matriko, vendar te v samih dokazih ne potrebujemo, je potem tukaj oznaka M vredu?}\vspace{10cm}}
  % \item Z oznako $M \in \mathbb{R}^{n_1 \times n_2}$ označujemo bitno matriko M, ki predstavlja masko, s katero označimo katere vrednosti poznamo in katere ne. Vrednost 1 označuje, da je vrednost na tisti poziciji znana, medtem ko 0 označuje, da ni. 
\end{enumerate}

\section{Minimizacija nuklearne norme}
Minimizacija nuklearne norme (Nuclear Norm Minimization oziroma NNM) se zanaša na dejstvo, da je rang matrike povezan z nuklearno normo matrike. Ta je definirana kot

\[
  ||A||_* = \sum_{i = 0}^{n} \sigma_i(A)
\].

Minimizacijo nuklearne norme je možno pretvoriti v semidefinitni problem, ki ga lahko rešujemo z različnimi pripomočki, na primer SeDuMi \cite{SeDuMi}.

Po \cite{CR08} lahko problem definiramo kot

\begin{align*}
  \textrm{min }     & \hspace{0.5cm} tr(Y)                                    \\
  \textrm{tako da } & \hspace{0.5cm} (Y, A_k) = b_k, k = 1, \cdots , |\Omega| \\
                    & \hspace{0.5cm} Y \succcurlyeq 0
\end{align*}

kjer
\[
  Y = \begin{bmatrix}
    W_1 & X   \\
    X^T & W_2
  \end{bmatrix}
\] tak problem pa lahko že rešujemo z semidefinitnimi programi.


\section{Algoritem praga singularnih vrednosti}
\textbf{Algoritem praga singularnih vrednosti (SVT)} \cite{CCS}, uporabi idejo, da imajo matrike z majhnim rangom nekaj velikih singularnih vrednosti, ostale pa 0 ali pa vsaj blizu 0. Ključna parametra v SVT-ju sta \textit{izbira premika} in \textit{izbira praga},  
%Za svoje delovanje uvede dva nova pomembna koncepta, prvi je premik, drugi pa prag, potreben za uporabo operatorja $\shrink_\tau$ \eqref{1007-1959}. 
algoritem pa temelji na iteraciji
\begin{align}
\label{2407-1910}
        X^{(k)} &= \shrink_\tau(Y^{(k-1)}), \\
        Y^{(k)} &= Y^{(k-1)} + \delta_k \proj(M - X^{(k)}), 
\end{align}
kjer so $\tau > 0$ izbran prag, $\delta_k$ izbran premik, $X^{(0)} = 0 \in \mathbb{R}^{n_1 \times n_2}$ in
$Y^{(0)} = 0 \in \mathbb{R}^{n_1 \times n_2}$. \cite{CCS}

V nadaljevanju bomo opisali glavno idejo zgornje iteracije. V grobem pa temelji na uporabi metode za iskanje vezanih ekstremov, kjer elementi matrik $Y^{(k)}$ predstavljajo Lagrangove množitelje. 

Uvedimo funkcijo 
\begin{align}
    \label{1007-2007}
    f_\tau(X) = \tau\nnorm{X} + \frac{1}{2}\fnorm{X}^2
\end{align}
in optimizacijski problem
\begin{align}
\label{2706-0957}
\begin{split}
    \min_{X\in \mathbb R^{n_1\times n_2}} & \hspace{0.5cm} f_\tau(X), \\
    \textrm{pri pogojih} & \hspace{0.5cm} \proj(X) = \proj(M).
\end{split}
\end{align}
Opazimo lahko, da za velike vrednosti $\tau$ velja $f_\tau(X) \approx \tau\nnorm{X}$, kar pomeni, da bo s primerno izbranim $\tau$, optimizacijski problem minimiziral nuklearno normo.

Denimo, da želimo poiskati minimum funkcije $f(x)$ pri pogojih $g_1(x)=g_2(x)=\ldots=g_{k}(x)=0$.
V teoriji vezanih ekstremov se za tovrstne probleme uvede \textbf{Lagrangeovo funkcijo}
\[\mathcal{L}(x, \lambda_1,\lambda_2,\ldots,\lambda_k) = f(x) + \lambda_1 g_1(x)+\lambda_2g_2(x)+\ldots+\lambda_kg_k(x),\]
nato pa išče ekstreme med njenimi stacionarnimi točkami.
Problemu \eqref{2706-0957} lahko priredimo Lagrangeovo funkcijo
\[
    \mathcal{L}(X, Y) = f_\tau(X) + \left< Y, \proj(M - X) \right>,
\] 
nato pa iščemo njene stacionarne točke.
Zaradi velikega števila parametrov pa ta pristop navadno ni izvedljiv, zato se v SVT-ju uporabi za iskanje ekstremov $\mathcal{L}(X, Y)$ t.i.\ \textit{Uzawa algoritem} \cite{CCS}. Ta ekstreme išče prek iterativnega postopka:
\begin{align}
        X^{(k)} &=  \arg \min_{X} \hspace{0.2cm}\mathcal{L}(X^{(k)}, Y^{(k-1)}) \label{1007-2018},\\
        Y^{(k)} &= Y^{(k-1)} + \delta_k \proj(M - X^{(k)}) \label{1007-2019}
\end{align}

\CR{$\arg\min$ ni tako pogost pojem, tako da dodaj definicijo.}

Izkaže se, da je rešitev \eqref{1007-2018} enaka $\shrink_\tau(Y^{k-1})$. To spodaj dokažemo, še prej pa izpeljimo pomožen rezultat.

\begin{trditev}
Velja:
\begin{align}
    \arg \min_X \hspace{0.2cm} \mathcal{L}(X, Y)
    = \arg \min_X \hspace{0.2cm} \tau\nnorm{X} + \frac{1}{2}\fnorm{X - Y}^2
\end{align}
\end{trditev}

\begin{proof}
Trditev sledi iz krajšega računa:
\begin{align*}
    &\arg \min_X \hspace{0.2cm} \tau\nnorm{X} + \frac{1}{2}\fnorm{X - Y}^2 \\
    &=\arg \min_X \hspace{0.2cm} \tau\nnorm{X} + \frac{1}{2}\trOp{X - Y}{X - Y}\\
    &=\arg \min_X \hspace{0.2cm} \tau\nnorm{X} + \frac{1}{2}(\fnorm{X}^2 - 2\trOp{X}{Y} + \fnorm{Y}^2) \\ 
    &=\arg \min_X \hspace{0.2cm} \tau\nnorm{X} + \frac{1}{2}\fnorm{X}^2 - \trOp{X} {\proj(Y)}\\
    &=\arg \min_X \hspace{0.2cm} \tau\nnorm{X} + \frac{1}{2}\fnorm{X}^2 + \tr(- \proj(X)Y^T) + \tr(\proj(M)Y^T)\\
    &=\arg \min_X \hspace{0.2cm} \tau\nnorm{X} + \frac{1}{2}\fnorm{X}^2 + \tr(\proj(M - X)Y^T)\\
    &=\arg \min_X \hspace{0.2cm} \tau\nnorm{X} + \frac{1}{2}\fnorm{X}^2 + \trOp{Y}{\proj(M - X)}\\ 
    &= \arg \min_X \hspace{0.2cm} \mathcal{L}(X, Y)
\end{align*}
kjer smo v prvi enakosti uporabili definicijo Frobeniusove norme, v drugi bilinearnost skalarnega produkta, 
v tretji smo ignorirali konstanto $\fnorm{Y}^2$, saj ne vpliva na rezultat, 
in upoštevali $\proj(Y^{(k)}) = Y^{(k)}$ za vse $k \in \mathbb{N}$. Zadnje dejstvo sledi iz definicije \eqref{1007-2019} in $Y^{(0)} = 0$. V četrti enakosti smo upoštevali $\trOp{X}{\proj(Y)} = \trOp{\proj(X)}{Y}$, kar je lahko videti, ko se spomnimo, da za skalarni produkt $\trOp{A}{B}$ matrik $A, B \in \mathbb{R}^{n_1 \times n_2}$ velja 
\[
    \trOp{A}{B} = \sum_{i = 1}^{n_1}\sum_{j = 1}^{n_2} a_{ij}b_{ij}.
\]
Prišteli smo tudi konstanto $\tr(\proj(M)Y^T)$, ki ne vpliva na rezultat.
V peti enakosti smo upoštevali linearnost sledi in aditivnost operatorja $P_\Omega$,
v šesti definicijo skalarnega produkta in v zadnji definicijo Lagrangeove funkcije.
\end{proof}
\begin{theorem} \label{1907-2240}
\CG{Za matriki $X \in \mathbb{R}^{n_1 \times n_2}, Y \in \mathbb{R}^{n_1 \times n_2}$} velja:
\begin{align}
    \label{2906-1056}
    \shrink_\tau(Y) = \arg \min_{X} \hspace{0.2cm}\frac{1}{2} \fnorm{X-Y}^2 + \tau\nnorm{X} 
\end{align}
\end{theorem}

\begin{proof} 
\todo{Ali potrebujem strogo konveksnost}
\CR{Po čem odvajaš normo?}
\CG{Najprej se spomnimo definicije konveksne funkcije. Funkcija $f$ je konveksna, če za katerikoli dve točki $x_1, x_2$ v domeni funkcije $f$ velja, da je premica čez ti dve točki na odseku med tema dvema točkama večja ali enaka funkciji $f$.}
\CG{Funkcija $h(X) := \frac{1}{2} \fnorm{X-Y}^2 + \tau\nnorm{X} $ je konveksna funkcija, saj potreben pogoj trikotniške neenakosti matričnih norm zagotavlja, da je matrična norma konveksna funkcija. Vsota konveksnih funkcij pa je prav tako konveksna funkcija. \todo{ali je potrebno citirati} Zaradi konveksnosti lahko tako uporabimo definicijo subgradienta $Z$ v točki $X_0$. Ta je definiran kot: \[\forall X: f(X) \geq  f(X_0) + \trOp{Z}{X - X_0}\] Ali drugače povedano, premica na točko $X_0$ s smernim koeficientom $Z$ je povsod na ali pod funkcijo $h(X)$.}

Pri iskanju minimuma torej iščemo tako točko $X'$, da bo
\CG{eden izmed subgradientov po spremenljivki $X$} v točki $X'$ enak 0.  Izkaže se \cite{CCS}, da je množica subgradientov nuklearne norme definirana kot
\[
    \partial\nnorm{X} = \{UV^* + W: W \in \mathbb{R}^{n_1 \times n_2}, U^*W = 0, WV = 0, \norm{W}_2 \leq 1 \}.
\]
kjer $U \Sigma V^T$ predstavlja SVD razcep matrike $X$. \CG{Prav tako pa vemo, da velja $\frac{\partial}{\partial X}\fnorm{X-Y}^2 = 2(X - Y)$.} \todo{kaj citiram, ali dokazem?} Problem sedaj zapišemo kot $0 \in X' - Y + \tau \partial\nnorm{X'}$. 

Trditev izreka bo sledila, če pokažemo, da velja $X' = \shrink_\tau(Y)$. Najprej razčlenimo SVD razcep matrike $Y$ kot 
\[
    Y = U_0\Sigma_0V_0^T + U_1\Sigma_1V_1^T
\]
kjer $U_0, \Sigma_0$ in $V_0$ predstavljajo singularne vrednosti in pripadajoče singularne vektorje večje od $\tau$, $U_1, \Sigma_1$ in $V_1$ pa tiste manjše od $\tau$. Pokazati želimo, da velja 
\[
    X' = U_0(\Sigma_0 - \tau I)V_0^T.
\] 
\todo{ali moram to pokazati}V nadaljevanju bomo videli, da je to samo alternativen zapis operatorja $\shrink_\tau(Y)$.
\CR{Tega nadaljevanja povsem ne razumem. Kako prideš do preoblikovanja, prek katerega iščeš matriko $W$, mi ni jasno.}
Če zapis vstavimo v prejšnji podan pogoj dobimo
\begin{align*}
    0 = X' - Y + \tau \partial \nnorm{X'}\\
    Y- X' = \tau (U_0 V_0^T + W)
\end{align*}
primerna izbira za $W = \tau^{-1} U_1 \Sigma_1 V_1^T$, saj
\begin{align*}
    Y-X' &= U_0\Sigma_0V_0^T + U_1\Sigma_1V_1^T - U_0(\Sigma_0 - \tau I)V_0^T \\ 
    &= U_0V_0^T(\Sigma_0 - \Sigma_0 + \tau I) + U_1\Sigma_1 V_1^T  \\
    &= \tau U_0 V_0^T + U_1\Sigma_1 V_1^T
\end{align*}
in 
\begin{align*}
    \tau(U_0 V_0^T + W) &= \tau(U_0V_0^T + \tau^{-1} U_1 \Sigma_1 V_1^T)\\ 
    &= \tau U_0 V_0^T + U_1 \Sigma_1 V_1^T 
\end{align*}

Sedaj je zgolj potrebno pokazati, da veljajo potrebne lastnosti matrike $W$.
Po sami definiciji SVD vemo, da so vsi stolpci matrik U in V ortogonalni. Torej velja $U_0^TW = 0$ in $WV_0 = 0$. Ker pa ima matrika $\Sigma_1$ vse elemente manjše od $\tau$ velja tudi $\norm{W}_2 \leq 1$. \CG{$\norm{A}_2$ je namreč definirana kot največja singularna vrednost matrike $A$}. S tem smo pokazali, da $Y - X' \in \tau \partial \nnorm{X'}$.
\end{proof}
Z uporabo \eqref{1007-2018}, \eqref{1007-2019}
in izreka \ref{1907-2240}
res pridemo do iteracije
\eqref{2407-1910}, ki jo uporablja algoritem SVT.
\iffalse
Tako res pridemo 
Po trditvi lahko sedaj zapišemo algoritem \eqref{1007-2018} - \eqref{1007-2019} kot \cite{CCS}
\[
    \begin{cases}
        X^k = \shrink_\tau(Y^{k-1}) \\
        Y^k = Y^{k-1} + \delta_k \proj(M - X^k) 
    \end{cases}
\]
\fi

\subsection{Nastavljanje parametrov $\tau$ in $\delta$} \label{1907-1648}
Opazimo lahko, da algoritem SVT potrebuje dva parametra, $\tau$ in $\delta$, ki ju moramo izbrati 
že pred vstopom v algoritem.

Po priporočilih \cite{CCS}
sta primerni izbiri za $\delta$ in $\tau$
enaki 
\[
    \delta = 1.2\, \dfrac{n_1 n_2}{m}\qquad\text{in}\qquad
    \tau = 5n,
\]
pri čemer je $\tau$ naveden za kvadratne $n\times n$ matrike.
V poglavju z rezultati smo prvotno uporabljali ti dve konstanti, pri 
čemer smo zaradi
pravokotnosti $n_1\times n_2$ matrik uporabljali
$\tau = 5\frac{n_1+n_2}{2}$.
\iffalse
Medtem, ko so koraki v samem algoritmu definirani kot množica korakov, 
smo v okviru rezultatov diplomske naloge, prvotno za premik uporabljali konstanto, ter korak nastavili na 
 po priporočilih \cite{CCS}. 
 \fi
\iffalse
Prav tako članek \cite{CCS} navaja, da je za matrike velikosti $\mathbb{R}^{n \times n}$ smiselno nastaviti $\tau = 5n$, vendar sem v moji implementaciji zaradi posploševanja na nekvadratne matrike, za matrike velikosti $\mathbb{R}^{n_1 \times n_2}$ parameter nastavil na
\[
    \tau = 5\, \frac{n_1+n_2}{2}
\]
\fi
V naših testiranjih se je izkazalo, da sta taka parametra dobra za večje matrike. V razdelku \ref{1307-2251} pa bomo videli, da moramo pri manjših matrikah pogosto zmanjšati premik in povečati prag.
\section{Algoritem minimizacije prirezane nuklearne norme} \label{2807-1442}
Kot nam že samo ime pove, je \textbf{algoritem minimizacije prirezane nuklearne norme (TNNM)} \cite{TNNM-HZYLH12} soroden algoritmu NNM. Dodatna informacija, ki pa jo uporabi TNNM, je rang $r$ originalne, nezašumljene matrike.

Osrednjo vlogo v algoritmu ima \textbf{$r$-prirezana nuklearna norma}, ki za dano matriko $X \in \mathbb{R}^{n_1 \times n_2}$ vrne vsoto njenih $\min(n_1,n_2) - r$ najmanjših singularnih vrednosti:
\[
    \norm{X}_r = \sum^{\min(n_1, n_2)}_{i = r + 1} \sigma_i(X).
\]
TNNM rešuje optimizacijski problem
\begin{align}
    \label{1107-1305}
    \begin{split}
        \min_{X\in  \mathbb{R}^{n_1 \times n_2}}              & \hspace{0.5cm} \norm{X}_r, \\
        \textrm{pri pogoju} & \hspace{0.5cm} \proj(X) = \proj(M).
    \end{split}
\end{align}
Cilj algoritma je torej čim bolj zmanjšati najmanjše singularne vrednosti, medtem ko velikih ne omejujemo. S tem problem minimizacije omilimo.

Problem \eqref{1107-1305} lahko zapišemo v ekvivalentni obliki
\begin{align}
    \label{2507-0835}
    \begin{split}
        \min_{X\in  \mathbb{R}^{n_1 \times n_2}}              & \hspace{0.5cm} \nnorm{X} - \sum_{i=1}^{r} \sigma_i(X), \\
        \textrm{pri pogojih} & \hspace{0.5cm} \proj(X) = \proj(M).
    \end{split}
\end{align}
V izpeljavah, ki sledijo, bomo potrebovali naslednji izrek.

\begin{theorem}
    \label{2507-0850}
    Za matrike $X \in \mathbb{R}^{n_1 \times n_2}$, $A \in \mathbb{R}^{r \times n_1}$ in $B \in \mathbb{R}^{r \times n_2}$,
    ter naravno število $r \in \mathbb{N}$, ki zadoščajo pogojem $r \leq \min(n_1, n_2)$, $AA^T = I_{r}$  in $BB^T = I_{r}$, velja neenakost
    \begin{align}
        \label{2507-0836}
        \tr(AXB^T) \leq \sum_{i=1}^{r} \sigma_i(X)
    \end{align}
\end{theorem}

\begin{proof}
    Velja
    \begin{align}
        \label{2507-0839}
        \tr(AXB^T) = \tr(XB^TA) \leq \sum^{\min(n_1, n_2)}_{i=1} \sigma_i(X) \sigma_i(B^TA),
    \end{align}
    kjer smo v enakosti uporabili komutativnost $\tr(ZW)=\tr(WZ)$ sledi,
    v neenakosti pa von Neumannovo neenakost za sled \cite{TNNM-HZYLH12}.

    Po definiciji so singularne vrednosti matrike $Y$ enake korenom lastnih vrednosti matrike $Y^TY$. Torej so singularne vrednosti matrike $B^TA$ enake korenom lastnih vrednosti matrike $A^TBB^TA=A^TI_rA = A^TA$.
    %https://cutt.ly/UwaPOfaK
    Ker imata matriki $XY$ in $YX$ enake neničelne lastne vrednosti in po predpostavki velja $AA^T = I_r$, lahko
    od tod sklepamo, da ima produkt $B^TA$ $r$ singularnih vrednosti enakih 1.
    Zato velja
    \[
        \sum^{\min(n_1, n_2)}_{i=1} \sigma_i(X) \sigma_i(B^TA) = \sum^{r}_{i=1} \sigma_i(X),
    \]
    kar skupaj z \eqref{2507-0839}
    dokaže neenakost \eqref{2507-0836}
    v izreku.
\end{proof}

\begin{theorem}
    \label{2507-0851}
    Naj bo $X = U \Sigma V^T$
    SVD razcep matrike $X$.
    Naj bosta $A$ in $B$ matriki, sestavljeni
    iz prvih $r$ stolpcev matrik $U$ in $V$.
    Velja
    \[\tr(AXB^T) = \sum^{r}_{i=1} \sigma_i(X).\]
\end{theorem}

\begin{proof}
    Označimo matriki $A$ in $B$ z
    $A = (u_1, \hdots , u_r)^T$ in $B = (v_1, \hdots , v_r)^T$,
    kjer je $u_i$ $i$-ti stolpec matrike $U$ ter $v_i$ $i$-ti stolpec
    matrike $V$.
    \begin{align*}
        \tr(AXB^T) & = \tr\big((u_1, \hdots , u_r)^T X (u_1, \hdots , u_r)^T\big)                                                                                                      \\
                   & = \tr\big((u_1, \hdots , u_r)^T U \Sigma V^T (u_1, \hdots , u_r)^T\big)                                                                                           \\
                   & =\tr\Big( \begin{bmatrix} I_r & 0 \\ 0 & 0 \end{bmatrix} \Sigma \begin{bmatrix} I_r & 0 \\ 0 & 0 \end{bmatrix}\Big)                                               \\
                   & = \tr\Big(\begin{bmatrix} \sigma_1 \\[-8pt] & \ddots & & \\[-8pt] & & \sigma_r \\[-8pt] & &  & 0 \\[-8pt]  & & & & \ddots \\[-8pt] & & & & & 0 \end{bmatrix}\Big) \\
                   & = \sum_{i = 1}^{r} \sigma_i(X),
    \end{align*}
    kar dokaže trditev izreka.
\end{proof}
Po izrekih \ref{2507-0850} in \ref{2507-0851}
velja
\[
    \max_{
        \substack{AA^T = I,\\ BB^T = I}} \tr(AXB^T) = \sum^{r}_{i = 1} \sigma_i(X)
\]
Torej je optimizacijski problem
\eqref{2507-0835} ekvivalenten
problemu
\begin{align*}
    \min_{X\in  \mathbb{R}^{n_1 \times n_2}}\hspace{0.5cm} & \nnorm{X} - \max_{\substack{AA^T = I, \\ BB^T = I}} \tr(AXB^T), \\
    \text{pri pogoju} \hspace{0.5cm}                       & \proj(X) = \proj(M).
\end{align*}

Sedaj lahko opišemo idejo algoritma TNNM:
\begin{enumerate}
    \item Izračunamo $X^{(0)} = \proj(M)$.
    \item Za $k=0,1,2,\ldots$ ponavljamo iteracijo:
          \begin{enumerate}
              \item Izračunamo SVD razcep
                    $X^{(k)} = U^{(k)} \Sigma^{(k)} (V^{(k)})^T$.
              \item $A^{(k)}$ definiramo kot prvih    $r$ stolpcev matrike $U^{(k)}$,
                    $B^{(k)}$ pa kot prvih
                    $r$ stolpcev matrike $V^{(k)}$.
              \item $X^{(k+1)}$ je enak rešitvi
                    optimizacijskega problema
                    \begin{align}
                        \label{2507-0900}
                        \begin{split}
                            \min_{X\in  \mathbb{R}^{n_1 \times n_2}} \hspace{0.5cm}         & \nnorm{X} - \tr(A^{(k)}X(B^{(k)})^T), \\
                            \text{pri pogoju} \hspace{0.5cm} & \proj(X) = \proj(M).
                        \end{split}
                    \end{align}
          \end{enumerate}
\end{enumerate}
Z nadaljnjim preoblikovanjem problema
\eqref{2507-0900} v ekvivalentnega
\begin{align}
    \label{1307-1527}
    \begin{split}
        \min_{X\in  \mathbb{R}^{n_1 \times n_2}}  \hspace{0.5cm}         & \nnorm{X} - \tr(A^{(k)}W(B^{(k)})^T),  \\
        \text{pri pogojih} \hspace{0.5cm} & W=X, \hspace{0.2cm} \proj(W) = \proj(M),
    \end{split}
\end{align}
lahko za reševanje uporabimo \textbf{algoritem ADMM} \cite{TNNM-HZYLH12}.

Gre za reševanje problema vezanih ekstremov, ki ga lahko zapišemo s pomočjo Lagrangeove funkcije, pri čemer algoritem ADMM doda še
člen, pomnožen z \textit{regularizacijskim parametrom} $\beta$.
\[
    \mathcal{L}(X, Y, W, \beta) = \nnorm{X} - \tr(A W B^T) + \frac{\beta}{2} \fnorm{X - W}^2 + \tr(Y^T(X-W)).
\]
Matriki $A$ in $B$ sta v okviru algoritma ADMM konstantni. Ti na začetku nastavimo na vrednost $A^{(k)}$ in $B^{(k)}$ iz prejšnje iteracije. Opazimo lahko, da funkcija upošteva zgolj pogoj $X = W$. Videli bomo, da algoritem pogoj $\proj(X) = \proj(M)$ definira posredno, s popravljanjem znanih vrednosti (korak \eqref{2906-1248} spodaj).%\todo{ali lahko citiram naprej}

Matriko $X^{(k+1)}$ definiramo kot
\[
    X^{(k+1)} = \argmin_X \mathcal{L}(X, Y^{(k)}, W^{(k)}, \beta)
\]
Z ignoriranjem konstantnih členov, pa lahko zapišemo
\begin{align*}
    X^{(k+1)} & = \argmin_X \Big( \nnorm{X} + \frac{\beta}{2}\trOp{X-W^{(k)}}{X-W^{(k)}} + \frac{\beta}{2}\trOp{\frac{2}{\beta}Y}{X} \Big) \\
              & = \argmin_X \Big( \nnorm{X} + \frac{\beta}{2}\tr\Big(X^TX - X^TW^{(k)} -                                                   \\
              & \hspace{2cm} (W^{(k)})^TX + (W^{(k)})^TW^{(k)} + \frac{2}{\beta}(Y^{(k)})^TX \Big)\Big).
\end{align*}
Z namenom faktorizacije dodamo konstantne člene
\[-(W^{(k)})^T\frac{1}{\beta}(Y^{(k)}) + (\frac{1}{\beta}(Y^{(k)}))^T(-W^{(k)} + \frac{1}{\beta}(Y^{(k)}))\]
in dobimo
\begin{align*}
     & \argmin_X \Big( \nnorm{X} + \frac{\beta}{2}\tr\Big(X^T\big(X - W^{(k)} + \frac{1}{\beta}Y^{(k)}\big) - (W^{(k)})^T \big(X - W^{(k)} + \frac{1}{\beta}Y^{(k)}\big) \\
     & \hspace{2cm} +\frac{1}{\beta}\big(Y^{(k)})^T(X - W^{(k)} + \frac{1}{\beta}Y^{(k)}\big)\Big) \Big)                                                                 \\
     & = \argmin_X \Big( \nnorm{X} + \frac{\beta}{2} \fnorm{X-W^{(k)} + \frac{1}{\beta}Y^{(k)}}^2 \Big)                                                                        \\
     & = \argmin_X \Big( \frac{1}{\beta}\nnorm{X} +  \frac{1}{2}\fnorm{X-(W^{(k)} - \frac{1}{\beta}Y^{(k)})}^2\Big)  
\end{align*}
Po izreku \ref{1907-2240} \CG{uporabljenem za $X=X^{(k+1)}$, $Y=W^{(k)} - \frac{1}{\beta}Y^{(k)}$ in $\tau=\frac{1}{\beta}$} pa sledi
\[
    X^{(k+1)} = \shrink_\frac{1}{\beta}(W^{(k)} - \frac{1}{\beta} Y^{(k)}).
\]
Matriko $W^{(k+1)}$ podobno izračunamo kot
\begin{align*}
    W^{(k+1)} & = \argmin_{W} \mathcal{L}(X^{(k+1)}, Y^{(k)}, W, \beta)                                   \\
              & = \argmin_W \frac{\beta}{2} \fnorm{W - (X^{(k+1)} + \frac{1}{\beta}(A^T B + Y^{(k)})) }^2
\end{align*}
Očitno je
\[
    W^{(k+1)} = X^{(k+1)} + \frac{1}{\beta}(A^T B + Y^{(k)})
\]
Sedaj uporabimo še pogoj $\proj(W^{(k+1)}) = \proj(M)$ in tiste elemente, ki jih poznamo, popravimo.
\begin{equation}
    \label{2906-1248}
    W^{(k+1)} = W^{(k+1)} + \proj(M - W^{(k+1)}).
\end{equation}
Z besedami, predpis \eqref{2906-1248}
preprosto spremeni mesta $W^{(k+1)}$, kjer vrednosti poznamo, na znane vrednosti, ostalih pa ne spremeni.

\CG{
Po algoritmu ADMM \cite{admmForNNM} na koncu iteracije preprosto posodobimo matriko $Y$, kot
\[
    Y^{(k+1)} = Y^{(k)} + \beta(X^{(k+1) - W^{k+1}}).
\]
}

\section{Izmenjajoč gradientni spust}
Ker je računanje SVD razcepa zahtevna operacija, saj ima časovno zahtevnost
$O(n^3)$, je bilo predlaganih nekaj algoritmov, ki za svoje delovanje ne potrebujejo SVD-ja. Algoritem Izmenjajočega gradientnega spusta, oziroma v nadaljevnu ASD (Alternating Steepest Descent) sloni na računanju gradienta in premikanju po njem. Glavni cilj algoritma je, najti dve matriki $X \in \mathbb{R}^{n_1 \times r}$ ter $Y \in \mathbb{R}^{r \times n_2}$, tako da bo veljalo $M = XY$. Vidimo lahko, da ponovno potrebujemo informacijo o rangu matrike, ki jo rekonstruiramo. Ker imata tako $X$ in $Y$ kvečjemu rang $r$, vemo, da tudi njun produkt $XY$ ne bo imel ranga večjega od $r$. 

Cilj algoritma je minimizirati \todo{zakaj 1/2}
\[
    \min_{X, Y} \hspace{0.5cm} \frac{1}{2}\, ||\proj(M) - \proj(XY)||^2_F
\] 
Algoritem minimizacijo razdeli na dve, nato pa izmenično rešuje eno in nato drugo kot 
\begin{align*}
    X_{i+1} &= \arg \min_{X} \hspace{0.3cm} ||\proj(M) - \proj(XY_i)||^2_F \\
    Y_{i+1} &= \arg \min_{Y} \hspace{0.3cm} ||\proj(M) - \proj(X_{i+1}Y)||^2_F
\end{align*}
\cite{AST-TK15}

Za uporabo algoritma ASD potrebujemo odvoda funkcije $f(X,Y) = \frac{1}{2}\, ||\proj(M) - \proj(XY)||^2_F$, ki ga izračunamo za vsak element posebej.
\todo{Zakaj pride $Y^T$ \href{https://math.stackexchange.com/questions/2128462/gradient-of-squared-frobenius-norm-of-a-matrix}{math stackexchange}} 
\begin{align*}
    \frac{\partial}{\partial x_{a,b}} f &= \frac{\partial}{\partial x_{a,b}} \frac{1}{2}\, ||\proj(M) - \proj(XY)||^2_F = \\
    &= \frac{\partial}{\partial x_{a,b}} \frac{1}{2}\, \sum_{i}^{n_1}\sum_{j}^{n_2}(\delta_{i,j}m_{i,j} - \delta_{i,j}\sum_{k}^{r}(x_{i,k}y_{k,j}))^2 = \\
    &= \, \sum_{j}^{n_2}(\delta_{a,j}m_{a,j} - \delta_{i,j}\sum_{k}^{r}(x_{a,k}y_{k,j}))(-y_{b,j}) \implies \\
    &\implies \frac{\partial}{\partial X}f = -(\proj(M) - \proj(XY))Y^T
    % &= \sum_{i,j}(\delta_{i,j}m_{i,j} - \delta_{i,j}\sum_{k}(x_{i,k}y_{k,j}))
    % (\sum_{k}y_{k,j}) = \\
    % &= (P(M) - P(XY))Y^T
\end{align*}
kjer 
\[
    \delta_{i,j} = \begin{cases}
        1, (i, j) \in \Omega \\
        0, (i, j) \notin \Omega
    \end{cases}
\]
Na podoben način bi lahko pokazali tudi
\[
    \frac{\partial}{\partial Y} = -X^T (\proj(M) - \proj(XY))
\]
Poiščimo še najboljši korak gradientnega spusta.
Cilj je pokazati, da je premik po gradientu s korakom $t_x$ najboljši.
Najprej definirajmo sam premik, kot
\[
    X^{k+1} = X^k - t_X \nabla f_Y(X)
\]
Ker želimo, da bodo znane vrednosti produkta $X^{k+1}Y^{k}$ kar se da podobne znanim vrednostim matrike $M$, nastavimo $t_x$ kot 
\begin{align*}
    t_x &= \arg \min_t \hspace{0.3cm} g(t)
\end{align*} kjer
\begin{align*}
    g(t) &= \frac{1}{2} ||\proj(M) - \proj((X - t\nabla f_Y(X))Y)||_F^2
\end{align*}
Ker je \cite{AST-TK15} \todo{ali lahko tako?}
\[
  g'(t) = -||\nabla f_Y(X)||_F^2 + t||\proj(f_Y(X)Y)||_F^2 
\]
vidimo, da funkcija $g(t)$ doseže minimum pri 
\[
  t_x = \frac{||\nabla f_Y(X)||_F^2}{||\proj(\nabla f_Y(X)Y)||_F^2}  
\]
Podobno velja za korak v smeri gradientnega spusta matrike $Y$, kjer
\[
  t_y = \frac{||\nabla f_X(Y)||_F^2}{||\proj(X \nabla f_X(Y))||_F^2}  
\]

% \begin{align*}
%     g'(t) &=  \proj(M - XY) \proj(X'Y)^T = \\
%     &= \proj(M - XY) \proj((\proj(XY) - \proj(M))Y^T Y)^T = \\
%     &= \proj(M - XY) (\proj(\proj(XY)Y^TY)^T - \proj(\proj(M)Y^TY)^T) = \\
%     &= (\proj(M) - \proj(XY)) (\proj(\proj(XY)Y^TY)^T - \proj(\proj(M)Y^TY)^T) = \\
%     &= \proj(M)\proj(\proj(XY)Y^TY)^T - \proj(M)\proj(\proj(M)Y^TY)^T \\ &- \proj(XY)\proj(\proj(XY)Y^TY)^T + \proj(XY)\proj(\proj(M)Y^TY)^T = \\
%     &= (\proj(M) - \proj(XY))\proj(\proj(XY)Y^TY)^T + (\proj(XY) - \proj(M))(\proj(\proj(M)Y^TY)^T)
% \end{align*}

% % \begin{align*}
% %     g'(t) &= \sum_{i}^{n_1}\sum_{j}^{n_2}(m_{i,j} - \sum_{k}^{r} x_{i,r}y_{r,j} + t \sum_{k}^{r}f_{i,r}y_{r,j})(\sum_{k}^{r}f_{i,r}y_{r,j}) = \\
% % \end{align*}

% % &= \tr(\proj(M - XY) \proj(-(\proj(M) - \proj(XY))Y^T Y)^T) + t||X'Y||^2_F = \\
% % &= -\tr(\proj(M - XY) \proj((\proj(M)Y^T - \proj(XY)Y^T) Y)^T) + t||X'Y||^2_F = \\
% % &= -\tr(\proj(M - XY) \proj(Y^T(\proj(M)Y^T - \proj(XY)Y^T)^T)) + t||X'Y||^2_F = \\
\section{LMaFit}
LMaFit je algoritem, ki podobno kot ASD, rešuje problem matričnih napolnitev z uporabo dveh manjših matrik $X^{n_1 \times r}$ in $Y^{n_1 \times r}$. Podobno kot v prejšnjih algoritmih rešujemo problem 
\begin{align*}
    \min_{X, Y, Z}& \hspace{0.5cm} \frac{1}{2}\fnorm{XY - Z}^2\\
    \text{pri pogojih}& \hspace{0.5cm} \proj(M) = \proj(Z)
\end{align*}
le da sam problem rešujemo s pomočjo Moore-Penrose inverza (Označen z $\dagger$).

Lahko je videti, da bo funkcija $f(X,Y,Z) = \frac{1}{2}\fnorm{XY - Z}^2$ imela najmanjšo vrednost, ko bo $XY = Z$. Zato uvedemo iterativni algoritem, ki izmenično posodablja $X, Y$ in $Z$ tako, da fiksira dve izmed matrik. Minimizacijo matrik $X$ in $Y$ torej dosežemo na naslednji način.
\begin{align*}
    X^{i+1}Y^{i} &= Z^{i} \hspace{0.5cm} \\
    X^{i+1} &= Z^{i}(Y^i)^\dagger
\end{align*}
Kjer se zanašamo na dejstvo, da nam množenje z Moore-Penrose inverzom z desne 
da najboljši možen rezultat.
Podobno posodobimo matriko $Y$.
\begin{align*}
    \hspace{0.5cm} X^{i+1}Y^{i+1} &= Z^{i}   \\
    Y^{i+1} &= (X^{i+1})^\dagger Z^{i}
\end{align*}

Sedaj le še posodobimo matriko Z, kot
\[
    Z^{i+1} = X^{i+1}Y^{i+1} + \proj(M - X^{i+1}Y^{i+1})
\]
Torej podobno kot prej poskušamo doseči $XY = Z$, vendar zaradi omejitve znanih vrednosti matrike M, na tistem mestu vrednosti popravimo. \cite{LMaFit-WY12}


\chapter{Rezultati}
V tem poglavju bom opisal rezultate, ki sem jih pridobil. Kot sem že omenil, bo večji del preizkušanja programa opravljen na slikah, kjer bodo nekateri piksli manjkali. Gre za problem, ki ga je moč lepo vizualizirati, saj pogosto pri surovih podatkih ni lahko definirati njihovo točnost, zaradi česar težko interpretiramo, kako koristen je sam algoritem.

Prav tako bom opisal točnost rezulatov različnih metod kot tudi čas izvajanja posameznih metod. Probleme bom zagnal tudi na različnih vrst podatkov, npr. podatkih ki so generirani normalno kot tudi enakomerno porazdeljeno.
Zaradi interpretacije, slike razdelimo v več skupin, za katere opišemo ugotovitve.

Vse slike si je mogoče v boljši kakovosti ogledati na povezavi \url{https://tinyurl.com/yb7cjdv7}. 

\section{Velika črno-bela slika}
Same teste algoritmov najprej poženemo na veliki, črno-beli sliki, velikosti $1000\times1000$ pikslov. Taka izbira je smiselna, iz vidika, da imamo dovolj podatkov, potrebnih za rekonstrukcijo. Ker je časovna zahtevnost algoritmov pri večjih slikah, kot bomo videli v nadaljevanju že precej velika, nam ta faza testiranja služi kot preverjanje delovanja samih algoritmov. Same podrobnosti razlik med rezultati si bomo zato podrobneje pogledali na manjših slikah v nadaljevanju. Algoritme preizkušamo trikrat, na podatkih z deleži znanih vrednosti $0.35, 0.45$ in $0.6$. \todo{Smiselna postavitev teskst-slik in velikosti}
\renewcommand{\mapa}{Poglavja/Slike/grayscale1000}

\begin{figure}[!ht]
    \centering
    \begin{subfigure}{0.49\linewidth}
        \includegraphics[width=\linewidth]{\mapa/slikaInput.png}
        \caption{Originalna slika.}
    \end{subfigure}
    \hfill
    \begin{subfigure}{0.49\linewidth}
        \includegraphics[width=\linewidth]{\mapa/slikaInput35.png}
        \caption{Slika s $35\%$ znanimi podatki.}
    \end{subfigure}
    \begin{subfigure}{0.49\linewidth}
        \includegraphics[width=\linewidth]{\mapa/slikaInput45.png}
        \caption{Slika s $45\%$ znanimi podatki.}
    \end{subfigure}
    \hfill
    \begin{subfigure}{0.49\linewidth}
        \includegraphics[width=\linewidth]{\mapa/slikaInput60.png}
        \caption{Slika s $60\%$ znanimi podatki.}
    \end{subfigure}
    \caption{Slika, uporabljena za rekonstrukcijo. Vir slike: \cite{UnsplashGora}.}
\end{figure}

\begin{figure}[!ht]
    \centering
    \begin{subfigure}{0.325\linewidth}
        \includegraphics[width=\linewidth]{\mapa/slikaRez35SVT.png}
        \caption{SVT, $35\%$}
    \end{subfigure}
    \hfill
    \begin{subfigure}{0.325\linewidth}
        \includegraphics[width=\linewidth]{\mapa/slikaRez45SVT.png}
        \caption{SVT, $45\%$}
    \end{subfigure}
    \hfill
    \begin{subfigure}{0.325\linewidth}
        \includegraphics[width=\linewidth]{\mapa/slikaRez60SVT.png}
        \caption{SVT, $60\%$}
    \end{subfigure}
    \begin{subfigure}{0.325\linewidth}
        \includegraphics[width=\linewidth]{\mapa/slikaRez35TNNM.png}
        \caption{TNNM, $35\%$}
    \end{subfigure}
    \hfill
    \begin{subfigure}{0.325\linewidth}
        \includegraphics[width=\linewidth]{\mapa/slikaRez45TNNM.png}
        \caption{TNNM, $45\%$}
    \end{subfigure}
    \hfill
    \begin{subfigure}{0.325\linewidth}
        \includegraphics[width=\linewidth]{\mapa/slikaRez60TNNM.png}
        \caption{TNNM, $60\%$}
    \end{subfigure}
    \begin{subfigure}{0.325\linewidth}
        \includegraphics[width=\linewidth]{\mapa/slikaRez35ASD400.png}
        \caption{ASD, $35\%$}
    \end{subfigure}
    \hfill
    \begin{subfigure}{0.325\linewidth}
        \includegraphics[width=\linewidth]{\mapa/slikaRez45ASD600.png}
        \caption{ASD, $45\%$}
    \end{subfigure}
    \begin{subfigure}{0.325\linewidth}
        %ASD 60?%
        \hfill
    \end{subfigure}
    \begin{subfigure}{0.325\linewidth}
        \includegraphics[width=\linewidth]{\mapa/slikaRez35LmaFIT50.png}
        \caption{LMaFit, $35\%$}
    \end{subfigure}
    \hfill
    \begin{subfigure}{0.325\linewidth}
        \includegraphics[width=\linewidth]{\mapa/slikaRez45LmaFIT73.png}
        \caption{LMaFit, $45\%$}
    \end{subfigure}
    \begin{subfigure}{0.325\linewidth}
        \includegraphics[width=\linewidth]{\mapa/slikaRez60LmaFIT77.png}
        \caption{LMaFit, $60\%$}
    \end{subfigure}
    \caption{Rekonstrukcija zašumljenih slik z uporabo različnih algoritmov in pri različnih odstotkih znanih vrednosti. Kratica pod sliko označuje uporabljen algoritem, odstotek za vejico pa odstotek znanih vrednosti.}
\end{figure}

\FloatBarrier

\begin{table}[h]
    \centering
    \begin{tabular}{|c|c|c|c|c|}
        \hline
        \diagbox{OZP}{Algoritem}
             & SVT                & TNNM               & LMAFIT             & ASD                  \\ \hline
        35\% & $4.69 \times 10^4$ & $7.70 \times 10^3$ & $7.50 \times 10^3$ & $3.97 \times 10^7$ \\ \hline
        45\% & $3.15 \times 10^4$ & $5.30 \times 10^3$ & $5.87 \times 10^3$ & $6.09 \times 10^7$ \\ \hline
        60\% & $1.25 \times 10^4$ & $3.58 \times 10^3$ & $4.20 \times 10^3$ & -                    \\ \hline
    \end{tabular}
    \caption{Napake algoritmov, izračunane v Frobeniusovi normi. Kratica OZP stoji za odstotek znanih podatkov.
    % \CR{Kratico OZP sem umaknil iz seznama uporabljenih kratic. Tam je bolj smiselno dati uveljavljene kratice za algoritme. OZP si sam uvedel in kar v vsako tabeli spodaj dodaj v opis zadnji stavek s pomenom OZP.}
    }
\end{table}
\begin{figure}[!ht]
    \centering
    \includegraphics[width=\linewidth]{Poglavja/Slike/grayscale1000/grafNapake.png}
    \caption{Napake algoritmov v Frobeniusovi normi. Na abscisni osi so deleži znanih podatkov slik.}
\end{figure}

\begin{table}[h]
    \centering
    \begin{tabular}{|c|c|c|c|c|}
        \hline
        \diagbox{OZP}{Algoritem}
             & SVT   & TNNM & LMAFIT & ASD   \\ \hline
        35\% & 338s  & 824s & 275s   & 1012s \\ \hline
        45\% & 510s  & 498s & 248s   & 328s  \\ \hline
        60\% & 1674s & 350s & 45.6s  & -     \\ \hline
    \end{tabular}
    \caption{Časi do dosega zaustavitvenega pogoja. Kratica OZP stoji za odstotek znanih podatkov.}
\end{table}
\begin{figure}[!ht]
    \centering
    \includegraphics[width=\linewidth]{Poglavja/Slike/grayscale1000/grafCas.png}
    \caption{Časi izvajanja algoritmov. Na abscisni osi so deleži znanih podatkov slik.}
\end{figure}
\FloatBarrier

Kot vidimo, je med rezultati velika razlika. Očitno je, da algoritmi TNNM, SVT in LMaFit delujejo najbolje, medtem ko ima algoritem ADDM vprašljive rezultate. Te si lahko interpretiramo kot posledico lastnosti, da lahko algoritem končata v lokalnem minimumu. Prav tako algoritem ADMM ni našel rešitve, ko je imel poznanih $0.60$ podatkov. Zato je ta algoritma smiselno uporabljati, kadar imamo dober začeten približek matrik $X$ in $Y$ ter manj poznanih vrednosti. Algoritem NNM smo med rezultati izpustili, saj je zaradi velikega števila matrik, potrebnih za definicijo omejitev, algoritem preveč prostorsko kompleksen. Ta algoritem bomo zato obravnavali posebej. Zaradi teh opazk se v naslednjih podpoglavjih v večini osredotočamo na algoritme SVT, TNNM in LMaFit.
\todo{je potrebno in graf in tabelo?}
\begin{table}[h]
    \centering
    \begin{tabular}{|c|c|c|c|c|}
    \hline
    & SVT & TNNM & LMAFIT & ASD \\ \hline
    0.35 & $4.69 \times 10^4$ & $7.70 \times 10^3$ & $8.03 \times 10^3$ & $3.9743 \times 10^7$ \\ \hline
    0.45 & $3.15 \times 10^4$ & $5.30 \times 10^3$ & $6.40 \times 10^3$ & $6.0910 \times 10^7$ \\ \hline
    0.6 & $1.25 \times 10^4$ & $3.58 \times 10^3$ & $5.35 \times 10^3$ & - \\ \hline
    \end{tabular}
\end{table}
\begin{figure}[!ht]
    \centering
    \includegraphics[width=\linewidth]{Poglavja/Slike/grayscale1000/grafNapake.png}
    \caption{Napake algoritmov glede na delež znanih vrednosti}
\end{figure}

\begin{table}[h]
    \centering
    \begin{tabular}{|c|c|c|c|c|}
    \hline
    & SVT & TNNM & LMAFIT & ASD \\ \hline
    0.35 & 338s & 824s & 235s & 1012s \\ \hline
    0.45 & 510s & 498s & 342s & 328s\\ \hline
    0.6 & 1674s & 350s & 48s & - \\ \hline
    \end{tabular}
\end{table}
\begin{figure}[!ht]
    \centering
    \includegraphics[width=\linewidth]{Poglavja/Slike/grayscale1000/grafCas.png}
    \caption{Časi izvajanja algoritmov glede na delež znanih vrednosti}
\end{figure}

\section{Vpliv kompleksnosti slik na napolnjevanje}
Eno izmed glavnih vprašanj, ki se nam lahko porodi pri implementaciji algoritmov za napolnitev matrik je, kako sama kompleksnost slik vpliva na točnost rezultatov. Ker slike naključnih vrednosti ni mogoče rekonstruirati, lahko sklepamo, da bodo slike s preprostimi motivi napolnjene bolje. Za namene testiranja je torej smiselno izbrati tako preprosto kot tudi vizualno nasičeno sliko. V naših testiranjih uporabljamo sliki knjige in mesta. Sliki sta velikosti $300 \times 300$ pikslov
\renewcommand{\mapa}{Poglavja/Slike/kompleksnost}

\begin{figure}
    \begin{subfigure}{0.5\linewidth}
        \includegraphics[width=\linewidth]{\mapa/preprosta grayscale 300/knjiga.png}
        \caption{Slika s preprostim motivom}
    \end{subfigure}
    \hfill
    \begin{subfigure}{0.5\linewidth}
        \includegraphics[width=\linewidth]{\mapa/kompleksna grayscale 300/mesto.png}
        \caption{Slika s kompleksnim motivom}
    \end{subfigure}
    \caption{Vir slik: Unsplash}
\end{figure}

\begin{figure}
    \begin{subfigure}{0.325\linewidth}
        \includegraphics[width=\linewidth]{\mapa/preprosta grayscale 300/rez35SVT.png}
    \end{subfigure}
    \hfill
    \begin{subfigure}{0.325\linewidth}
        \includegraphics[width=\linewidth]{\mapa/preprosta grayscale 300/rez45SVT.png}
    \end{subfigure}
    \hfill
    \begin{subfigure}{0.325\linewidth}
        \includegraphics[width=\linewidth]{\mapa/preprosta grayscale 300/rez60SVT.png}
    \end{subfigure}
    \caption{Rekonstrukcija preprostega motiva z algoritmom SVT}
\end{figure}
    
\begin{figure}
    \begin{subfigure}{0.325\linewidth}
        \includegraphics[width=\linewidth]{\mapa/kompleksna grayscale 300/rez35SVT.png}
    \end{subfigure}
    \hfill
    \begin{subfigure}{0.325\linewidth}
        \includegraphics[width=\linewidth]{\mapa/kompleksna grayscale 300/rez45SVT.png}
    \end{subfigure}
    \hfill
    \begin{subfigure}{0.325\linewidth}
        \includegraphics[width=\linewidth]{\mapa/kompleksna grayscale 300/rez60SVT.png}
    \end{subfigure}
    \caption{Rekonstrukcija kompleksnega motiva z algoritmom SVT}
\end{figure}

\begin{figure}
    \begin{subfigure}{0.325\linewidth}
        \includegraphics[width=\linewidth]{\mapa/preprosta grayscale 300/rez35TNNM.png}
    \end{subfigure}
    \hfill
    \begin{subfigure}{0.325\linewidth}
        \includegraphics[width=\linewidth]{\mapa/preprosta grayscale 300/rez45TNNM.png}
    \end{subfigure}
    \hfill
    \begin{subfigure}{0.325\linewidth}
        \includegraphics[width=\linewidth]{\mapa/preprosta grayscale 300/rez60TNNM.png}
    \end{subfigure}
    \caption{Rekonstrukcija preprostega motiva z algoritmom TNNM}
\end{figure}

\begin{figure}
    \begin{subfigure}{0.325\linewidth}
        \includegraphics[width=\linewidth]{\mapa/kompleksna grayscale 300/rez35TNNM.png}
    \end{subfigure}
    \hfill
    \begin{subfigure}{0.325\linewidth}
        \includegraphics[width=\linewidth]{\mapa/kompleksna grayscale 300/rez45TNNM.png}
    \end{subfigure}
    \hfill
    \begin{subfigure}{0.325\linewidth}
        \includegraphics[width=\linewidth]{\mapa/kompleksna grayscale 300/rez60TNNM.png}
    \end{subfigure}
    \caption{Rekonstrukcija kompleksnega motiva z algoritmom TNNM}
\end{figure}

\begin{figure}
    \begin{subfigure}{0.325\linewidth}
        \includegraphics[width=\linewidth]{\mapa/preprosta grayscale 300/rez35LMaFit.png}
    \end{subfigure}
    \hfill
    \begin{subfigure}{0.325\linewidth}
        \includegraphics[width=\linewidth]{\mapa/preprosta grayscale 300/rez45LMaFit.png}
    \end{subfigure}
    \hfill
    \begin{subfigure}{0.325\linewidth}
        \includegraphics[width=\linewidth]{\mapa/preprosta grayscale 300/rez60LMaFit.png}
    \end{subfigure}
    \caption{Rekonstrukcija preprostega motiva z algoritmom LMaFit}
\end{figure}

\begin{figure}
    \begin{subfigure}{0.325\linewidth}
        \includegraphics[width=\linewidth]{\mapa/kompleksna grayscale 300/rez35LMaFit.png}
    \end{subfigure}
    \hfill
    \begin{subfigure}{0.325\linewidth}
        \includegraphics[width=\linewidth]{\mapa/kompleksna grayscale 300/rez45LMaFit.png}
    \end{subfigure}
    \hfill
    \begin{subfigure}{0.325\linewidth}
        \includegraphics[width=\linewidth]{\mapa/kompleksna grayscale 300/rez60LMaFit.png}
    \end{subfigure}
    \caption{Rekonstrukcija preprostega motiva z algoritmom LMaFit}
\end{figure}
\todo{Lmafit vcasih potrebno zagnati veckat}
\begin{figure}[!ht]
    \centering
    \includegraphics[width=\linewidth]{Poglavja/Slike/kompleksnost/kompleksna grayscale 300/kompleksnost.png}
    \caption{Graf napak algoritmov}
\end{figure}

Kot smo pričakovali, so rezultati rekonstrukcije slike s preprostim motivom boljše. Prav tako lahko opazimo, da ima delež znanih vrednosti močnejši vpliv pri sliki s kompleksnim motivom. Napake z dodajanjem informacij torej hitreje padajo pri matrikah večjega ranga. Spomnimo se, da algoritma LMaFit in TNNM za svoje delovanje potrebujeta informacijo o rangu. Pri testiranju je bilo zato potrebno kompleksni sliki podati večjo vrednost ranga, da sta lahko algoritma prišla do dobrih rezultatov.

Sama točnost algoritmov pa ostaja zelo podobna rekonstrukciji velike slike, torej z najboljšimi rezultati pridobljenimi z algoritmom TNNM, nato LMaFit in z najslabšimi rezultati algoritem SVT.

\begin{figure}[!ht]
    \centering
    \includegraphics[width=\linewidth]{Poglavja/Slike/kompleksnost/kompleksna grayscale 300/cas.png}
    \caption{Graf časov izvajanja algoritmov}
\end{figure}
Sami časi izvajanja pa tu niso tako intuitivni. Prva glavna opazka je, da algoritem SVT potrebuje veliko več časa pri preprostem motivu kot pri kompleksem. \todo{razmisli to interpretacijo} To si lahko razlagamo kot posledico praga. Za dobre rezultate smo pri tako majhni matriki prag nastavili visoko. V našem primeru je imel ta vrednost $3600$. V primeru preprostega motiva, lahko pričakujemo, da bomo imeli malo zelo velikih singularnih vrednosti. Zaradi tega se algoritem težko premika in išče rešitev. Iz tega sledi opazka, da je algoritem SVT bolj smiselno uporabljati za kompleksne motive.

Naslednja pomembna opazka pa je, da delež znanih vrednosti različno vpliva na sam čas izvajanja. Tudi ta faktor je torej lahko pomemben pri izbiri algoritma za reševanje problema. Algoritem SVT potrebuje za rekonstrukcijo več časa, kadar ima poznanih več vrednosti, medtem ko se algoritmu TNNM z deležem znanih vrednosti čas izvajanja manjša. Algoritmu LMaFit težko določimo pravilo, saj najpočasnejše izvede rekonstrukcijo pri $0.45$ znanih podatkov. Iz tega sklepamo, da je pri nekem deležu med $0.35$ in $0.6$ rekonstrukcija najpočasnejša. Tu pa je vredno tudi omeniti, da zaradi naključnega generiranja začetne matrike algoritem lahko različno dolgo rekonstruira isti primer. \todo{Preveri znova} Ker pa smo teste pognali večkrat ter v povprečju vedno najdlje čakali pri vrednosti $0.45$ lahko sklepamo, da je v takih primerih rekonstrukcija res bolj zahtevna.

\section{Rekonstrukcija barvnih slik}
Naslednje vprašanje, ki se nam lahko porodi pri testiranju algoritmov je, kako vplivajo barvne slike na samo rekonstrukcijo. Barvne slike so podatkovno podane kot kombinacija barvnih kanalov rdeče, zelene in modre barve. Glavno vprašanje, na katerega bomo poskušali odgovoriti v tem poglavju je, ali je bolje napolnjevati vsak barvni kanal posebej, ali sliko kot celotno. V prvem primeru torej algoritem zaženemo trikrat, medtem ko v drugem sestavimo veliko matriko sestavljeno kot 
\[
    A = \begin{bmatrix}
        R\\G\\B
    \end{bmatrix} 
\] 
kjer $R$ predstavlja matriko z vrednostmi rdečega kanala, G vrednosti zelenega kanala ter B vrednosti modrega kanala.
Vsi testi v tej fazi so bili izvedeni na podatkih, kjer imamo poznanih $0.35$ informacij.
\todo{kako omeniti da je frobenius}
\begin{table}[h]
    \centering
    \begin{tabular}{|c|c|c|c|}
    \hline
    & SVT & TNNM & LMAFIT \\
    \hline
    Enojna rekonstrukcija & $1,50 \times 10^4$ & $9,60\times 10^3$ & $1,15\times 10^4$ \\
    Trojna rekonstrukcija & $1,66\times 10^4$ & $9,79\times 10^3$ & $1,14\times 10^4$ \\
    \hline
    \end{tabular}
\end{table}

\begin{table}[h]
    \centering
    \begin{tabular}{|c|c|c|c|}
    \hline
    & SVT & TNNM & LMAFIT \\
    \hline
    Enojna rekonstrukcija & 352s & 124s & 66s \\
    Trojna rekonstrukcija & 112s & 100s & 42s \\
    \hline
    \end{tabular}
\end{table}
Lahko je videti, da medtem ko obe metodi vrnete približno enako dobre rezultate, je rekonstrukcija pri vseh algoritmih hitrejša, če ločene barvne kanale rekonstruiramo posebej. Iz tu je lahko videti, da je smiselno med sabo neodvisne podatke ločiti, ter jih reševati samostojno. Rezultat je smiselen, saj nam iskanje podobnosti med nepodobnimi podatki poveča količino dela. 
\chapter{Zaključek}\label{1407-1013}

Tekom diplomskega dela smo si ogledali 5 različnih algoritmov, za reševanje matričnih napolnitev. Komentirali smo njihovo delovanje ter se spoznali z definicijami, ki so pomembne in pogoste na tem področju. Nato smo pregledali in primerjali rezultate algoritmov, ter jih interpretirali, glede na delovanje algoritmov. Poskušali smo odgovoriti na vprašanja, ki so pri rekonstrukciji pomembna. Prav tako smo algoritme primerjali z drugim popularnim algoritmom za rekonstrukcijo, predstavili razliko v njegovem delovanju, ter podali primer, kjer algoritmi matričnih napolnitev vrnejo boljši rezultat.

Ker je področje matričnih napolnitev široko, bi lahko delo diplomske naloge še razširili. Algoritmi TNNM, LMaFit in ASD imajo še druge, bolj napredne različice implementacije, ki bi jih lahko preizkusili. Seveda pa obstajajo tudi drugi algoritmi, katerih se v tem delu nismo dotaknili. V razdelku \ref{1307-2251} smo omenili, da bi lahko algoritem LMaFit razširili tako, da bi začeli z več pari matrik $X$ in $Y$. \CG{Prav tako pa bi bilo algoritme smiselno preizkusiti še na drugih podatkih, ki niso slike. Ker imajo slike lokalne podatke podobne, na kar pa se pri problemu matričnih napolnitvah ne moremo zanašati, obstajajo boljši algoritmi za rekonstrukcijo slik. Literatura pogosto testira algoritme na naključno generiranih podatkih.}




%--/Paper--
\printbibliography

\end{document}