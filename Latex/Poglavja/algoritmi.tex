\chapter{Algoritmi}

\section{Pomembe definicije}
Nekatere definicije so uporabljene čez več algoritmov. Z namenom preglednosti, te opisujem v tem poglavju
\begin{enumerate}
  \item $\Omega$ je definirana množica znanih vrednosti
  \item \[ [\proj_\Omega(A)]_{i,j} = \begin{cases}
            a_{ij} & (i, j) \in \Omega \\
            0      & \textrm{drugače}
          \end{cases}
        \]
  \item Operator $\shrink_\tau$ kot \[
            \shrink_\tau(A) := U \shrink_\tau(\Sigma) V^T, \hspace{0.3cm} \shrink_\tau(\Sigma) = diag(max(\sigma_i - \tau, 0))
        \] \cite{CCS}
\end{enumerate}

\section{Minimizacija nuklearne norme}
Minimizacija nuklearne norme (Nuclear Norm Minimization oziroma NNM) se zanaša na dejstvo, da je rang matrike povezan z nuklearno normo matrike. Ta je definirana kot

\[
  ||A||_* = \sum_{i = 0}^{n} \sigma_i
\].

Minimizacijo nuklearne norme je možno pretvoriti v semidefinitni problem, ki ga lahko rešujemo z različnimi pripomočki, na primer SeDuMi \cite{SeDuMi}.

Po \cite{CR08} lahko problem definiramo kot

\begin{align*}
  \textrm{min }     & \hspace{0.5cm} tr(Y)                                    \\
  \textrm{tako da } & \hspace{0.5cm} (Y, A_k) = b_k, k = 1, \cdots , |\Omega| \\
                    & \hspace{0.5cm} Y \succcurlyeq 0
\end{align*}

kjer
\[
  Y = \begin{bmatrix}
    W_1 & X   \\
    X^T & W_2
  \end{bmatrix}
\] tak problem pa lahko že rešujemo z semidefinitnimi programi.


\section{Algoritem praga singularnih vrednosti}
\textbf{Algoritem praga singularnih vrednosti (SVT)} \cite{CCS}, uporabi idejo, da imajo matrike z majhnim rangom nekaj velikih singularnih vrednosti, ostale pa 0 ali pa vsaj blizu 0. Ključna parametra v SVT-ju sta \textit{izbira premika} in \textit{izbira praga},  
%Za svoje delovanje uvede dva nova pomembna koncepta, prvi je premik, drugi pa prag, potreben za uporabo operatorja $\shrink_\tau$ \eqref{1007-1959}. 
algoritem pa temelji na iteraciji
\begin{align}
\label{2407-1910}
        X^{(k)} &= \shrink_\tau(Y^{(k-1)}), \\
        Y^{(k)} &= Y^{(k-1)} + \delta_k \proj(M - X^{(k)}), 
\end{align}
kjer so $\tau > 0$ izbran prag, $\delta_k$ izbran premik, $X^{(0)} = 0 \in \mathbb{R}^{n_1 \times n_2}$ in
$Y^{(0)} = 0 \in \mathbb{R}^{n_1 \times n_2}$. \cite{CCS}

V nadaljevanju bomo opisali glavno idejo zgornje iteracije. V grobem pa temelji na uporabi metode za iskanje vezanih ekstremov, kjer elementi matrik $Y^{(k)}$ predstavljajo Lagrangove množitelje. 

Uvedimo funkcijo 
\begin{align}
    \label{1007-2007}
    f_\tau(X) = \tau\nnorm{X} + \frac{1}{2}\fnorm{X}^2
\end{align}
in optimizacijski problem
\begin{align}
\label{2706-0957}
\begin{split}
    \min_{X\in \mathbb R^{n_1\times n_2}} & \hspace{0.5cm} f_\tau(X), \\
    \textrm{pri pogojih} & \hspace{0.5cm} \proj(X) = \proj(M).
\end{split}
\end{align}
Opazimo lahko, da za velike vrednosti $\tau$ velja $f_\tau(X) \approx \tau\nnorm{X}$, kar pomeni, da bo s primerno izbranim $\tau$, optimizacijski problem minimiziral nuklearno normo.

Denimo, da želimo poiskati minimum funkcije $f(x)$ pri pogojih $g_1(x)=g_2(x)=\ldots=g_{k}(x)=0$.
V teoriji vezanih ekstremov se za tovrstne probleme uvede \textbf{Lagrangeovo funkcijo}
\[\mathcal{L}(x, \lambda_1,\lambda_2,\ldots,\lambda_k) = f(x) + \lambda_1 g_1(x)+\lambda_2g_2(x)+\ldots+\lambda_kg_k(x),\]
nato pa išče ekstreme med njenimi stacionarnimi točkami.
Problemu \eqref{2706-0957} lahko priredimo Lagrangeovo funkcijo
\[
    \mathcal{L}(X, Y) = f_\tau(X) + \left< Y, \proj(M - X) \right>,
\] 
nato pa iščemo njene stacionarne točke.
Zaradi velikega števila parametrov pa ta pristop navadno ni izvedljiv, zato se v SVT-ju uporabi za iskanje ekstremov $\mathcal{L}(X, Y)$ t.i.\ \textit{Uzawa algoritem} \cite{CCS}. Ta ekstreme išče prek iterativnega postopka:
\begin{align}
        X^{(k)} &=  \arg \min_{X} \hspace{0.2cm}\mathcal{L}(X^{(k)}, Y^{(k-1)}) \label{1007-2018},\\
        Y^{(k)} &= Y^{(k-1)} + \delta_k \proj(M - X^{(k)}) \label{1007-2019}
\end{align}

\CR{$\arg\min$ ni tako pogost pojem, tako da dodaj definicijo.}

Izkaže se, da je rešitev \eqref{1007-2018} enaka $\shrink_\tau(Y^{k-1})$. To spodaj dokažemo, še prej pa izpeljimo pomožen rezultat.

\begin{trditev}
Velja:
\begin{align}
    \arg \min_X \hspace{0.2cm} \mathcal{L}(X, Y)
    = \arg \min_X \hspace{0.2cm} \tau\nnorm{X} + \frac{1}{2}\fnorm{X - Y}^2
\end{align}
\end{trditev}

\begin{proof}
Trditev sledi iz krajšega računa:
\begin{align*}
    &\arg \min_X \hspace{0.2cm} \tau\nnorm{X} + \frac{1}{2}\fnorm{X - Y}^2 \\
    &=\arg \min_X \hspace{0.2cm} \tau\nnorm{X} + \frac{1}{2}\trOp{X - Y}{X - Y}\\
    &=\arg \min_X \hspace{0.2cm} \tau\nnorm{X} + \frac{1}{2}(\fnorm{X}^2 - 2\trOp{X}{Y} + \fnorm{Y}^2) \\ 
    &=\arg \min_X \hspace{0.2cm} \tau\nnorm{X} + \frac{1}{2}\fnorm{X}^2 - \trOp{X} {\proj(Y)}\\
    &=\arg \min_X \hspace{0.2cm} \tau\nnorm{X} + \frac{1}{2}\fnorm{X}^2 + \tr(- \proj(X)Y^T) + \tr(\proj(M)Y^T)\\
    &=\arg \min_X \hspace{0.2cm} \tau\nnorm{X} + \frac{1}{2}\fnorm{X}^2 + \tr(\proj(M - X)Y^T)\\
    &=\arg \min_X \hspace{0.2cm} \tau\nnorm{X} + \frac{1}{2}\fnorm{X}^2 + \trOp{Y}{\proj(M - X)}\\ 
    &= \arg \min_X \hspace{0.2cm} \mathcal{L}(X, Y)
\end{align*}
kjer smo v prvi enakosti uporabili definicijo Frobeniusove norme, v drugi bilinearnost skalarnega produkta, 
v tretji smo ignorirali konstanto $\fnorm{Y}^2$, saj ne vpliva na rezultat, 
in upoštevali $\proj(Y^{(k)}) = Y^{(k)}$ za vse $k \in \mathbb{N}$. Zadnje dejstvo sledi iz definicije \eqref{1007-2019} in $Y^{(0)} = 0$. V četrti enakosti smo upoštevali $\trOp{X}{\proj(Y)} = \trOp{\proj(X)}{Y}$, kar je lahko videti, ko se spomnimo, da za skalarni produkt $\trOp{A}{B}$ matrik $A, B \in \mathbb{R}^{n_1 \times n_2}$ velja 
\[
    \trOp{A}{B} = \sum_{i = 1}^{n_1}\sum_{j = 1}^{n_2} a_{ij}b_{ij}.
\]
Prišteli smo tudi konstanto $\tr(\proj(M)Y^T)$, ki ne vpliva na rezultat.
V peti enakosti smo upoštevali linearnost sledi in aditivnost operatorja $P_\Omega$,
v šesti definicijo skalarnega produkta in v zadnji definicijo Lagrangeove funkcije.
\end{proof}
\begin{theorem} \label{1907-2240}
\CG{Za matriki $X \in \mathbb{R}^{n_1 \times n_2}, Y \in \mathbb{R}^{n_1 \times n_2}$} velja:
\begin{align}
    \label{2906-1056}
    \shrink_\tau(Y) = \arg \min_{X} \hspace{0.2cm}\frac{1}{2} \fnorm{X-Y}^2 + \tau\nnorm{X} 
\end{align}
\end{theorem}

\begin{proof} 
\todo{Ali potrebujem strogo konveksnost}
\CR{Po čem odvajaš normo?}
\CG{Najprej se spomnimo definicije konveksne funkcije. Funkcija $f$ je konveksna, če za katerikoli dve točki $x_1, x_2$ v domeni funkcije $f$ velja, da je premica čez ti dve točki na odseku med tema dvema točkama večja ali enaka funkciji $f$.}
\CG{Funkcija $h(X) := \frac{1}{2} \fnorm{X-Y}^2 + \tau\nnorm{X} $ je konveksna funkcija, saj potreben pogoj trikotniške neenakosti matričnih norm zagotavlja, da je matrična norma konveksna funkcija. Vsota konveksnih funkcij pa je prav tako konveksna funkcija. \todo{ali je potrebno citirati} Zaradi konveksnosti lahko tako uporabimo definicijo subgradienta $Z$ v točki $X_0$. Ta je definiran kot: \[\forall X: f(X) \geq  f(X_0) + \trOp{Z}{X - X_0}\] Ali drugače povedano, premica na točko $X_0$ s smernim koeficientom $Z$ je povsod na ali pod funkcijo $h(X)$.}

Pri iskanju minimuma torej iščemo tako točko $X'$, da bo
\CG{eden izmed subgradientov po spremenljivki $X$} v točki $X'$ enak 0.  Izkaže se \cite{CCS}, da je množica subgradientov nuklearne norme definirana kot
\[
    \partial\nnorm{X} = \{UV^* + W: W \in \mathbb{R}^{n_1 \times n_2}, U^*W = 0, WV = 0, \norm{W}_2 \leq 1 \}.
\]
kjer $U \Sigma V^T$ predstavlja SVD razcep matrike $X$. \CG{Prav tako pa vemo, da velja $\frac{\partial}{\partial X}\fnorm{X-Y}^2 = 2(X - Y)$.} \todo{kaj citiram, ali dokazem?} Problem sedaj zapišemo kot $0 \in X' - Y + \tau \partial\nnorm{X'}$. 

Trditev izreka bo sledila, če pokažemo, da velja $X' = \shrink_\tau(Y)$. Najprej razčlenimo SVD razcep matrike $Y$ kot 
\[
    Y = U_0\Sigma_0V_0^T + U_1\Sigma_1V_1^T
\]
kjer $U_0, \Sigma_0$ in $V_0$ predstavljajo singularne vrednosti in pripadajoče singularne vektorje večje od $\tau$, $U_1, \Sigma_1$ in $V_1$ pa tiste manjše od $\tau$. Pokazati želimo, da velja 
\[
    X' = U_0(\Sigma_0 - \tau I)V_0^T.
\] 
\todo{ali moram to pokazati}V nadaljevanju bomo videli, da je to samo alternativen zapis operatorja $\shrink_\tau(Y)$.
\CR{Tega nadaljevanja povsem ne razumem. Kako prideš do preoblikovanja, prek katerega iščeš matriko $W$, mi ni jasno.}
Če zapis vstavimo v prejšnji podan pogoj dobimo
\begin{align*}
    0 = X' - Y + \tau \partial \nnorm{X'}\\
    Y- X' = \tau (U_0 V_0^T + W)
\end{align*}
primerna izbira za $W = \tau^{-1} U_1 \Sigma_1 V_1^T$, saj
\begin{align*}
    Y-X' &= U_0\Sigma_0V_0^T + U_1\Sigma_1V_1^T - U_0(\Sigma_0 - \tau I)V_0^T \\ 
    &= U_0V_0^T(\Sigma_0 - \Sigma_0 + \tau I) + U_1\Sigma_1 V_1^T  \\
    &= \tau U_0 V_0^T + U_1\Sigma_1 V_1^T
\end{align*}
in 
\begin{align*}
    \tau(U_0 V_0^T + W) &= \tau(U_0V_0^T + \tau^{-1} U_1 \Sigma_1 V_1^T)\\ 
    &= \tau U_0 V_0^T + U_1 \Sigma_1 V_1^T 
\end{align*}

Sedaj je zgolj potrebno pokazati, da veljajo potrebne lastnosti matrike $W$.
Po sami definiciji SVD vemo, da so vsi stolpci matrik U in V ortogonalni. Torej velja $U_0^TW = 0$ in $WV_0 = 0$. Ker pa ima matrika $\Sigma_1$ vse elemente manjše od $\tau$ velja tudi $\norm{W}_2 \leq 1$. \CG{$\norm{A}_2$ je namreč definirana kot največja singularna vrednost matrike $A$}. S tem smo pokazali, da $Y - X' \in \tau \partial \nnorm{X'}$.
\end{proof}
Z uporabo \eqref{1007-2018}, \eqref{1007-2019}
in izreka \ref{1907-2240}
res pridemo do iteracije
\eqref{2407-1910}, ki jo uporablja algoritem SVT.
\iffalse
Tako res pridemo 
Po trditvi lahko sedaj zapišemo algoritem \eqref{1007-2018} - \eqref{1007-2019} kot \cite{CCS}
\[
    \begin{cases}
        X^k = \shrink_\tau(Y^{k-1}) \\
        Y^k = Y^{k-1} + \delta_k \proj(M - X^k) 
    \end{cases}
\]
\fi

\subsection{Nastavljanje parametrov $\tau$ in $\delta$} \label{1907-1648}
Opazimo lahko, da algoritem SVT potrebuje dva parametra, $\tau$ in $\delta$, ki ju moramo izbrati 
že pred vstopom v algoritem.

Po priporočilih \cite{CCS}
sta primerni izbiri za $\delta$ in $\tau$
enaki 
\[
    \delta = 1.2\, \dfrac{n_1 n_2}{m}\qquad\text{in}\qquad
    \tau = 5n,
\]
pri čemer je $\tau$ naveden za kvadratne $n\times n$ matrike.
V poglavju z rezultati smo prvotno uporabljali ti dve konstanti, pri 
čemer smo zaradi
pravokotnosti $n_1\times n_2$ matrik uporabljali
$\tau = 5\frac{n_1+n_2}{2}$.
\iffalse
Medtem, ko so koraki v samem algoritmu definirani kot množica korakov, 
smo v okviru rezultatov diplomske naloge, prvotno za premik uporabljali konstanto, ter korak nastavili na 
 po priporočilih \cite{CCS}. 
 \fi
\iffalse
Prav tako članek \cite{CCS} navaja, da je za matrike velikosti $\mathbb{R}^{n \times n}$ smiselno nastaviti $\tau = 5n$, vendar sem v moji implementaciji zaradi posploševanja na nekvadratne matrike, za matrike velikosti $\mathbb{R}^{n_1 \times n_2}$ parameter nastavil na
\[
    \tau = 5\, \frac{n_1+n_2}{2}
\]
\fi
V naših testiranjih se je izkazalo, da sta taka parametra dobra za večje matrike. V razdelku \ref{1307-2251} pa bomo videli, da moramo pri manjših matrikah pogosto zmanjšati premik in povečati prag.