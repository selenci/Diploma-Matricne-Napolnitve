\chapter{Algoritmi}

\section{Pomembe definicije}
Nekatere definicije so uporabljene čez več algoritmov. Z namenom preglednosti, te opisujem v tem poglavju
\begin{enumerate}
  \item $\Omega$ je definirana množica znanih vrednosti
  \item \[ [\proj]_{i,j} = \begin{cases}
            a_{ij} & (i, j) \in \Omega \\
            0      & \textrm{drugače}
          \end{cases}
        \]
  \item Operator $\shrink_\tau$ kot \[
            \shrink_\tau(A) := U \shrink_\tau(\Sigma) V^T, \hspace{0.3cm} \shrink_\tau(\Sigma) = diag(max(\sigma_i - \tau, 0))
        \] \cite{CCS}
  \item Z oznako $M \in \mathbb{R}^{n_1 \times n_2}$ označujemo vhodno matriko, torej tisto, ki ima nekatere podatke neznane. {\vspace{-10cm}\todo{V programu nato uporabljamo oznako M za bitno matriko, vendar te v samih dokazih ne potrebujemo, je potem tukaj oznaka M vredu?}\vspace{10cm}}
  % \item Z oznako $M \in \mathbb{R}^{n_1 \times n_2}$ označujemo bitno matriko M, ki predstavlja masko, s katero označimo katere vrednosti poznamo in katere ne. Vrednost 1 označuje, da je vrednost na tisti poziciji znana, medtem ko 0 označuje, da ni. 
\end{enumerate}

\section{Minimizacija nuklearne norme}
Minimizacija nuklearne norme (Nuclear Norm Minimization oziroma NNM) se zanaša na dejstvo, da je rang matrike povezan z nuklearno normo matrike. Ta je definirana kot

\[
  ||A||_* = \sum_{i = 0}^{n} \sigma_i(A)
\].

Minimizacijo nuklearne norme je možno pretvoriti v semidefinitni problem, ki ga lahko rešujemo z različnimi pripomočki, na primer SeDuMi \cite{SeDuMi}.

Po \cite{CR08} lahko problem definiramo kot

\begin{align*}
  \textrm{min }     & \hspace{0.5cm} tr(Y)                                    \\
  \textrm{tako da } & \hspace{0.5cm} (Y, A_k) = b_k, k = 1, \cdots , |\Omega| \\
                    & \hspace{0.5cm} Y \succcurlyeq 0
\end{align*}

kjer
\[
  Y = \begin{bmatrix}
    W_1 & X   \\
    X^T & W_2
  \end{bmatrix}
\] tak problem pa lahko že rešujemo z semidefinitnimi programi.


\section{Algoritem praga singularnih vrednosti}
\textbf{Algoritem praga singularnih vrednosti (SVT)} \cite{CCS}, uporabi idejo, da imajo matrike z majhnim rangom nekaj velikih singularnih vrednosti, ostale pa 0 ali pa vsaj blizu 0. Ključna parametra v SVT-ju sta \textit{izbira premika} in \textit{izbira praga},  
%Za svoje delovanje uvede dva nova pomembna koncepta, prvi je premik, drugi pa prag, potreben za uporabo operatorja $\shrink_\tau$ \eqref{1007-1959}. 
algoritem pa temelji na iteraciji
\begin{align}
\label{2407-1910}
        X^{(k)} &= \shrink_\tau(Y^{(k-1)}), \\
        Y^{(k)} &= Y^{(k-1)} + \delta_k \proj(M - X^{(k)}), 
\end{align}
kjer so $\tau > 0$ izbran prag, $\delta_k$ izbran premik, $X^{(0)} = 0 \in \mathbb{R}^{n_1 \times n_2}$ in
$Y^{(0)} = 0 \in \mathbb{R}^{n_1 \times n_2}$. \cite{CCS}

V nadaljevanju bomo opisali glavno idejo zgornje iteracije. V grobem pa temelji na uporabi metode za iskanje vezanih ekstremov, kjer elementi matrik $Y^{(k)}$ predstavljajo Lagrangove množitelje. 

Uvedimo funkcijo 
\begin{align}
    \label{1007-2007}
    f_\tau(X) = \tau\nnorm{X} + \frac{1}{2}\fnorm{X}^2
\end{align}
in optimizacijski problem
\begin{align}
\label{2706-0957}
\begin{split}
    \min_{X\in \mathbb R^{n_1\times n_2}} & \hspace{0.5cm} f_\tau(X), \\
    \textrm{pri pogojih} & \hspace{0.5cm} \proj(X) = \proj(M).
\end{split}
\end{align}
Opazimo lahko, da za velike vrednosti $\tau$ velja $f_\tau(X) \approx \tau\nnorm{X}$, kar pomeni, da bo s primerno izbranim $\tau$, optimizacijski problem minimiziral nuklearno normo.

Denimo, da želimo poiskati minimum funkcije $f(x)$ pri pogojih $g_1(x)=g_2(x)=\ldots=g_{k}(x)=0$.
V teoriji vezanih ekstremov se za tovrstne probleme uvede \textbf{Lagrangeovo funkcijo}
\[\mathcal{L}(x, \lambda_1,\lambda_2,\ldots,\lambda_k) = f(x) + \lambda_1 g_1(x)+\lambda_2g_2(x)+\ldots+\lambda_kg_k(x),\]
nato pa išče ekstreme med njenimi stacionarnimi točkami.
Problemu \eqref{2706-0957} lahko priredimo Lagrangeovo funkcijo
\[
    \mathcal{L}(X, Y) = f_\tau(X) + \left< Y, \proj(M - X) \right>,
\] 
nato pa iščemo njene stacionarne točke.
Zaradi velikega števila parametrov pa ta pristop navadno ni izvedljiv, zato se v SVT-ju uporabi za iskanje ekstremov $\mathcal{L}(X, Y)$ t.i.\ \textit{Uzawa algoritem} \cite{CCS}. Ta ekstreme išče prek iterativnega postopka:
\begin{align}
        X^{(k)} &=  \arg \min_{X} \hspace{0.2cm}\mathcal{L}(X^{(k)}, Y^{(k-1)}) \label{1007-2018},\\
        Y^{(k)} &= Y^{(k-1)} + \delta_k \proj(M - X^{(k)}) \label{1007-2019}
\end{align}

\CR{$\arg\min$ ni tako pogost pojem, tako da dodaj definicijo.}

Izkaže se, da je rešitev \eqref{1007-2018} enaka $\shrink_\tau(Y^{k-1})$. To spodaj dokažemo, še prej pa izpeljimo pomožen rezultat.

\begin{trditev}
Velja:
\begin{align}
    \arg \min_X \hspace{0.2cm} \mathcal{L}(X, Y)
    = \arg \min_X \hspace{0.2cm} \tau\nnorm{X} + \frac{1}{2}\fnorm{X - Y}^2
\end{align}
\end{trditev}

\begin{proof}
Trditev sledi iz krajšega računa:
\begin{align*}
    &\arg \min_X \hspace{0.2cm} \tau\nnorm{X} + \frac{1}{2}\fnorm{X - Y}^2 \\
    &=\arg \min_X \hspace{0.2cm} \tau\nnorm{X} + \frac{1}{2}\trOp{X - Y}{X - Y}\\
    &=\arg \min_X \hspace{0.2cm} \tau\nnorm{X} + \frac{1}{2}(\fnorm{X}^2 - 2\trOp{X}{Y} + \fnorm{Y}^2) \\ 
    &=\arg \min_X \hspace{0.2cm} \tau\nnorm{X} + \frac{1}{2}\fnorm{X}^2 - \trOp{X} {\proj(Y)}\\
    &=\arg \min_X \hspace{0.2cm} \tau\nnorm{X} + \frac{1}{2}\fnorm{X}^2 + \tr(- \proj(X)Y^T) + \tr(\proj(M)Y^T)\\
    &=\arg \min_X \hspace{0.2cm} \tau\nnorm{X} + \frac{1}{2}\fnorm{X}^2 + \tr(\proj(M - X)Y^T)\\
    &=\arg \min_X \hspace{0.2cm} \tau\nnorm{X} + \frac{1}{2}\fnorm{X}^2 + \trOp{Y}{\proj(M - X)}\\ 
    &= \arg \min_X \hspace{0.2cm} \mathcal{L}(X, Y)
\end{align*}
kjer smo v prvi enakosti uporabili definicijo Frobeniusove norme, v drugi bilinearnost skalarnega produkta, 
v tretji smo ignorirali konstanto $\fnorm{Y}^2$, saj ne vpliva na rezultat, 
in upoštevali $\proj(Y^{(k)}) = Y^{(k)}$ za vse $k \in \mathbb{N}$. Zadnje dejstvo sledi iz definicije \eqref{1007-2019} in $Y^{(0)} = 0$. V četrti enakosti smo upoštevali $\trOp{X}{\proj(Y)} = \trOp{\proj(X)}{Y}$, kar je lahko videti, ko se spomnimo, da za skalarni produkt $\trOp{A}{B}$ matrik $A, B \in \mathbb{R}^{n_1 \times n_2}$ velja 
\[
    \trOp{A}{B} = \sum_{i = 1}^{n_1}\sum_{j = 1}^{n_2} a_{ij}b_{ij}.
\]
Prišteli smo tudi konstanto $\tr(\proj(M)Y^T)$, ki ne vpliva na rezultat.
V peti enakosti smo upoštevali linearnost sledi in aditivnost operatorja $P_\Omega$,
v šesti definicijo skalarnega produkta in v zadnji definicijo Lagrangeove funkcije.
\end{proof}
\begin{theorem} \label{1907-2240}
\CG{Za matriki $X \in \mathbb{R}^{n_1 \times n_2}, Y \in \mathbb{R}^{n_1 \times n_2}$} velja:
\begin{align}
    \label{2906-1056}
    \shrink_\tau(Y) = \arg \min_{X} \hspace{0.2cm}\frac{1}{2} \fnorm{X-Y}^2 + \tau\nnorm{X} 
\end{align}
\end{theorem}

\begin{proof} 
\todo{Ali potrebujem strogo konveksnost}
\CR{Po čem odvajaš normo?}
\CG{Najprej se spomnimo definicije konveksne funkcije. Funkcija $f$ je konveksna, če za katerikoli dve točki $x_1, x_2$ v domeni funkcije $f$ velja, da je premica čez ti dve točki na odseku med tema dvema točkama večja ali enaka funkciji $f$.}
\CG{Funkcija $h(X) := \frac{1}{2} \fnorm{X-Y}^2 + \tau\nnorm{X} $ je konveksna funkcija, saj potreben pogoj trikotniške neenakosti matričnih norm zagotavlja, da je matrična norma konveksna funkcija. Vsota konveksnih funkcij pa je prav tako konveksna funkcija. \todo{ali je potrebno citirati} Zaradi konveksnosti lahko tako uporabimo definicijo subgradienta $Z$ v točki $X_0$. Ta je definiran kot: \[\forall X: f(X) \geq  f(X_0) + \trOp{Z}{X - X_0}\] Ali drugače povedano, premica na točko $X_0$ s smernim koeficientom $Z$ je povsod na ali pod funkcijo $h(X)$.}

Pri iskanju minimuma torej iščemo tako točko $X'$, da bo
\CG{eden izmed subgradientov po spremenljivki $X$} v točki $X'$ enak 0.  Izkaže se \cite{CCS}, da je množica subgradientov nuklearne norme definirana kot
\[
    \partial\nnorm{X} = \{UV^* + W: W \in \mathbb{R}^{n_1 \times n_2}, U^*W = 0, WV = 0, \norm{W}_2 \leq 1 \}.
\]
kjer $U \Sigma V^T$ predstavlja SVD razcep matrike $X$. \CG{Prav tako pa vemo, da velja $\frac{\partial}{\partial X}\fnorm{X-Y}^2 = 2(X - Y)$.} \todo{kaj citiram, ali dokazem?} Problem sedaj zapišemo kot $0 \in X' - Y + \tau \partial\nnorm{X'}$. 

Trditev izreka bo sledila, če pokažemo, da velja $X' = \shrink_\tau(Y)$. Najprej razčlenimo SVD razcep matrike $Y$ kot 
\[
    Y = U_0\Sigma_0V_0^T + U_1\Sigma_1V_1^T
\]
kjer $U_0, \Sigma_0$ in $V_0$ predstavljajo singularne vrednosti in pripadajoče singularne vektorje večje od $\tau$, $U_1, \Sigma_1$ in $V_1$ pa tiste manjše od $\tau$. Pokazati želimo, da velja 
\[
    X' = U_0(\Sigma_0 - \tau I)V_0^T.
\] 
\todo{ali moram to pokazati}V nadaljevanju bomo videli, da je to samo alternativen zapis operatorja $\shrink_\tau(Y)$.
\CR{Tega nadaljevanja povsem ne razumem. Kako prideš do preoblikovanja, prek katerega iščeš matriko $W$, mi ni jasno.}
Če zapis vstavimo v prejšnji podan pogoj dobimo
\begin{align*}
    0 = X' - Y + \tau \partial \nnorm{X'}\\
    Y- X' = \tau (U_0 V_0^T + W)
\end{align*}
primerna izbira za $W = \tau^{-1} U_1 \Sigma_1 V_1^T$, saj
\begin{align*}
    Y-X' &= U_0\Sigma_0V_0^T + U_1\Sigma_1V_1^T - U_0(\Sigma_0 - \tau I)V_0^T \\ 
    &= U_0V_0^T(\Sigma_0 - \Sigma_0 + \tau I) + U_1\Sigma_1 V_1^T  \\
    &= \tau U_0 V_0^T + U_1\Sigma_1 V_1^T
\end{align*}
in 
\begin{align*}
    \tau(U_0 V_0^T + W) &= \tau(U_0V_0^T + \tau^{-1} U_1 \Sigma_1 V_1^T)\\ 
    &= \tau U_0 V_0^T + U_1 \Sigma_1 V_1^T 
\end{align*}

Sedaj je zgolj potrebno pokazati, da veljajo potrebne lastnosti matrike $W$.
Po sami definiciji SVD vemo, da so vsi stolpci matrik U in V ortogonalni. Torej velja $U_0^TW = 0$ in $WV_0 = 0$. Ker pa ima matrika $\Sigma_1$ vse elemente manjše od $\tau$ velja tudi $\norm{W}_2 \leq 1$. \CG{$\norm{A}_2$ je namreč definirana kot največja singularna vrednost matrike $A$}. S tem smo pokazali, da $Y - X' \in \tau \partial \nnorm{X'}$.
\end{proof}
Z uporabo \eqref{1007-2018}, \eqref{1007-2019}
in izreka \ref{1907-2240}
res pridemo do iteracije
\eqref{2407-1910}, ki jo uporablja algoritem SVT.
\iffalse
Tako res pridemo 
Po trditvi lahko sedaj zapišemo algoritem \eqref{1007-2018} - \eqref{1007-2019} kot \cite{CCS}
\[
    \begin{cases}
        X^k = \shrink_\tau(Y^{k-1}) \\
        Y^k = Y^{k-1} + \delta_k \proj(M - X^k) 
    \end{cases}
\]
\fi

\subsection{Nastavljanje parametrov $\tau$ in $\delta$} \label{1907-1648}
Opazimo lahko, da algoritem SVT potrebuje dva parametra, $\tau$ in $\delta$, ki ju moramo izbrati 
že pred vstopom v algoritem.

Po priporočilih \cite{CCS}
sta primerni izbiri za $\delta$ in $\tau$
enaki 
\[
    \delta = 1.2\, \dfrac{n_1 n_2}{m}\qquad\text{in}\qquad
    \tau = 5n,
\]
pri čemer je $\tau$ naveden za kvadratne $n\times n$ matrike.
V poglavju z rezultati smo prvotno uporabljali ti dve konstanti, pri 
čemer smo zaradi
pravokotnosti $n_1\times n_2$ matrik uporabljali
$\tau = 5\frac{n_1+n_2}{2}$.
\iffalse
Medtem, ko so koraki v samem algoritmu definirani kot množica korakov, 
smo v okviru rezultatov diplomske naloge, prvotno za premik uporabljali konstanto, ter korak nastavili na 
 po priporočilih \cite{CCS}. 
 \fi
\iffalse
Prav tako članek \cite{CCS} navaja, da je za matrike velikosti $\mathbb{R}^{n \times n}$ smiselno nastaviti $\tau = 5n$, vendar sem v moji implementaciji zaradi posploševanja na nekvadratne matrike, za matrike velikosti $\mathbb{R}^{n_1 \times n_2}$ parameter nastavil na
\[
    \tau = 5\, \frac{n_1+n_2}{2}
\]
\fi
V naših testiranjih se je izkazalo, da sta taka parametra dobra za večje matrike. V razdelku \ref{1307-2251} pa bomo videli, da moramo pri manjših matrikah pogosto zmanjšati premik in povečati prag.
\section{Algoritem minimizacije prirezane nuklearne norme} \label{2807-1442}
Kot nam že samo ime pove, je \textbf{algoritem minimizacije prirezane nuklearne norme (TNNM)} \cite{TNNM-HZYLH12} soroden algoritmu NNM. Dodatna informacija, ki pa jo uporabi TNNM, je rang $r$ originalne, nezašumljene matrike.

Osrednjo vlogo v algoritmu ima \textbf{$r$-prirezana nuklearna norma}, ki za dano matriko $X \in \mathbb{R}^{n_1 \times n_2}$ vrne vsoto njenih $\min(n_1,n_2) - r$ najmanjših singularnih vrednosti:
\[
    \norm{X}_r = \sum^{\min(n_1, n_2)}_{i = r + 1} \sigma_i(X).
\]
TNNM rešuje optimizacijski problem
\begin{align}
    \label{1107-1305}
    \begin{split}
        \min_{X\in  \mathbb{R}^{n_1 \times n_2}}              & \hspace{0.5cm} \norm{X}_r, \\
        \textrm{pri pogoju} & \hspace{0.5cm} \proj(X) = \proj(M).
    \end{split}
\end{align}
Cilj algoritma je torej čim bolj zmanjšati najmanjše singularne vrednosti, medtem ko velikih ne omejujemo. S tem problem minimizacije omilimo.

Problem \eqref{1107-1305} lahko zapišemo v ekvivalentni obliki
\begin{align}
    \label{2507-0835}
    \begin{split}
        \min_{X\in  \mathbb{R}^{n_1 \times n_2}}              & \hspace{0.5cm} \nnorm{X} - \sum_{i=1}^{r} \sigma_i(X), \\
        \textrm{pri pogojih} & \hspace{0.5cm} \proj(X) = \proj(M).
    \end{split}
\end{align}
V izpeljavah, ki sledijo, bomo potrebovali naslednji izrek.

\begin{theorem}
    \label{2507-0850}
    Za matrike $X \in \mathbb{R}^{n_1 \times n_2}$, $A \in \mathbb{R}^{r \times n_1}$ in $B \in \mathbb{R}^{r \times n_2}$,
    ter naravno število $r \in \mathbb{N}$, ki zadoščajo pogojem $r \leq \min(n_1, n_2)$, $AA^T = I_{r}$  in $BB^T = I_{r}$, velja neenakost
    \begin{align}
        \label{2507-0836}
        \tr(AXB^T) \leq \sum_{i=1}^{r} \sigma_i(X)
    \end{align}
\end{theorem}

\begin{proof}
    Velja
    \begin{align}
        \label{2507-0839}
        \tr(AXB^T) = \tr(XB^TA) \leq \sum^{\min(n_1, n_2)}_{i=1} \sigma_i(X) \sigma_i(B^TA),
    \end{align}
    kjer smo v enakosti uporabili komutativnost $\tr(ZW)=\tr(WZ)$ sledi,
    v neenakosti pa von Neumannovo neenakost za sled \cite{TNNM-HZYLH12}.

    Po definiciji so singularne vrednosti matrike $Y$ enake korenom lastnih vrednosti matrike $Y^TY$. Torej so singularne vrednosti matrike $B^TA$ enake korenom lastnih vrednosti matrike $A^TBB^TA=A^TI_rA = A^TA$.
    %https://cutt.ly/UwaPOfaK
    Ker imata matriki $XY$ in $YX$ enake neničelne lastne vrednosti in po predpostavki velja $AA^T = I_r$, lahko
    od tod sklepamo, da ima produkt $B^TA$ $r$ singularnih vrednosti enakih 1.
    Zato velja
    \[
        \sum^{\min(n_1, n_2)}_{i=1} \sigma_i(X) \sigma_i(B^TA) = \sum^{r}_{i=1} \sigma_i(X),
    \]
    kar skupaj z \eqref{2507-0839}
    dokaže neenakost \eqref{2507-0836}
    v izreku.
\end{proof}

\begin{theorem}
    \label{2507-0851}
    Naj bo $X = U \Sigma V^T$
    SVD razcep matrike $X$.
    Naj bosta $A$ in $B$ matriki, sestavljeni
    iz prvih $r$ stolpcev matrik $U$ in $V$.
    Velja
    \[\tr(AXB^T) = \sum^{r}_{i=1} \sigma_i(X).\]
\end{theorem}

\begin{proof}
    Označimo matriki $A$ in $B$ z
    $A = (u_1, \hdots , u_r)^T$ in $B = (v_1, \hdots , v_r)^T$,
    kjer je $u_i$ $i$-ti stolpec matrike $U$ ter $v_i$ $i$-ti stolpec
    matrike $V$.
    \begin{align*}
        \tr(AXB^T) & = \tr\big((u_1, \hdots , u_r)^T X (u_1, \hdots , u_r)^T\big)                                                                                                      \\
                   & = \tr\big((u_1, \hdots , u_r)^T U \Sigma V^T (u_1, \hdots , u_r)^T\big)                                                                                           \\
                   & =\tr\Big( \begin{bmatrix} I_r & 0 \\ 0 & 0 \end{bmatrix} \Sigma \begin{bmatrix} I_r & 0 \\ 0 & 0 \end{bmatrix}\Big)                                               \\
                   & = \tr\Big(\begin{bmatrix} \sigma_1 \\[-8pt] & \ddots & & \\[-8pt] & & \sigma_r \\[-8pt] & &  & 0 \\[-8pt]  & & & & \ddots \\[-8pt] & & & & & 0 \end{bmatrix}\Big) \\
                   & = \sum_{i = 1}^{r} \sigma_i(X),
    \end{align*}
    kar dokaže trditev izreka.
\end{proof}
Po izrekih \ref{2507-0850} in \ref{2507-0851}
velja
\[
    \max_{
        \substack{AA^T = I,\\ BB^T = I}} \tr(AXB^T) = \sum^{r}_{i = 1} \sigma_i(X)
\]
Torej je optimizacijski problem
\eqref{2507-0835} ekvivalenten
problemu
\begin{align*}
    \min_{X\in  \mathbb{R}^{n_1 \times n_2}}\hspace{0.5cm} & \nnorm{X} - \max_{\substack{AA^T = I, \\ BB^T = I}} \tr(AXB^T), \\
    \text{pri pogoju} \hspace{0.5cm}                       & \proj(X) = \proj(M).
\end{align*}

Sedaj lahko opišemo idejo algoritma TNNM:
\begin{enumerate}
    \item Izračunamo $X^{(0)} = \proj(M)$.
    \item Za $k=0,1,2,\ldots$ ponavljamo iteracijo:
          \begin{enumerate}
              \item Izračunamo SVD razcep
                    $X^{(k)} = U^{(k)} \Sigma^{(k)} (V^{(k)})^T$.
              \item $A^{(k)}$ definiramo kot prvih    $r$ stolpcev matrike $U^{(k)}$,
                    $B^{(k)}$ pa kot prvih
                    $r$ stolpcev matrike $V^{(k)}$.
              \item $X^{(k+1)}$ je enak rešitvi
                    optimizacijskega problema
                    \begin{align}
                        \label{2507-0900}
                        \begin{split}
                            \min_{X\in  \mathbb{R}^{n_1 \times n_2}} \hspace{0.5cm}         & \nnorm{X} - \tr(A^{(k)}X(B^{(k)})^T), \\
                            \text{pri pogoju} \hspace{0.5cm} & \proj(X) = \proj(M).
                        \end{split}
                    \end{align}
          \end{enumerate}
\end{enumerate}
Z nadaljnjim preoblikovanjem problema
\eqref{2507-0900} v ekvivalentnega
\begin{align}
    \label{1307-1527}
    \begin{split}
        \min_{X\in  \mathbb{R}^{n_1 \times n_2}}  \hspace{0.5cm}         & \nnorm{X} - \tr(A^{(k)}W(B^{(k)})^T),  \\
        \text{pri pogojih} \hspace{0.5cm} & W=X, \hspace{0.2cm} \proj(W) = \proj(M),
    \end{split}
\end{align}
lahko za reševanje uporabimo \textbf{algoritem ADMM} \cite{TNNM-HZYLH12}.

Gre za reševanje problema vezanih ekstremov, ki ga lahko zapišemo s pomočjo Lagrangeove funkcije, pri čemer algoritem ADMM doda še
člen, pomnožen z \textit{regularizacijskim parametrom} $\beta$.
\[
    \mathcal{L}(X, Y, W, \beta) = \nnorm{X} - \tr(A W B^T) + \frac{\beta}{2} \fnorm{X - W}^2 + \tr(Y^T(X-W)).
\]
Matriki $A$ in $B$ sta v okviru algoritma ADMM konstantni. Ti na začetku nastavimo na vrednost $A^{(k)}$ in $B^{(k)}$ iz prejšnje iteracije. Opazimo lahko, da funkcija upošteva zgolj pogoj $X = W$. Videli bomo, da algoritem pogoj $\proj(X) = \proj(M)$ definira posredno, s popravljanjem znanih vrednosti (korak \eqref{2906-1248} spodaj).%\todo{ali lahko citiram naprej}

Matriko $X^{(k+1)}$ definiramo kot
\[
    X^{(k+1)} = \argmin_X \mathcal{L}(X, Y^{(k)}, W^{(k)}, \beta)
\]
Z ignoriranjem konstantnih členov, pa lahko zapišemo
\begin{align*}
    X^{(k+1)} & = \argmin_X \Big( \nnorm{X} + \frac{\beta}{2}\trOp{X-W^{(k)}}{X-W^{(k)}} + \frac{\beta}{2}\trOp{\frac{2}{\beta}Y}{X} \Big) \\
              & = \argmin_X \Big( \nnorm{X} + \frac{\beta}{2}\tr\Big(X^TX - X^TW^{(k)} -                                                   \\
              & \hspace{2cm} (W^{(k)})^TX + (W^{(k)})^TW^{(k)} + \frac{2}{\beta}(Y^{(k)})^TX \Big)\Big).
\end{align*}
Z namenom faktorizacije dodamo konstantne člene
\[-(W^{(k)})^T\frac{1}{\beta}(Y^{(k)}) + (\frac{1}{\beta}(Y^{(k)}))^T(-W^{(k)} + \frac{1}{\beta}(Y^{(k)}))\]
in dobimo
\begin{align*}
     & \argmin_X \Big( \nnorm{X} + \frac{\beta}{2}\tr\Big(X^T\big(X - W^{(k)} + \frac{1}{\beta}Y^{(k)}\big) - (W^{(k)})^T \big(X - W^{(k)} + \frac{1}{\beta}Y^{(k)}\big) \\
     & \hspace{2cm} +\frac{1}{\beta}\big(Y^{(k)})^T(X - W^{(k)} + \frac{1}{\beta}Y^{(k)}\big)\Big) \Big)                                                                 \\
     & = \argmin_X \Big( \nnorm{X} + \frac{\beta}{2} \fnorm{X-W^{(k)} + \frac{1}{\beta}Y^{(k)}}^2 \Big)                                                                        \\
     & = \argmin_X \Big( \frac{1}{\beta}\nnorm{X} +  \frac{1}{2}\fnorm{X-(W^{(k)} - \frac{1}{\beta}Y^{(k)})}^2\Big)  
\end{align*}
Po izreku \ref{1907-2240} \CG{uporabljenem za $X=X^{(k+1)}$, $Y=W^{(k)} - \frac{1}{\beta}Y^{(k)}$ in $\tau=\frac{1}{\beta}$} pa sledi
\[
    X^{(k+1)} = \shrink_\frac{1}{\beta}(W^{(k)} - \frac{1}{\beta} Y^{(k)}).
\]
Matriko $W^{(k+1)}$ podobno izračunamo kot
\begin{align*}
    W^{(k+1)} & = \argmin_{W} \mathcal{L}(X^{(k+1)}, Y^{(k)}, W, \beta)                                   \\
              & = \argmin_W \frac{\beta}{2} \fnorm{W - (X^{(k+1)} + \frac{1}{\beta}(A^T B + Y^{(k)})) }^2
\end{align*}
Očitno je
\[
    W^{(k+1)} = X^{(k+1)} + \frac{1}{\beta}(A^T B + Y^{(k)})
\]
Sedaj uporabimo še pogoj $\proj(W^{(k+1)}) = \proj(M)$ in tiste elemente, ki jih poznamo, popravimo.
\begin{equation}
    \label{2906-1248}
    W^{(k+1)} = W^{(k+1)} + \proj(M - W^{(k+1)}).
\end{equation}
Z besedami, predpis \eqref{2906-1248}
preprosto spremeni mesta $W^{(k+1)}$, kjer vrednosti poznamo, na znane vrednosti, ostalih pa ne spremeni.

\CG{
Po algoritmu ADMM \cite{admmForNNM} na koncu iteracije preprosto posodobimo matriko $Y$, kot
\[
    Y^{(k+1)} = Y^{(k)} + \beta(X^{(k+1) - W^{k+1}}).
\]
}

\section{Izmenjajoč gradientni spust}
Ker je računanje SVD razcepa zahtevna operacija, saj ima časovno zahtevnost
$O(n^3)$, je bilo predlaganih nekaj algoritmov, ki za svoje delovanje ne potrebujejo SVD-ja. Algoritem Izmenjajočega gradientnega spusta, oziroma v nadaljevnu ASD (Alternating Steepest Descent) sloni na računanju gradienta in premikanju po njem. Glavni cilj algoritma je, najti dve matriki $X \in \mathbb{R}^{n_1 \times r}$ ter $Y \in \mathbb{R}^{r \times n_2}$, tako da bo veljalo $M = XY$. Vidimo lahko, da ponovno potrebujemo informacijo o rangu matrike, ki jo rekonstruiramo. Ker imata tako $X$ in $Y$ kvečjemu rang $r$, vemo, da tudi njun produkt $XY$ ne bo imel ranga večjega od $r$. 

Cilj algoritma je minimizirati \todo{zakaj 1/2}
\[
    \min_{X, Y} \hspace{0.5cm} \frac{1}{2}\, ||\proj(M) - \proj(XY)||^2_F
\] 
Algoritem minimizacijo razdeli na dve, nato pa izmenično rešuje eno in nato drugo kot 
\begin{align*}
    X_{i+1} &= \arg \min_{X} \hspace{0.3cm} ||\proj(M) - \proj(XY_i)||^2_F \\
    Y_{i+1} &= \arg \min_{Y} \hspace{0.3cm} ||\proj(M) - \proj(X_{i+1}Y)||^2_F
\end{align*}
\cite{AST-TK15}

Za uporabo algoritma ASD potrebujemo odvoda funkcije $f(X,Y) = \frac{1}{2}\, ||\proj(M) - \proj(XY)||^2_F$, ki ga izračunamo za vsak element posebej.
\todo{Zakaj pride $Y^T$ \href{https://math.stackexchange.com/questions/2128462/gradient-of-squared-frobenius-norm-of-a-matrix}{math stackexchange}} 
\begin{align*}
    \frac{\partial}{\partial x_{a,b}} f &= \frac{\partial}{\partial x_{a,b}} \frac{1}{2}\, ||\proj(M) - \proj(XY)||^2_F = \\
    &= \frac{\partial}{\partial x_{a,b}} \frac{1}{2}\, \sum_{i}^{n_1}\sum_{j}^{n_2}(\delta_{i,j}m_{i,j} - \delta_{i,j}\sum_{k}^{r}(x_{i,k}y_{k,j}))^2 = \\
    &= \, \sum_{j}^{n_2}(\delta_{a,j}m_{a,j} - \delta_{i,j}\sum_{k}^{r}(x_{a,k}y_{k,j}))(-y_{b,j}) \implies \\
    &\implies \frac{\partial}{\partial X}f = -(\proj(M) - \proj(XY))Y^T
    % &= \sum_{i,j}(\delta_{i,j}m_{i,j} - \delta_{i,j}\sum_{k}(x_{i,k}y_{k,j}))
    % (\sum_{k}y_{k,j}) = \\
    % &= (P(M) - P(XY))Y^T
\end{align*}
kjer 
\[
    \delta_{i,j} = \begin{cases}
        1, (i, j) \in \Omega \\
        0, (i, j) \notin \Omega
    \end{cases}
\]
Na podoben način bi lahko pokazali tudi
\[
    \frac{\partial}{\partial Y} = -X^T (\proj(M) - \proj(XY))
\]
Poiščimo še najboljši korak gradientnega spusta.
Cilj je pokazati, da je premik po gradientu s korakom $t_x$ najboljši.
Najprej definirajmo sam premik, kot
\[
    X^{k+1} = X^k - t_X \nabla f_Y(X)
\]
Ker želimo, da bodo znane vrednosti produkta $X^{k+1}Y^{k}$ kar se da podobne znanim vrednostim matrike $M$, nastavimo $t_x$ kot 
\begin{align*}
    t_x &= \arg \min_t \hspace{0.3cm} g(t)
\end{align*} kjer
\begin{align*}
    g(t) &= \frac{1}{2} ||\proj(M) - \proj((X - t\nabla f_Y(X))Y)||_F^2
\end{align*}
Ker je \cite{AST-TK15} \todo{ali lahko tako?}
\[
  g'(t) = -||\nabla f_Y(X)||_F^2 + t||\proj(f_Y(X)Y)||_F^2 
\]
vidimo, da funkcija $g(t)$ doseže minimum pri 
\[
  t_x = \frac{||\nabla f_Y(X)||_F^2}{||\proj(\nabla f_Y(X)Y)||_F^2}  
\]
Podobno velja za korak v smeri gradientnega spusta matrike $Y$, kjer
\[
  t_y = \frac{||\nabla f_X(Y)||_F^2}{||\proj(X \nabla f_X(Y))||_F^2}  
\]

% \begin{align*}
%     g'(t) &=  \proj(M - XY) \proj(X'Y)^T = \\
%     &= \proj(M - XY) \proj((\proj(XY) - \proj(M))Y^T Y)^T = \\
%     &= \proj(M - XY) (\proj(\proj(XY)Y^TY)^T - \proj(\proj(M)Y^TY)^T) = \\
%     &= (\proj(M) - \proj(XY)) (\proj(\proj(XY)Y^TY)^T - \proj(\proj(M)Y^TY)^T) = \\
%     &= \proj(M)\proj(\proj(XY)Y^TY)^T - \proj(M)\proj(\proj(M)Y^TY)^T \\ &- \proj(XY)\proj(\proj(XY)Y^TY)^T + \proj(XY)\proj(\proj(M)Y^TY)^T = \\
%     &= (\proj(M) - \proj(XY))\proj(\proj(XY)Y^TY)^T + (\proj(XY) - \proj(M))(\proj(\proj(M)Y^TY)^T)
% \end{align*}

% % \begin{align*}
% %     g'(t) &= \sum_{i}^{n_1}\sum_{j}^{n_2}(m_{i,j} - \sum_{k}^{r} x_{i,r}y_{r,j} + t \sum_{k}^{r}f_{i,r}y_{r,j})(\sum_{k}^{r}f_{i,r}y_{r,j}) = \\
% % \end{align*}

% % &= \tr(\proj(M - XY) \proj(-(\proj(M) - \proj(XY))Y^T Y)^T) + t||X'Y||^2_F = \\
% % &= -\tr(\proj(M - XY) \proj((\proj(M)Y^T - \proj(XY)Y^T) Y)^T) + t||X'Y||^2_F = \\
% % &= -\tr(\proj(M - XY) \proj(Y^T(\proj(M)Y^T - \proj(XY)Y^T)^T)) + t||X'Y||^2_F = \\
\section{LMaFit}
LMaFit je algoritem, ki podobno kot ASD, rešuje problem matričnih napolnitev z uporabo dveh manjših matrik $X^{n_1 \times r}$ in $Y^{n_1 \times r}$. Podobno kot v prejšnjih algoritmih rešujemo problem 
\begin{align*}
    \min_{X, Y, Z}& \hspace{0.5cm} \frac{1}{2}\fnorm{XY - Z}^2\\
    \text{pri pogojih}& \hspace{0.5cm} \proj(M) = \proj(Z)
\end{align*}
le da sam problem rešujemo s pomočjo Moore-Penrose inverza (Označen z $\dagger$).

Lahko je videti, da bo funkcija $f(X,Y,Z) = \frac{1}{2}\fnorm{XY - Z}^2$ imela najmanjšo vrednost, ko bo $XY = Z$. Zato uvedemo iterativni algoritem, ki izmenično posodablja $X, Y$ in $Z$ tako, da fiksira dve izmed matrik. Minimizacijo matrik $X$ in $Y$ torej dosežemo na naslednji način.
\begin{align*}
    X^{i+1}Y^{i} &= Z^{i} \hspace{0.5cm} \\
    X^{i+1} &= Z^{i}(Y^i)^\dagger
\end{align*}
Kjer se zanašamo na dejstvo, da nam množenje z Moore-Penrose inverzom z desne 
da najboljši možen rezultat.
Podobno posodobimo matriko $Y$.
\begin{align*}
    \hspace{0.5cm} X^{i+1}Y^{i+1} &= Z^{i}   \\
    Y^{i+1} &= (X^{i+1})^\dagger Z^{i}
\end{align*}

Sedaj le še posodobimo matriko Z, kot
\[
    Z^{i+1} = X^{i+1}Y^{i+1} + \proj(M - X^{i+1}Y^{i+1})
\]
Torej podobno kot prej poskušamo doseči $XY = Z$, vendar zaradi omejitve znanih vrednosti matrike M, na tistem mestu vrednosti popravimo. \cite{LMaFit-WY12}

