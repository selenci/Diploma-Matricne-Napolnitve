\chapter{Algoritmi} \label{1407-1011}

\CB{Na koncu napisati kratek odstavek o tem, kaj je vsebina poglavja.}

\section{Definicije in oznake}
V tem razdelku uvedemo definicije in oznake, ki se bodo pojavljale v preostanku dela.
\begin{enumerate}
  \item $\Omega$ je podmnožica urejenih parov $\{ (i, j) : i = 1, \hdots , n_1, \hspace{0.3cm} j = 1, \hdots, n_2 \}$ V delu je poimenujemo kot \textbf{množica znanih vrednosti}.
  \item Naj bo matrika $A$ definirana kot $A = [a_{ij}]_{i,j}$. Z $\proj(A)$ označimo preslikavo \todo{preveri oznako}
        \[ [\proj(A)_{ij}]_{i,j} = \begin{cases}
            a_{ij}, & (i, j) \in \Omega \\
            0,      & \text{sicer}
          \end{cases}
        \]
  \item Naj bo $\tau > 0$ pozitivno realno število. Operator $\shrink_\tau: \mathbb{R}^{n_1 \times n_2} \rightarrow \mathbb{R}^{n_1 \times n_2}$ definiran kot
        \begin{align}
          \label{1007-1959}
          \shrink_\tau(A) := U \shrink_\tau(\Sigma) V^T, \hspace{0.3cm} \shrink_\tau(\Sigma) = \diag(\max(\sigma_i - \tau, 0)),
        \end{align}
        imenujemo \textbf{pragovni operator}. \todo{pražni operator?}\cite{CCS}
  \item Matriko $M \in \mathbb{R}^{n_1 \times n_2}$ imenujemo \textbf{delno določena matrika}. $M$ označuje vhodno matriko, ki ima nekatere elemente neznane.
  \item \textbf{Nuklearna norma} je definirana kot \[
          \nnorm{X} = \sum_{i = 1}^{n} \sigma_i(X),
        \] pri čemer $\sigma_i(X)$ označuje $i$-to največjo singularno vrednost matrike $X$.
  \item Produkt $\trOp{A}{B}$ je definiran kot \[
          \trOp{A}{B} = \tr(AB^T)
        \]
  \item Oznaka $A \succeq 0$ označuje, da je matrika $A$ pozitivno semidefinitna. Velja $A \succeq 0 \iff x^TAx \ge 0 \text{ za vsak } x \in \mathbb{R}^n$
  \item Za množico subgradientov v točki $x_0$, označeno z $A_{x_0}$ konveksne funkcije $f: \mathbb{R}^n \rightarrow \mathbb{R}$ velja 
  \[
    c \in A_{x_0} \iff \forall x: f(x) - f(x_0) \geq c^T(x - x_0),
  \] kjer $c \in \mathbb{R}^n$.
         % \item Z oznako $M \in \mathbb{R}^{n_1 \times n_2}$ označujemo bitno matriko M, ki predstavlja masko, s katero označimo katere vrednosti poznamo in katere ne. Vrednost 1 označuje, da je vrednost na tisti poziciji znana, medtem ko 0 označuje, da ni. 
\end{enumerate}

\section{Minimizacija nuklearne norme}
Ker je minimizacija ranga matrike NP-poln problem, so se razvile druge metode, ki samo kompleksnost problema zmanjšujejo. Minimizacija nuklearne norme (Nuclear Norm Minimization) se zanaša na dejstvo, da je rang matrike povezan z nuklearno normo matrike. Ta je definirana kot
\[
  ||X||_* = \sum_{i = i}^{n} \sigma_i(X)
\] \todo{se lahko sklicujem?}
Dokazano je bilo, da nuklearna norma predstavlja konveksno ovojnico ranga. \cite{NNM-PHD} Konveksna ovojnica funkcije $f : \mathcal{C} \rightarrow \mathbb{R}$ je največja konveksna funkcija $g$ tako da velja $f(x) \geq g(x)$ za vse $x \in \mathcal{C}$. \cite{Survey-NKS19}

Minimizacijo nuklearne norme je možno pretvoriti v semidefinitni problem, ki ga lahko rešujemo z različnimi pripomočki, na primer SeDuMi \cite{SeDuMi}.

Standardna oblika semidefinitnega programa je
\begin{align*}
    \min_Y \hspace{0.5cm} &\trOp{C}{Y} \\
    \text{tako da} \hspace{0.5cm} &\trOp{A_k}{Y} = b_k, \hspace{0.5cm} k = 1, \hdots , l\\
    &Y \succeq 0
\end{align*}
Kjer je $C$ dana matrika, $\{A_k\}$ in $\{b_k\}$ pa množici matrik in vektorjev. 

Problem minimizacije nuklearne norme lahko zapišemo kot 
\begin{align*}
    \min_{X, t} \hspace{0.5cm} &t \\
    \text{tako da} \hspace{0.5cm} &||X||_* \leq t,\\
    &\proj(X) = \proj(M)
\end{align*}
Trdimo, da za matriko $X \in \mathbb{R}^{n_1 \times n_2}$ in $t \in \mathbb{R}$ 
velja $||X||_* \leq t \iff \exists W_1\in \mathbb{R}^{n_1 \times n_1}, W_2 \in \mathbb{R}^{n_2 \times n_2}$ tako da \cite{NNM-PHD} \todo{se lahko sklicujem?}
\begin{align*}
    &Y = \begin{bmatrix}
        W_1 & X \\
        X^T & W_2
    \end{bmatrix}, \\
    &Y \succeq 0, \hspace{0.2cm} \tr(Y) \leq 2t
\end{align*} 

Minimizacijski problem lahko tako redefiniramo kot 
\begin{align*}
    \min_{Y,t} \hspace{0.5cm} &2t\\
    \text{tako da} \hspace{0.5cm} & Y \succeq 0,\\
    &\trOp{Y}{A_{a,b}} = b_{a,b}
\end{align*} 
kjer je $A_{a,b} \in \mathbb{R}^{(n_1 + n_2) \times (n_1 + n_2)}$ matrika v množici $A$. Velja 
$A_{a, b} \in A \iff (a, b) \in \Omega$. Matrika $A_{a,b}$ ima vse elemente ničelne, razen na mestu $(a, n_1 + b)$, kjer ima vrednost $1$.
Podobno je definiran tudi $b_{a,b} \in \mathbb{R}$, ki ima vrednost $M_{a, b}$.
Ker velja 
\[
    \trOp{A}{B} = \sum_{i}^{n_1} \sum_{j}^{n_2} a_{i,j}b_{i,j}
\] je lahko videti, da je tak pogoj smiselen. Programi za reševanje SDP pa lahko tako obliko že sprejmejo.
\cite{Survey-NKS19}


% Za SVD razcep matrike $X = U\Sigma V^T$ velja 
% \[
%     \tr \begin{bmatrix}
%         UU^T & -UV^T \\
%         -VU^T & VV^T
%     \end{bmatrix}
%     \begin{bmatrix}
%         Y & X \\
%         X^T & Z
%     \end{bmatrix} \geq 0
% \] \todo{zakaj je prva matrika pozitivno semidefinitna?}
% ker je vsota diagonalnih elementov produkta dveh pozitivnih semidefinitnih matrik vedno nenegativna. Tako vemo, da 
% \[
%     \tr(UU^TY) - \tr(UV^TX^T) - \tr(VU^TX) + \tr(VV^TZ) \geq 0
% \]

% Po \cite{CR08} lahko problem definiramo kot
% \begin{align*}
%   \min    & \hspace{0.5cm} \tr(Y)                                    \\
%   \text{tako da} & \hspace{0.5cm} (Y, A_k) = b_k, k = 1, \cdots , |\Omega| \\
%                     & \hspace{0.5cm} Y \succcurlyeq 0
% \end{align*}
% kjer
% \begin{align*}
%   &Y = \begin{bmatrix}
%     W_1 & X   \\
%     X^T & W_2
%   \end{bmatrix}
% \end{align*} tak problem pa lahko že rešujemo s semidefinitnimi programi.
\section{Algoritem praga singularnih vrednosti}\label{2807-1441}
\textbf{Algoritem praga singularnih vrednosti (SVT)} \cite{CCS} uporabi idejo, da imajo matrike z majhnim rangom nekaj velikih singularnih vrednosti, ostale pa 0 ali pa vsaj blizu 0. Ključna parametra v SVT-ju sta \textit{izbira premika} in \textit{izbira praga},  
%Za svoje delovanje uvede dva nova pomembna koncepta, prvi je premik, drugi pa prag, potreben za uporabo operatorja $\shrink_\tau$ \eqref{1007-1959}. 
algoritem pa temelji na iteraciji
\begin{align}
\label{2407-1910}
        X^{(k)} &= \shrink_\tau(Y^{(k-1)}), \\
        Y^{(k)} &= Y^{(k-1)} + \delta_k \proj(M - X^{(k)}), 
\end{align}
kjer so $\tau > 0$ izbran prag, $\delta_k$ izbran premik, $X^{(0)} = 0 \in \mathbb{R}^{n_1 \times n_2}$ in
$Y^{(0)} = 0 \in \mathbb{R}^{n_1 \times n_2}$. \cite{CCS}

V nadaljevanju bomo opisali glavno idejo zgornje iteracije. V grobem pa temelji na uporabi metode za iskanje vezanih ekstremov, kjer elementi matrik $Y^{(k)}$ predstavljajo Lagrangove množitelje. 

Uvedimo funkcijo 
\begin{align}
    \label{1007-2007}
    f_\tau(X) = \tau\nnorm{X} + \frac{1}{2}\fnorm{X}^2
\end{align}
in optimizacijski problem
\begin{align}
\label{2706-0957}
\begin{split}
    \min_{X\in \mathbb R^{n_1\times n_2}} & \hspace{0.5cm} f_\tau(X), \\
    \textrm{pri pogojih} & \hspace{0.5cm} \proj(X) = \proj(M).
\end{split}
\end{align}
Opazimo lahko, da za velike vrednosti $\tau$ velja $f_\tau(X) \approx \tau\nnorm{X}$, kar pomeni, da bo s primerno izbranim $\tau$, optimizacijski problem minimiziral nuklearno normo.

Denimo, da želimo poiskati minimum funkcije $f(x)$ pri pogojih $g_1(x)=g_2(x)=\ldots=g_{k}(x)=0$.
V teoriji vezanih ekstremov se za tovrstne probleme uvede \textbf{Lagrangeovo funkcijo}
\[\mathcal{L}(x, \lambda_1,\lambda_2,\ldots,\lambda_k) = f(x) + \lambda_1 g_1(x)+\lambda_2g_2(x)+\ldots+\lambda_kg_k(x),\]
nato pa išče ekstreme med njenimi stacionarnimi točkami.
Problemu \eqref{2706-0957} lahko priredimo Lagrangeovo funkcijo
\[
    \mathcal{L}(X, Y) = f_\tau(X) + \left< Y, \proj(M - X) \right>,
\] 
nato pa iščemo njene stacionarne točke.
Zaradi velikega števila parametrov pa ta pristop navadno ni izvedljiv, zato se v SVT-ju za iskanje ekstremov $\mathcal{L}(X, Y)$ uporabi t.i.\ \textit{Uzawa algoritem} \cite{CCS}. Ta ekstreme išče prek iterativnega postopka:
\begin{align}
        X^{(k)} &=  \argmin_{X} \mathcal{L}(X^{(k)}, Y^{(k-1)}) \label{1007-2018},\\
        Y^{(k)} &= Y^{(k-1)} + \delta_k \proj(M - X^{(k)}). \label{1007-2019}
\end{align}

Izkaže se, da je rešitev \eqref{1007-2018} enaka $\shrink_\tau(Y^{(k-1)})$. To spodaj dokažemo, še prej pa izpeljimo pomožen rezultat.

\begin{trditev}
Velja:
\begin{align}
    \argmin_X  \mathcal{L}(X, Y)
    = \argmin_X \Big(\tau\nnorm{X} + \frac{1}{2}\fnorm{X - Y}^2\Big)
\end{align}
\end{trditev}

\begin{proof}
Trditev sledi iz krajšega računa:
\begin{align*}
    &\argmin_X  \Big(\tau\nnorm{X} + \frac{1}{2}\fnorm{X - Y}^2\Big) \\
    &=\argmin_X \Big(\tau\nnorm{X} + \frac{1}{2}\trOp{X - Y}{X - Y}\Big)\\
    &=\argmin_X \Big(\tau\nnorm{X} + \frac{1}{2}(\fnorm{X}^2 - 2\trOp{X}{Y} + \fnorm{Y}^2)\Big) \\ 
    &=\argmin_X \Big(\tau\nnorm{X} + \frac{1}{2}\fnorm{X}^2 - \trOp{X} {\proj(Y)}\Big)\\
    &=\argmin_X \Big(\tau\nnorm{X} + \frac{1}{2}\fnorm{X}^2 + \tr(- \proj(X)Y^T) + \tr(\proj(M)Y^T)\Big)\\
    &=\argmin_X \Big(\tau\nnorm{X} + \frac{1}{2}\fnorm{X}^2 + \tr(\proj(M - X)Y^T)\Big)\\
    &=\argmin_X \Big(\tau\nnorm{X} + \frac{1}{2}\fnorm{X}^2 + \trOp{Y}{\proj(M - X)}\Big)\\ 
    &= \argmin_X \mathcal{L}(X, Y)
\end{align*}
kjer smo v prvi enakosti uporabili definicijo Frobeniusove norme, v drugi bilinearnost skalarnega produkta, 
v tretji smo ignorirali konstanto $\fnorm{Y}^2$, saj ne vpliva na rezultat, 
in upoštevali $\proj(Y^{(k)}) = Y^{(k)}$ za vse $k \in \mathbb{N}$. Zadnje dejstvo sledi iz definicije \eqref{1007-2019} in $Y^{(0)} = 0$. V četrti enakosti smo upoštevali $\trOp{X}{\proj(Y)} = \trOp{\proj(X)}{Y}$, kar je lahko videti, ko se spomnimo, da za skalarni produkt $\trOp{A}{B}$ matrik $A, B \in \mathbb{R}^{n_1 \times n_2}$ velja 
\[
    \trOp{A}{B} = \sum_{i = 1}^{n_1}\sum_{j = 1}^{n_2} a_{ij}b_{ij}.
\]
Prišteli smo tudi konstanto $\tr(\proj(M)Y^T)$, ki ne vpliva na rezultat.
V peti enakosti smo upoštevali linearnost sledi in aditivnost operatorja $P_\Omega$,
v šesti definicijo skalarnega produkta in v zadnji definicijo Lagrangeove funkcije.
\end{proof}
\begin{theorem} \label{1907-2240}
Za matriki $X \in \mathbb{R}^{n_1 \times n_2}$ in 
$Y \in \mathbb{R}^{n_1 \times n_2}$ velja:
\begin{align}
    \label{2906-1056}
    \shrink_\tau(Y) = \argmin_{X} \Big(\frac{1}{2} \fnorm{X-Y}^2 + \tau\nnorm{X} \Big) 
\end{align}
\end{theorem}

\begin{proof} 
Najprej se spomnimo definicije konveksne funkcije. Funkcija $f$ je konveksna, če za katerikoli dve točki $x_1, x_2$ v domeni funkcije $f$ velja, da je premica čez ti dve točki na odseku med tema dvema točkama nad grafom funkcije $f$.
Funkcija $h(X) := \frac{1}{2} \fnorm{X-Y}^2 + \tau\nnorm{X} $ je konveksna funkcija, saj potreben pogoj trikotniške neenakosti matričnih norm zagotavlja, da je matrična norma konveksna funkcija. Vsota konveksnih funkcij pa je prav tako konveksna funkcija. 
%\todo{ali je potrebno citirati} 
Zaradi konveksnosti funkcije $f$ je 
subgradient v vsaki točki iz domene dobro definiran.
Tega smo definirali v razdelku \ref{2607-1502}.
Matrika $Z\in \mathbb R^{n_1\times n_2}$ je subgradient funkcije $f$ v točki 
$X_0\in \mathbb R^{n_1\times n_2}$,
če velja 
\[\forall X\in \mathbb R^{n_1\times n_2}: 
f(X) \geq  f(X_0) + \trOp{Z}{X - X_0}.\]
Iz definicije subgradienta sledi, da bo imela funkcija $f$ minimum v točki $X'$ natanko tedaj,
ko bo ničelna matrika $\mathbf{0}$ eden izmed subgradientov funkcije $f$ v točki $X_0$.
\iffalse 
V minimumu funkcije $f$ bo eden od subgradientov iščemo minimum $X'$ funkcije $f$,  iščemo tako točko $X'$, da bo
\CG{eden izmed subgradientov po spremenljivki $X$} v točki $X'$ enak 0.  
\fi 

Izkaže se \cite{CCS}, da je množica subgradientov nuklearne norme v točki $X$ enaka
\begin{equation}  \label{2807-0932}  
    \partial\nnorm{X} = \{UV^* + W: W \in \mathbb{R}^{n_1 \times n_2}, U^*W = \mathbf{0}, WV = \mathbf{0}, \norm{W}_2 \leq 1 \},
\end{equation}
kjer $U \Sigma V^T$ predstavlja SVD razcep matrike $X$. S krajšim računom lahko preverimo, da za vsak par $(i,j)$ velja $\frac{\partial}{\partial X_{ij}}\fnorm{X-Y}^2 = 2(X_{ij} - Y_{ij})$. 
Če te parcialne odvode zložimo v matriko, dobimo $2(X-Y)$. Od tod sledi, da je $X-Y$ eden od subgradientov funkcije $h_1(X):=\frac{1}{2}\fnorm{X-Y}^2$ v točki $X$ \cite[3.1.3]{boyd2004convex}.
Iz teh dveh premislekov sledi, da bo $X'$ minimum $h$ natanko tedaj, ko velja 
\begin{equation} \label{2807-0940}
    \mathbf{0} \in X' - Y + \tau \partial\nnorm{X'}.
\end{equation}
Trditev izreka bo sledila, če pokažemo, da velja $X' = \shrink_\tau(Y)$. Najprej razčlenimo SVD razcep matrike $Y$ kot 
\[
    Y = U_0\Sigma_0V_0^T + U_1\Sigma_1V_1^T,
\]
kjer se $U_0, \Sigma_0$ in $V_0$ nanaša na singularne vrednosti in pripadajoče singularne vektorje večje od $\tau$, $U_1, \Sigma_1$ in $V_1$ pa tiste manjše od $\tau$. Pokazati želimo, da velja 
\[
    X' = U_0(\Sigma_0 - \tau I)V_0^T,
\] 
kar pa je po definiciji natanko enako operatorju $\shrink_\tau(Y)$. S pretvorbo \eqref{2807-0940} pridemo do zapisa
\begin{equation} \label{2807-0934}
    Y- X' \in \tau \partial \nnorm{X'}.
\end{equation}
Sedaj iščemo tako matriko $W$, da bodo zanjo držali pogoji, podani v \eqref{2807-0932}, prav tako pa bo po \eqref{2807-0934} držalo $Y - X' = \tau (U_0V_0^T + W)$. Spodaj bomo preverili, da je primerna izbira za $W$ enaka $W= \tau^{-1} U_1 \Sigma_1 V_1^T$. 
Veljavnost željene enakosti
preverimo s kratkima računoma:
\begin{align*}
    Y-X' &= U_0\Sigma_0V_0^T + U_1\Sigma_1V_1^T - U_0(\Sigma_0 - \tau I)V_0^T \\ 
    &= U_0(\Sigma_0 - \Sigma_0 + \tau I)V_0^T + U_1\Sigma_1 V_1^T  \\
    &= \tau U_0 V_0^T + U_1\Sigma_1 V_1^T
\end{align*}
in 
\begin{align*}
    \tau(U_0 V_0^T + W) &= \tau(U_0V_0^T + \tau^{-1} U_1 \Sigma_1 V_1^T)\\ 
    &= \tau U_0 V_0^T + U_1 \Sigma_1 V_1^T. 
\end{align*}
Sedaj je zgolj potrebno pokazati, da veljajo potrebne lastnosti matrike $W$.
Po sami definiciji SVD vemo, da so stolpci matrik $U$ in $V$ ortogonalni. Torej velja $U_0^TW = 0$ in $WV_0 = 0$. Ker pa ima matrika $\Sigma_1$ vse elemente manjše od $\tau$, velja tudi $\norm{W}_2 \leq 1$. $\norm{A}_2$ je namreč definirana kot največja singularna vrednost matrike $A$. S tem smo pokazali, da je $Y - X' \in \tau \partial \nnorm{X'}$.
\end{proof}
Z uporabo \eqref{1007-2018}, \eqref{1007-2019}
in izreka \ref{1907-2240}
res pridemo do iteracije
\eqref{2407-1910}, ki jo uporablja algoritem SVT.
\iffalse
Tako res pridemo 
Po trditvi lahko sedaj zapišemo algoritem \eqref{1007-2018} - \eqref{1007-2019} kot \cite{CCS}
\[
    \begin{cases}
        X^k = \shrink_\tau(Y^{k-1}) \\
        Y^k = Y^{k-1} + \delta_k \proj(M - X^k) 
    \end{cases}
\]
\fi

\subsection{Nastavljanje parametrov $\tau$ in $\delta$} \label{1907-1648}
Opazimo lahko, da algoritem SVT potrebuje dva parametra, $\tau$ in $\delta$, ki ju moramo izbrati 
že pred vstopom v algoritem.

Po priporočilih \cite{CCS}
sta primerni izbiri za $\delta$ in $\tau$
enaki 
\[
    \delta = 1.2\, \dfrac{n_1 n_2}{m}\qquad\text{in}\qquad
    \tau = 5n,
\]
pri čemer je $\tau$ naveden za kvadratne $n\times n$ matrike.
V poglavju z rezultati smo prvotno uporabljali ti dve konstanti, pri 
čemer smo zaradi
pravokotnosti $n_1\times n_2$ matrik uporabljali
$\tau = 5\frac{n_1+n_2}{2}$.
\iffalse
Medtem, ko so koraki v samem algoritmu definirani kot množica korakov, 
smo v okviru rezultatov diplomske naloge, prvotno za premik uporabljali konstanto, ter korak nastavili na 
 po priporočilih \cite{CCS}. 
 \fi
\iffalse
Prav tako članek \cite{CCS} navaja, da je za matrike velikosti $\mathbb{R}^{n \times n}$ smiselno nastaviti $\tau = 5n$, vendar sem v moji implementaciji zaradi posploševanja na nekvadratne matrike, za matrike velikosti $\mathbb{R}^{n_1 \times n_2}$ parameter nastavil na
\[
    \tau = 5\, \frac{n_1+n_2}{2}
\]
\fi
V naših testiranjih se je izkazalo, da sta taka parametra dobra za večje matrike. V razdelku \ref{1307-2251} pa bomo videli, da moramo pri manjših matrikah pogosto zmanjšati premik in povečati prag.
\section{Algoritem minimizacije prirezane nuklearne norme}
Že samo ime nam pove, da bo algoritem minimizacije prirezane nuklearne norme, oziroma TNNM podoben algoritmu NNM. Ideja tega algoritma pa je, da uporabimo dodatno informacijo $r \in \mathbb{N}$ o rangu originalne, nezašumljene matrike.

Sam algoritem uvede tako imenovano \textbf{$r$-prirezano nuklearno normo}, ki je za matriko $X \in \mathbb{R}^{n_1 \times n_2}$ definirana kot vsota $\min(n_1,n_2) - r$ najmanjših singularnih vrednosti
\[
    \norm{X}_r = \sum^{\min(n_1, n_2)}_{i = r + 1} \sigma_i(X)
\]
TNNM rešuje problem \cite{TNNM-HZYLH12}
\begin{align*}
    \min_X               & \hspace{0.5cm} \norm{X}_r \numberthis \label{1107-1305} \\
    \textrm{pri pogojih} & \hspace{0.5cm} \proj(X) = \proj(M)
\end{align*}
Cilj algoritma je torej čim bolj zmanjšati najmanjše singularne vrednosti, medtem ko velikih ne omejujemo. S tem problem minimizacije omilimo.

Problem \eqref{1107-1305} je ekvivalenten
\begin{align*}
    \min_X               & \hspace{0.5cm} \nnorm{X} - \sum_{i=1}^{r} \sigma_i \\
    \textrm{pri pogojih} & \hspace{0.5cm} \proj(X) = \proj(M)
\end{align*}
Za nadaljnje korake bomo potrebovali naslednji izrek
\begin{theorem}
    Za $X \in \mathbb{R}^{n_1 \times n_2}$, $A \in \mathbb{R}^{r \times n_1}$, $B \in \mathbb{R}^{r \times n_2}$ in $r \in \mathbb{N}$, pri pogojih $r \leq min(n_1, n_2)$, $AA^T = I_{r}, BB^T = I_{r}$ velja:
    \begin{align*}
        \tr(AXB^T) \leq \sum_{i=1}^{r} \sigma_i(X)
    \end{align*}
\end{theorem}

\begin{proof}
    Za dokaz uporabljamo von Neumannovo neenakost sledi \cite{TNNM-HZYLH12}, s katero lahko zapišemo
    \[
        \tr(AXB^T) = \tr(XB^TA) \leq \sum^{\min(n_1, n_2)}_{i=1} \sigma_i(X) \sigma_i(B^TA)
    \]
    Enakost $\tr(AXB^T) = \tr(XB^TA)$ sledi iz dejstva, da je sled produkta matrik invariantna pod cikličnimi permutacijami.

    Po definiciji lahko singularne vrednosti matrike $Y$ najdemo tako, da najdemo korene nenegativnih lastnih vrednosti matrike $Y^TY$. Tako lahko povemo, da so singularne vrednosti matrike $B^TA$ enake lastnim vrednostim matrike $A^TBB^TA$. Izraz razpišemo v
    \[
        A^TBB^TA = A^TI_rA = A^TA
    \]
    %https://shorturl.at/pqrRX
    Ker pa velja, da imata matriki $XY$ in $YX$ enake neničelne lastne vrednosti, ter vemo da $AA^T = I_r$, lahko povemo, da ima produkt $B^TA$ $r$ singularnih vrednosti enakih 1, saj ima $I_n$ $n$ lastnih vrednosti enakih 1.
    Tako lahko sedaj razpišemo izraz
    \[
        \sum^{\min(n_1, n_2)}_{i=1} \sigma_i(X) \sigma_i(B^TA) = \sum^{r}_{i=1} \sigma_i(X)
    \]
    Ugotovili smo, da velja
    \begin{equation}
        \label{2806-1320}
        \tr(AXB^T) \leq \sum^{r}_{i=1} \sigma_i(X)
    \end{equation}
\end{proof}

\begin{theorem}
    Za SVD razcep matrike $X = U \Sigma V^T$, ter matriki
    $A = (u_1, \hdots , u_r)^T$ in $B = (v_1, \hdots , v_r)^T$,
    kjer je $u_i$ $i$-ti stolpec matrike $U$ ter $v_i$ $i$-ti stolpec
    matrike $V$, velja \[\tr(AXB^T) = \sum^{r}_{i=1} \sigma_i(X)\]
\end{theorem}

\begin{proof}
    \begin{align*}
        \tr(AXB^T) & = \tr((u_1, \hdots , u_r)^T X (u_1, \hdots , u_r)^T)                                                                                                                                                                   \\
                   & = \tr((u_1, \hdots , u_r)^T U \Sigma V^T (u_1, \hdots , u_r)^T)                                                                                                                                                        \\
                   & = \begin{bmatrix} I_r & 0 \\ 0 & 0 \end{bmatrix} \Sigma \begin{bmatrix} I_r & 0 \\ 0 & 0 \end{bmatrix}                                                                                                                 \\
                   & = \tr(\begin{bmatrix} \sigma_1 \\[-8pt] & \ddots & & \\[-8pt] & & \sigma_r \\[-8pt] & &  & 0 \\[-8pt]  & & & & \ddots \\[-8pt] & & & & & 0 \end{bmatrix}) = \sum_{i = 1}^{r} \sigma_i(X)  \numberthis\label{2806-1321}
    \end{align*}
\end{proof}
Z združitvijo dokazov \eqref{2806-1320} in \eqref{2806-1321} zapišemo
\[
    \max_{AA^T = I, BB^T = I} \tr(AXB^T) = \sum^{r}_{i = 1} \sigma_i(X)
\]
Torej je optimizacijski problem ekvivalenten
\begin{align*}
    \min_X \hspace{0.5cm}             & \nnorm{X} - \max_{AA^T = I, BB^T = I} \tr(AXB^T) \\
    \text{pri pogojih} \hspace{0.5cm} & \proj(X) = \proj(M)
\end{align*}

Glede na vse ugotovitve, nato nastavimo iterativni algoritem, tako da, izračunamo $X^0 = \proj(M)$. V $i$-ti iteraciji izračunamo $A^i$ in $B^i$, tako da izračunamo SVD razcep $X^i = U \Sigma V^T$, ter $A$ nastavimo kot prvih $r$ stolpcev matrike $U$, $B$ pa kot prvih $r$ stolpcev matrike $V$. $X^{i+1}$ lahko sedaj izračunamo kot \cite{TNNM-HZYLH12}
\begin{align*}
    \min_X \hspace{0.5cm}         & \nnorm{X} - \tr(A^iX(B^i)^T) \\
    \text{tako da} \hspace{0.5cm} & \proj(X) = \proj(M)
\end{align*}
Če minimizacijo še malo obrnemo, pa lahko problem rešujemo z uporabo algoritma ADMM. \todo{iz kje pride izracun $Y^{i+1}$}
\begin{align*}
    \min_X \hspace{0.5cm}         & \nnorm{X} - \tr(A^iW(B^i)^T) \numberthis\label{1307-1527} \\
    \text{tako da} \hspace{0.5cm} & W=X, \hspace{0.2cm} \proj(W) = \proj(M)
\end{align*}
Ta problem ponovno zapišemo s pomočjo Lagrangeove funkcije, le da algoritem ADMM definira še regularizacijski parameter $\beta$. \cite{TNNM-HZYLH12}
\[
    \mathcal{L}(X, Y, W, \beta) = \nnorm{X} - \tr(A_l W B_l^T) + \frac{\beta}{2} \fnorm{X - W}^2 + \tr(Y^T(X-W))
\]
Opazimo lahko, da funkcija definira zgolj pogoj $X = W$. Videli bomo, da algoritem pogoj $\proj(X) = \proj(M)$ definira posredno, s popravljanjem znanih vrednosti (korak \eqref{2906-1248} spodaj).%\todo{ali lahko citiram naprej}

Matriko $X^{k+1}$ ponovno definiramo kot
\[
    X^{k+1} = \arg \min_X \mathcal{L}(X, Y^k, W^k, \beta)
\]
Z ignoriranjem konstantnih členov, pa lahko tako zapišemo
\begin{align*}
    X^{k+1} & = \arg \min_X \hspace{0.2cm} \nnorm{X} + \frac{\beta}{2}\trOp{X-W^k}{X-W^k} + \frac{\beta}{2}\trOp{\frac{2}{\beta}Y}{X}       \\
            & = \arg \min_X \hspace{0.2cm} \nnorm{X} + \frac{\beta}{2}\tr(X^TX - X^TW^k - W^{k^T}X - W^{k^T}W^k + \frac{2}{\beta}Y^{k^T}X )
\end{align*}
Z namenom faktorizacije dodamo konstantne člene.
\begin{align*}
     & \arg \min_X \hspace{0.2cm} \nnorm{X} + \frac{\beta}{2}\tr(X^T(X - W^k + \frac{1}{\beta}Y^k) - W^{k^T} (X - W^k + \frac{1}{\beta}Y^k) \\
     & +\frac{1}{\beta}Y^{k^T}(X - W^k + \frac{1}{\beta}Y^k))                                                               \\
     & = \arg \min_X \hspace{0.2cm} \nnorm{X} + \frac{\beta}{2} \fnorm{X-W^k + \frac{1}{\beta}Y^k}^2                                       \\
     & = \arg \min_X \hspace{0.2cm} \frac{1}{\beta}\nnorm{X} +  \frac{1}{2}\fnorm{X-(W^k - \frac{1}{\beta}Y^k)}^2
\end{align*}
\CB{Po enakkosti v izreku ... ali po izreku ...} Po izreku \ref{1907-2240} pa tako lahko zapišemo
\[
    X^{k+1} = \shrink_\frac{1}{\beta}(W_k - \frac{1}{\beta} Y_k)
\]
Matriko $W$ podobno izračunamo kot
\begin{align*}
    W^{k+1} & = \arg \min_{W} \mathcal{L}(X^{k+1}, Y^k, W, \beta)                                       \\
            & = \arg \min_W \frac{\beta}{2} \fnorm{W - (X^{k+1} + \frac{1}{\beta}(A_l^T B_l + Y_k)) }^2
\end{align*}
Lahko je videti, da velja
\[
    W^{k+1} = X^{k+1} + \frac{1}{\beta}(A_l^T B_l + Y_k)
\]
Zaradi pogoja $\proj(W) = \proj(M)$ pa tiste elemente, ki poznamo, popravimo. \cite{TNNM-HZYLH12}\todo{malo spremenjeno iz clanka}
\begin{equation}
    \label{2906-1248}
    W_{k+1} = W_{k+1} + \proj(M - W_{k+1})
\end{equation}
\CG{Korak preprosto spremeni mesta, kjer vrednosti poznamo na znane vrednosti, ostale pa pusti, kakršne so.}

\section{Izmenjajoč gradientni spust}
Ker je računanje SVD razcepa zahtevna operacija, saj ima časovno zahtevnost
$O(n^3)$, je bilo predlaganih nekaj algoritmov, ki za svoje delovanje ne potrebujejo SVD-ja. Algoritem Izmenjajočega gradientnega spusta, oziroma v nadaljevnu ASD sloni na računanju gradienta in premikanja po njem. Ideja algoritma je najti dve matriki $X \in \mathbb{R}^{n_1 \times r}$ ter $Y \in \mathbb{R}^{r \times n_2}$, tako da velja $\proj(M) = \proj(XY)$. Vidimo lahko, da ponovno potrebujemo informacijo o rangu matrike, ki jo rekonstruiramo. Ker imata tako $X$ in $Y$ kvečjemu rang $r$, vemo, da tudi njun produkt $XY$ ne bo imel ranga večjega od $r$. 

Cilj algoritma je minimizirati
\[
    \min_{X, Y} \hspace{0.5cm} \frac{1}{2}\, \fnorm{\proj(M) - \proj(XY)}^2
\] 
Algoritem minimizacijo razdeli na dve, nato pa izmenično rešuje eno in nato drugo kot \cite{AST-TK15}
\begin{align*}
    X_{i+1} &= \arg \min_{X} \hspace{0.3cm} \fnorm{\proj(M) - \proj(XY_i)}^2 \\
    Y_{i+1} &= \arg \min_{Y} \hspace{0.3cm} \fnorm{\proj(M) - \proj(X_{i+1}Y)}^2
\end{align*}


Za uporabo algoritma ASD potrebujemo odvoda funkcije $f(X,Y) = \frac{1}{2}\, \fnorm{\proj(M) - \proj(XY)}^2$, ki ga izračunamo za vsak element posebej.
\begin{align*}
    \frac{\partial}{\partial x_{a,b}} f &= \frac{\partial}{\partial x_{a,b}} \frac{1}{2}\, \fnorm{\proj(M) - \proj(XY)}^2  \\
    &= \frac{\partial}{\partial x_{a,b}} \frac{1}{2}\, \sum_{i}^{n_1}\sum_{j}^{n_2}(\delta_{i,j}m_{i,j} - \delta_{i,j}\sum_{k}^{r}(x_{i,k}y_{k,j}))^2  \\
    &= \, \sum_{j}^{n_2}(\delta_{a,j}m_{a,j} - \delta_{i,j}\sum_{k}^{r}(x_{a,k}y_{k,j}))(-y_{b,j}) \implies \\
    &\implies \frac{\partial}{\partial X}f = -(\proj(M) - \proj(XY))Y^T
    % &= \sum_{i,j}(\delta_{i,j}m_{i,j} - \delta_{i,j}\sum_{k}(x_{i,k}y_{k,j}))
    % (\sum_{k}y_{k,j}) = \\
    % &= (P(M) - P(XY))Y^T
\end{align*}
kjer 
\[
    \delta_{i,j} = \begin{cases}
        1, (i, j) \in \Omega \\
        0, (i, j) \notin \Omega
    \end{cases}
\]
Na podoben način bi lahko pokazali tudi
\[
    \frac{\partial}{\partial Y} = -X^T (\proj(M) - \proj(XY))
\]
Poiščimo še najboljši korak gradientnega spusta.
Cilj je pokazati, da je premik po gradientu s korakom $t_x$ najboljši.
Najprej definirajmo sam premik, kot
\[
    X^{k+1} = X^k - t_X \nabla f_Y(X)
\]
Ker želimo, da bodo znane vrednosti produkta $X^{k+1}Y^{k}$ kar se da podobne znanim vrednostim matrike $M$, nastavimo $t_x$ kot 
\begin{align*}
    t_x &= \arg \min_t \hspace{0.3cm} g(t)
\end{align*} kjer
\begin{align*}
    g(t) &= \frac{1}{2} \fnorm{\proj(M) - \proj((X - t\nabla f_Y(X))Y)}^2
\end{align*}
\CG{
\begin{align*}
    g(t) &= \frac{1}{2} \fnorm{\proj(M - XY)}^2 - t\tr(\proj(M - XY)\proj(\nabla f_Y(X)Y)^T) + \frac{t^2}{2}\fnorm{\proj(\nabla f_Y(X)Y)}^2
\end{align*}}
Ker je \cite{AST-TK15} \todo{ali lahko tako?}
\[
  g'(t) = -\fnorm{\nabla f_Y(X)}^2 + t\fnorm{\proj(f_Y(X)Y)}^2 
\]
vidimo, da funkcija $g(t)$ doseže minimum pri 
\[
  t_x = \frac{\fnorm{\nabla f_Y(X)}^2}{\fnorm{\proj(\nabla f_Y(X)Y)}^2}  
\]
Podobno velja za korak v smeri gradientnega spusta matrike $Y$, kjer
\[
  t_y = \frac{\fnorm{\nabla f_X(Y)}^2}{\fnorm{\proj(X \nabla f_X(Y))}^2}  
\]
\section{LMaFit}
LMaFit je algoritem, ki podobno kot ASD, rešuje problem matričnih napolnitev z uporabo dveh manjših matrik $X^{n_1 \times r}$ in $Y^{n_1 \times r}$. Podobno kot v prejšnjih algoritmih rešujemo problem 
\begin{align*}
    \min_{X, Y, Z}& \hspace{0.5cm} \frac{1}{2}||XY - Z||_F^2\\
    \text{tako da}& \hspace{0.5cm} \proj(M) = \proj(Z)
\end{align*}
le da sam problem rešujemo s pomočjo Moore-Penrose inverza (Označen z $\dagger$).

Lahko je videti, da bo funkcija $f(X,Y,Z) = frac{1}{2}||XY - Z||_F^2$ imela najmanjšo vrednost, ko bo $XY = Z$. Zato uvedemo iterativni algoritem, ki izmenično posodablja $X, Y$ in $Z$ tako, da fiksira dve izmed matrik. Minimizacijo matrik $X$ in $Y$ torej dosežemo na naslednji način.
\begin{align*}
    X^{i+1}Y^{i} &= Z^{i} \hspace{0.5cm} \\
    X^{i+1} &= Z^{i}(Y^i)^\dagger
\end{align*}
Kjer se zanašamo na dejstvo, da nam množenje z Moore-Penrose inverzom z desne 
da najboljši možen rezultat.

Podobno posodobimo matriko $Y$.
\begin{align*}
    \hspace{0.5cm} X^{i+1}Y^{i+1} &= Z^{i}   \\
    Y^{i+1} &= (X^{i+1})^\dagger Z^{i}
\end{align*}

Sedaj le še posodobimo matriko Z, kot
\[
    Z^{i+1} = X^{i+1}Y^{i+1} + \proj(M - X^{i+1}Y^{i+1})
\]
Torej podobno kot prej poskušamo doseči $XY = Z$, vendar zaradi omejitve znanih vrednosti matrike M, na tistem mestu vrednosti popravimo. \cite{LMaFit-WY12}

