\chapter{Pregled področja} \label{1407-1010}

Raziskovanje problema matričnih napolnitev (PMN) je zelo aktivno področje, saj se pojavljajo številni članki z novimi pristopi in izboljšavami obstoječih metod.  Splošnost problema in številne uporabe pojasnjujejo motivacijo za iskanje še učinkovitejših oz.\ bolj ciljno usmerjenih algoritmov od obstoječih.

Eno prvih del, ki obravnava PMN, je \cite{MCPAS}. Delo opiše problem in njegovo naravo, poleg napolnitev minimalnega ranga pa obravnava tudi nekatere druge vrste napolnitev. Obravnavane so tudi metode za napolnjevanje matrik s ciljem pozitivne semidefinitnosti napolnitve ali pa s ciljem maksimizacije determinante. Oba problema sta bila obsežno študirana v zadnjih dveh desetletjih 20.\ stoletja \cite{Barret, Dancis,Woerdeman}.

Delo \cite{NNM-PHD} pretvori problem minimizacije ranga napolnitve v problem iz področja semidefinitnega programiranja. Ideje in dokazi tega dela služijo kot temelj za več algoritmov, ki so se razvili v zadnjih letih.
Članek \cite{CCS} opisuje algoritem SVT, katerega ideja temelji na uvedbi \textit{operatorja praga}, ki vse singularne vrednosti manjše od izbranega praga postavi na 0. Članek tudi pokaže, zakaj je tak način reševanja smiselen.
Članek \cite{TNNM-HZYLH12} opisuje algoritem TNNM, ki uporabi informacijo o rangu nezašumljene matrike, in problem rešuje s pomočjo znanega algoritma ADMM \cite{admmForNNM} iz področja reševanja problema vezanih ekstremov. 
Vir \cite{AST-TK15} opisuje algoritem ASD, ki išče matriki $X$ in $Y$, katerih produkt bo enak napolnjeni matriki. Matriki išče z uporabo gradientnega spusta in izračunom optimalnega koraka v smeri gradienta v vsaki iteraciji.  
Algoritem LMaFit, predstavljen v \cite{LMaFit-WY12}, prav tako išče najboljši produkt dveh matrik, vendar pa za iskanje uporablja Moore-Penroseov inverz, ki v vsaki iteraciji najde optimalen rezultat po metodi najmanjših kvadratov.
Obstajajo še številne druge metode, ki temeljijo na povsem drugačnih idejah \cite{admira,Riemannian,SETalgo} in si jih zaradi obsežnosti v tem diplomskem delu ne bomo pogledali. 


Temeljna literatura pri nastajanju diplomske naloge je bil pregledni članek \cite{Survey-NKS19}, ki opiše problem in v grobem predstavi več algoritmov ter jih primerja. Sicer večina opisov ni dovolj podrobnih, da bi lahko algoritme implementirali, vseeno pa je  članek dobra osnova, saj na kratko poda pregled literature iz področja PMN in ustrezne reference. V članku so algoritmi tudi testirani na izbranem problemu uporabe,
rezultati pa primerjani med seboj. Ravno ta primerjava pa je v večini drugih virih bolj skopo obravnavana.

Kot smo videli zgoraj, obstajajo številni članki s področja PMN. Večina člankov predstavi neko novo metodo, razloži idejo v ozadju, izpelje nekaj konvergenčnih rezultatov in testiranj na izbranem problemu.
Zelo malo pa je literature, ki različne metode primerja med seboj in poskuša klasificirati algoritme glede na to, za kateri problem so najprimernejši. Prav tako je navadno pomanjkljivo razloženo, zakaj je bilo testiranje narejeno ravno na izbranih podatkih. Ponekod gre za naključno generirane podatke iz točno določene porazdelitve s točno določeno strukturo, pri čemer od tod ni moč sklepati, kako bi se algoritem obnesel na nekoliko drugačnih podatkih, ki ne bi bili pridobljeni ravno na tovrsten način. V delu se zato želimo osredotočiti tudi na ta vidik, tj.\ na implementacijo algoritmov in testiranje na podatkih istega tipa.
