\chapter{Pregled področja} \label{1407-1010}
Področje matričnih napolnitev je trenutno zelo aktivno, s številnimi raziskovalci, ki na danem področju raziskujejo in iščejo nove načine reševanja problema. Algoritmi, ki problem rešujejo, so zaradi splošnosti problema zelo uporabni, kar pojasnjuje motivacijo po iskanju učinkovitih algoritmov.

Eno prvih del, ki sam problem opisuje je \cite{MCPAS}. Delo opisuje sam problem in njegovo naravo, vendar se ne osredotoča zgolj na napolnjevanje s ciljem minimalnega ranga. Delo opisuje tudi ideje za napolnitve matrik tako, da je napolnjena matrika pozitivno semidefinitna, kot tudi maksimizacijo determinante napolnjene matrike.

Doktorska disertacija \cite{NNM-PHD} podaja pomembne opise pretvorbe problema minimizacije ranga v problem iz področja semidefinitnega programiranja. Ideje in dokazi tega dela služijo kot temelj za algoritme, ki so se razvili v zadnjih letih. 

Članki \cite{CCS,TNNM-HZYLH12,AST-TK15,LMaFit-WY12} opisujejo določene algoritme, ter predstavijo nadgradnjo del in idej pred njimi. Glavne ugotovitve člankov so v tem delu povzete v njihovih razdelkih poglavja \ref{1407-1011}. Ta dela so bila ključna za samo implementacijo algoritmov.

Vodilna literatura tekom pisanja diplomske naloge je bil članek \cite{Survey-NKS19}, ki opisuje problem, ter mnoge algoritme. Medtem ko opisi pogosto niso bili dovolj podrobni, da bi lahko začel algoritme implementirati, je članek ponujal dobro razumljive opise algoritmov, kot tudi navedel vire, ki so pomagali pri implementaciji. Prav tako je članek podal pomembno primerjavo rezultatov različnih algoritmov.