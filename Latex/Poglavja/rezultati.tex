\chapter{Rezultati}
V tem poglavju bom opisal rezultate, ki sem jih pridobil. Kot sem že omenil, bo večji del preizkušanja programa opravljen na slikah, kjer bodo nekateri piksli manjkali. Gre za problem, ki ga je moč lepo vizualizirati, saj pogosto pri surovih podatkih ni lahko definirati njihovo uporabnost, brez številnih metod in računalniških operacij. 

Prav tako bom opisal točnost rezulatov različnih metod kot tudi čas izvajanja posameznih metod. Probleme bom zagnal tudi na različnih vrst podatkov, npr. podatkih ki so generirani normalno kot tudi enakomerno porazdeljeno.

Poglavje bom začel pisati, ko končam z vsemi implementacijami, pri delu pa si bom pomagal z orodjem Microsoft Excel, kjer bom lahko napake in čas izvajanja tudi grafično vizualiziral. 