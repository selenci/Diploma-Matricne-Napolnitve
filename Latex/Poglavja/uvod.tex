\chapter{Uvod}
\section{Motivacija}
Problem matričnih napolnitev sprejme matriko, največkat označeno z $M$, pri kateri so nekateri elementi označeni kot neznani. Problem nato sprašuje po vrednostih, ki jih lahko vstavimo v neznane vrednosti, tako da bo rang matrike najmanjši možen. Gre za NP-poln problem, zato ga poskušamo poenostaviti, ter reševati lažje probleme, ki vrnejo dovolj dobre, a ne optimalne rešitve. 

Problem je v zadnjih letih postal zelo popularen, z njim pa se ukvarjajo tako številni matematiki kot računalničarji. Zaradi splošnosti problema so metode za reševanje uporabne na številnih področjih. Literatura navaja številne načine uporabe \cite{Survey-NKS19}, saj lahko algoritem uporabljamo v priporočilnih sistemih, kjer generiramo predvidene ocene uporabnikov kot tudi programih za računanje razdalj med napravami. V tej diplomski nalogi se osredotočamo na problem razreševanja neznanih pikslov v slikah.

V diplomi bomo predstavili par algoritmov, ki rešujejo omenjen problem ter pokazali in razložili njihove ideje. Algoritmi so bili izbrani glede na njihovo popularnost in priznanost v literaturi. Prav tako poskrbimo, da so algoritmi primerno različni in temeljijo na drugačnih principih.

\section{Cilji}
Cilj diplomske naloge je predstaviti uporabnost algoritmov, ter razložiti zakaj delujejo. Ideje različnih algoritmov ter njihove pristope do problema poskušamo kar se da jasno predstaviti in interpretirati.

Medtem ko obstaja veliko člankov o samih metodah reševanja, je literatura, ki različne metode primerja in odgovarja na razlike med njimi maloštevilna. To delo zato poskuša analizirati rezultate algoritmov na različnih realnih problemih, ter ugotoviti, kako se rezultati algoritmov v različnih primerih razlikujejo. Ponovno poskušamo rezultate tudi interpretirati in povezati s samo implementacijo algoritma.

Glavni prispevki te diplomske naloge so implementacija vseh omenjenih algoritmov, testiranje algoritmov na različnih primerih ter odgovori na vprašanja o uporabnosti metod, ki se nam med implementacijo porodijo. Podamo tudi vizualen prikaz rekonstruiranih slik, kot tudi grafičen prikaz napak in časov do konvergence programov.

\section{Struktura diplomskega dela}

Poglavje \ref{1407-1010} pregleda že napisana dela o samem reševanju matričnih napolnitev in v kratkem predstavi njihove ugotovitve. Poglavje \ref{1407-1011} predstavi algoritme in njihove ideje. V poglavju \ref{1407-1012} predstavimo rezultate algoritmov na zašumljenih slikah. Poglavje je razdeljeno na razdelke, v katerih poskušamo odgovoriti na različna vprašanja. V poglavju \ref{1407-1013} diplomsko delo strnemo in podamo ideje, kako bi lahko delo nadaljevali.