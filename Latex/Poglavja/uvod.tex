\chapter{Uvod}
\todo{spisi in popravi vse}
\section{Motivacija}
Problem matričnih napolnitev sprejme matriko, največkat označeno z $M$, pri kateri so nekateri elementi označeni kot neznani. Problem nato sprašuje po vrednostih, ki jih lahko vstavimo v neznane vrednosti, tako da bo rang matrike najmanjši možen. Gre za NP-poln problem, zato ga poskušamo poenostaviti, ter reševati lažje probleme, ki vrnejo dovolj dobre, a ne optimalne rešitve. 

Problem je v zadnjih letih zelo popularen, z njim pa se ukvarjajo tako številni matematiki kot računalničarji. Njegova splošnost naredi reševanje problema na številnih področjih, sam pa se v diplomski nalogi osredotočim na razreševanje neznanih pikslov v slikah. Prav tako omenjam in preizkusim algoritem na priporočilnih sistemih. Rezultate teh predstavim v poglavju X.

V tej diplomski nalogi bom predstavil par  algoritmov, ki rešujejo omenjen problem. Algoritmi so bili izbrani glede na njihovo popularnost in priznanost v literaturi. Prav tako sem poskrbel, da so algoritmi primerno različni in temeljijo na drugačnih principih. Algoritme sem tudi implementiral, nato pa še napravil analizo ter opisal ugotovitve v poglavju X.

\section{Cilji}

Nekaj o tem, kaj želimo narediti.

Glavni prispevki tega diplomskega dela so ...

\section{Struktura diplomskega dela}

V poglavju 2 ,,,, v poglavju 3 ....