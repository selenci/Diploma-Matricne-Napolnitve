\chapter{Pregled sorodnih del}


%--Paper--
\cite{BK07} Opisuje tehnike, ki jih je uporabila zmagovalna ekipa Netflixove nagrade.
Poglavje 3.1 opisuje uporabo k-NN metode ter izračun uteži glede na podatke o uporabnikih in filmih.

Poglavje 3.2 opisuje modele latentnih faktorjev. Metoda uporablja regulitacijo 
s členom $\lambda (||p_u||^2 + ||q_i||^2)$, kjer so ugotovili,
da je najboljše določiti $\lambda=0.05$. Algoritem omogoča iskanje medsebojnih vplivov
uporabnikov in filmov zaradi določenih lastnosti.

\cite{CR08} Opisuje uporabo matričnih napolnitev.
Poglavje 1.1.1 opisuje katere matrike lahko napolnimo.
Poglavja 1.1.3, 1.3 opisujejo katere algoritme je vredno obravnavati, poglavje 1.4 pa kako lahko to dosežemo z uporabo semi-definitnega programiranja.

Poglavje 2 opisuje inkoherenco, vrednost, ki nam pomaga določiti verjetnost,
da bomo matriko lahko obnovili.

Poglavje 7 definira preizkus uporabe algoritmov matričnih napolnitev,
ter predstavlja rezultate kdaj je bilo matrike možno obnoviti in kdaj ne.

\cite{VV20} Kratek video o matričnih napolnitvah, ki vsebuje preprost a počasen algoritem 
za reševanje problema.

\section{Singularni razcep}

Pri iskanju singularnega razcepa (SVD) matrike $A \in \mathbb{R}^{n_1 \times n_2}$ iščemo tri matrike, tako da velja

\[
A = U \Sigma V^T
\]

kjer velja, da je matrika $U \in \mathbb{R}^{n_1 \times n_1}$ ortogonalna in sestavljena iz levih singularnih vektorjev. Matrika $\Sigma \in \mathbb{R}^{n_1 \times n_2}$ je diagonalna, elementi na diagonali pa so po velikosti urejene singularne vrednosti matrike $A$. Matrika $V \in \mathbb{R}^{n_2 \times n_2}$ je ortogonalna, vendar sestavljena iz desnih singularnih vektorjev. 

Razcep SVD je bližnje povezan z dekompozicijo lastnih vrednosti simetričnih matrik $A^T A$, $A A^T$ ter 
$\begin{bsmallmatrix}
  0 & A^T\\
  A & 0
\end{bsmallmatrix}$.
Algoritmi nato reducirajo matriko $A$  na $A = U_1BV_1^T$, kjer sta $U_1$ ter $V_1^T$ ortogonalna, B pa ima elemente le na glavni diagonali ter na diagonali nad njo. Nato poiščemo SVD razcep matrike $B = U_2 \Sigma V_2^T$, ter razcep vstavimo v prejšnjo enačbo. Tako dobimo $A = U_1U_2 \Sigma (V_1V_2)^T$. Sedaj lahko vidimo da velja $U = U_1U_2$ ter $V = V_1V_2$.\cite{D97}