\section{Algoritem minimizacije nuklearne norme} \label{1707-1755}
Ker je minimizacija ranga matrike NP-poln problem \cite{NNM-PHD}, algoritmi matričnih napolnitev temeljijo na metodah, ki rešujejo sorodne, hkrati pa bistveno enostavnejše probleme. \textbf{Algoritem minimizacije nuklearne norme (NNM)} uporabi idejo, da je rang matrike tesno povezan z njeno nuklearno normo.
Izkaže se \cite{NNM-PHD}, da je nuklearna norma konveksna ovojnica funkcije ranga. \todo{poglej razliko convex hull in lower envelope} \textbf{Konveksna ovojnica funkcije} $f : \mathbb{R}^{n_1 \times n_2} \rightarrow \mathbb{R}$ je definirana kot 
\begin{align*}
    \tilde{f}(A) 
    &= \sup\{g(A) \mid g:\mathbb R^{n_1\times n_2}\to \mathbb R \text{ je konveksna funkcija in }\\ 
    &\hspace{3cm}\forall A\in \mathbb R^{n_1\times n_2}: g(A) \leq f(A) \}.
\end{align*}
Problem minimizacije nuklearne norme je možno pretvoriti v optimizacijski problem s področja \textbf{semidefinitnega programiranja (SDP)}, ki ga lahko rešujemo z učinkovitimi orodji, npr.\ knjižnico SeDuMi \cite{SeDuMi} v Matlabu.

Naj bodo $C,A_1,\ldots,A_\ell \in \mathbb{R}^{n \times n}$ dane matrike in $b_1,\ldots,b_\ell\in \mathbb R$ dana števila.
\textbf{Standardna oblika semidefinitnega programa} je
\begin{align}
\label{1007-1829} 
\begin{split}
    \min_{Y\in \mathbb R^{n\times n}} \hspace{0.5cm}             & \trOp{C}{Y},                                                                         \\
    \text{pri pogojih} \hspace{0.5cm} & \trOp{A_k}{Y} = b_k, \hspace{0.5cm} k = 1, \hdots , \ell,\\
                                      & Y \succeq 0
\end{split}
\end{align}

Vrnimo se k našemu problemu. Naj bo $M \in \mathbb{R}^{n_1 \times n_2}$ zašumljena matrika (tj.\ matrika z nekaterimi neznanimi vhodi), $\Omega$ pa množica urejenih parov, ki predstavljajo mesta, kjer vrednosti matrike poznamo.
Problem minimizacije nuklearne norme lahko zapišemo kot
\begin{align}
\label{1007-1830}
\begin{split}
    \min_{\substack{X\in \mathbb R^{n_1\times n_2},\\ t\in \mathbb R}} \hspace{0.5cm}        & t,                                               \\
    \text{pri pogojih} \hspace{0.5cm} & \nnorm{X} \leq t,  \\
                                      & \proj(X) = \proj(M),
\end{split}
\end{align}
Hitro se da videti \cite{NNM-PHD}, da za matriko $X \in \mathbb{R}^{n_1 \times n_2}$ in $t \in \mathbb{R}$
velja 
\begin{align}
\label{2407-1644}
\begin{split}
\nnorm{X} \leq t &\iff \exists W_1\in \mathbb{R}^{n_1 \times n_1}, W_2 \in \mathbb{R}^{n_2 \times n_2}:\\
&\hspace{2cm}
Y = \begin{bmatrix}
               W_1 & X                     \\
               X^T & W_2 
           \end{bmatrix},\;       Y \succeq 0,\; \tr(Y) \leq 2t.
\end{split}
\end{align}
Minimizacijski problem \eqref{1007-1830} lahko z uporabo \eqref{2407-1644} zapišemo kot
\begin{align}
\label{2407-1649}
\begin{split}
    \min_{\substack{Y\in \mathbb R^{(n_1+n_2)\times (n_1+n_2)},\\ t\in \mathbb R}}     \hspace{0.5cm} & 2t,                        \\
    \text{pri pogojih} \hspace{0.5cm} & \tr(Y) \leq 2t,           \\
                                  & Y \succeq 0,              \\
                                  & \trOp{Y}{A_{ab}} = M_{ab},\quad (a,b)\in \Omega,
\end{split}
\end{align}
kjer so $A_{ab} \in \mathbb{R}^{(n_1 + n_2) \times (n_1 + n_2)}$ matrike, ki imajo vse elemente ničelne, razen tistega na mestu $(a, n_1 + b)$, ki ima vrednost $1$.
\iffalse
Ker velja
\[
    \trOp{A}{B} = \sum_{i}^{n_1} \sum_{j}^{n_2} a_{ij}b_{ij}
\] je lahko videti, da je tak pogoj smiselen. 
\fi
Programi za reševanje SDP-jev pa lahko sprejmejo in rešujejo probleme v obliki \eqref{2407-1649}
\cite{Survey-NKS19}.


% Za SVD razcep matrike $X = U\Sigma V^T$ velja 
% \[
%     \tr \begin{bmatrix}
%         UU^T & -UV^T \\
%         -VU^T & VV^T
%     \end{bmatrix}
%     \begin{bmatrix}
%         Y & X \\
%         X^T & Z
%     \end{bmatrix} \geq 0
% \] \todo{zakaj je prva matrika pozitivno semidefinitna?}
% ker je vsota diagonalnih elementov produkta dveh pozitivnih semidefinitnih matrik vedno nenegativna. Tako vemo, da 
% \[
%     \tr(UU^TY) - \tr(UV^TX^T) - \tr(VU^TX) + \tr(VV^TZ) \geq 0
% \]

% Po \cite{CR08} lahko problem definiramo kot
% \begin{align*}
%   \min    & \hspace{0.5cm} \tr(Y)                                    \\
%   \text{tako da} & \hspace{0.5cm} (Y, A_k) = b_k, k = 1, \cdots , |\Omega| \\
%                     & \hspace{0.5cm} Y \succcurlyeq 0
% \end{align*}
% kjer
% \begin{align*}
%   &Y = \begin{bmatrix}
%     W_1 & X   \\
%     X^T & W_2
%   \end{bmatrix}
% \end{align*} tak problem pa lahko že rešujemo s semidefinitnimi programi.