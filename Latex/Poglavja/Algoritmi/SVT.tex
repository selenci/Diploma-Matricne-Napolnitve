\section{Prag singularnih vrednosti}

Algoritem praga singularnih vrednosti, oziroma v nadaljevanju SVT (Singular Value Thresholding) sloni na dejstvu, da imamo pri matrikah z majhnim rangom nekaj velikih singularnih vrednosti, ostale pa blizu ničli. Za svoje delovanje uvede dva nova pomembna koncepta, prvi je premik, drugi pa prag, potreben za uporabo operatorja $\shrink_\tau$. Algoritem lahko na kratko povzamemo z zapisom
\[
    \begin{cases}
        X^k = \shrink_\tau(Y^{k-1}) \\
        Y^k = Y^{k-1} + \delta_k \proj_\Omega(A - X^k) 
    \end{cases}
\]
kjer $\tau > 0$ predstavlja prag, $\delta_k$ predstavlja $k$-ti premik, $X \in \mathbb{R}^{n_1 \times n_2}$ ter
$Y^0 = 0 \in \mathbb{R}^{n_1 \times n_2}$. \cite{CCS}

Smiselnost algoritma lahko pokažemo s pomočjo Lagrangeovega multiplikatorja. Ponovno rešujemo minimizacijski problem, le da dodamo dodatne parametre in s tem minimizacijo omilimo. 

Uvedimo funkcijo $f_\tau(X) = \tau||X||_* + \frac{1}{2}||X||^2_F$. Problem lahko tako zapišemo kot 
\begin{align*}
    \textrm{min} & \hspace{0.5cm} f_\tau(X) \\
    \textrm{tako da} & \hspace{0.5cm} \proj_\Omega(X) = \proj_\Omega(M)
\end{align*}
Lahko je videti, da za velike vrednosti $\tau$ velja $f_\tau(X) \approx \tau||X||_*$, kar pokaže, da bo s primerno velikim $\tau$, algoritem res skoraj minimiziral nuklearno normo.

Lagrangeov multiplikator je definiran kot $\mathcal{L}(x, \lambda) = f(x) + \lambda \cdot g(x)$, kjer je $f(x)$ funkcija, ki jo minimiziramo, pod pogojem da velja $g(x) = 0$. Naš problem tako prevedemo v
\[
    \mathcal{L}(X, Y) = f_\tau(X) + \left< Y, \proj_\Omega(M - X) \right>
\] 
S pomočjo tako imenovanega Uzawoega algoritma, pa lahko problem pretvorimo v iterativni algoritem 
\[
    \begin{cases}
        \mathcal{L}(X^k, Y^{k-1}) =  \arg \min_{X} \mathcal{L}(X^k, Y^{k-1}) \\
        Y^k = Y^{k-1} + \delta_k \proj_\Omega(M - X^k) 
    \end{cases}
\] \cite{CCS}

Za reševanje minimizacijskega problema dokažimo teorem
\[
    \shrink_\tau(Y) = \arg \min_{X} \{ \frac{1}{2} ||X - Y||^2_F + \tau||X||_* \}
\]
Ker je $h(X) := \frac{1}{2} ||X - Y||^2_F + \tau||X||_*$ strogo konveksna funkcija, lahko za subgradient $Z$ v točki $X_0$ rečemo, da velja $\forall X: f(X) \geq  f(X_0) + \left< Z, X - X_0 \right>$. Ali drugače povedano, vse točke na vseh tangentah na funckijo $h(X)$ bodo pod ali na funkciji $h(X)$. To velja po sami definiciji, saj je to zahtevan pogoj subgradienta v neki točki.

Pri iskanju minimuma torej iščemo tako točko $X'$, da bo subgradient v točki $X'$ enak 0. Problem sedaj zapišemo kot $0 \in X' - Y + \tau \partial||X'||_*$. Izkaže se, da je množica subgradientov nuklearne norme definirana kot
\[
    \partial||X||_* = \{UV^* + W: W \in \mathbb{R}^{n_1 \times n_2}, U^*W = 0, WV = 0, ||W||_2 \leq 1 \}.
\] \todo{Preveri kako se to citira}
kjer $U \Sigma V^T$ predstavlja SVD razcep matrike X.

Cilj dokaza je pokazati, da velja $X' = \shrink_\tau(Y)$ Najprej razčlenimo SVD razcep matrike $Y$ kot 
\[
    Y = U_0\Sigma_0V_0^T + U_1\Sigma_1V_1^T
\],
kjer $U_0, \Sigma_0$ in $V_0$ predstavljajo lastne vrednosti ter njihove lastne vektorje večje od $\tau$, $U_1, \Sigma_1$ in $V_1$ pa tiste manjše od $\tau$. Ustrezno je torej pokazati, da velja 
\[
    X' = U_0(\Sigma_0 - \tau I)V_0^T
\] Gre preprosto za drugačen zapis operatorja $\shrink_\tau(Y)$.
Če zapis vstavimo v prejšnji podan pogoj dobimo
\begin{align*}
    0 = X' - Y + \tau \partial ||X||_*\\
    Y- X' = \tau (U_0 V_0^T + W)
\end{align*}
primerna izbira za $W = \tau^{-1} U_1 \Sigma_1 V_1^T$, saj
\begin{align*}
    Y-X' &= U_0\Sigma_0V_0^T + U_1\Sigma_1V_1^T - U_0(\Sigma_0 - \tau I)V_0^T =\\ 
    &= U_0V_0^T(\Sigma_0 - \Sigma_0 + \tau I) + U_1\Sigma_1 V_1^T = \\
    &= \tau U_0 V_0^T + U_1\Sigma_1 V_1^T
\end{align*}
in 
\begin{align*}
    \tau(U_0 V_0^T + W) &= \tau(U_0V_0^T + \tau^{-1} U_1 \Sigma_1 V_1^T)\\ 
    &= \tau U_0 V_0^T + U_1 \Sigma_1 V_1^T 
\end{align*}

Sedaj je zgolj potrebno pokazati, da veljajo potrebne lastnosti matrike $W$.
\todo{je to res pravilen razlog?} Po sami definiciji SVD vemo, da so vsi stolpci matrik U in V ortogonalni. Torej velja $U_0^TW = 0$ in $WV_0 = 0$. Ker pa ima matrika $\Sigma_1$ vse elemente manjše od $\tau$ velja tudi $||W||_2 \leq 1$. S tem smo pokazali, da $Y - X' \in \tau \partial ||X'||_*$.

Prav tako lahko sedaj zapišemo algoritem kot
\todo{zakaj lahko zapisemo $\mathcal{L}(X^k, Y^{k-1}) = X^k? $ }
\[
    \begin{cases}
        X^k = \shrink_\tau(Y^{k-1}) \\
        Y^k = Y^{k-1} + \delta_k \proj_\Omega(M - X^k) 
    \end{cases}
\] \cite{CCS}.

\subsection{Nastavljanje parametrov}
Medtem, ko so koraki v samem algoritmu definirani kot množica korakov, 
sem sam za premik uporabljal konstanto, ter korak nastavil na 
\[\delta = 1.2\frac{(n_1 * n_2)}{m}\] po priporočilih \cite{CCS}. 

Prav tako članek navaja, da je za matrike velikosti $\mathbb{R}^{n \times n}$ smiselno nastaviti $\tau = 5n$, vendar sem v moji implementaciji zaradi posploševanja na nekvadratne matrike, za matrike velikosti $\mathbb{R}^{n_1 \times n_2}$ parameter nastavil na
\[
    \tau = 5 \frac{n_1+n_2}{2}
\].

Medtem ko so taki parametri dobri za večje matrike, jih ne smemo uporabljati kot definitivno najboljše vrednosti. Med mojimi testiranji sem ugotovil, da je vrednost premika velikokrat treba zmanjšati, posebno za manjše matrike z več neznanimi vrednostmi, saj drugače program ni konvergiral. Prav tako, se je večkrat zgodilo, da je bil pridobljen rezultat še vedno zelo zašumljen. Takrat je bilo smiselno prag $\tau$ povečati. To je sicer upočasnilo program, vendar izboljšalo rezultat.