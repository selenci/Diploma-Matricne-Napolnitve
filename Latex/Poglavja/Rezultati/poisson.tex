\renewcommand{\mapa}{Poglavja/Slike/kompleksnost/kompleksna grayscale 300}

\begin{figure}[H]
    \begin{subfigure}{0.32\linewidth}
        \includegraphics[width=\linewidth]{\mapa/rez35Poisson.png}
        \caption{Rekonstrukcija na 35\% znanih podatkih.}
    \end{subfigure}
    \hfill
    \begin{subfigure}{0.32\linewidth}
        \includegraphics[width=\linewidth]{\mapa/rez45Poisson.png}
        \caption{Rekonstrukcija na 45\% znanih podatkih.}
    \end{subfigure}
    \hfill
    \begin{subfigure}{0.32\linewidth}
        \includegraphics[width=\linewidth]{\mapa/rez60Poisson.png}
        \caption{Rekonstrukcija na 60\% znanih podatkih.}
    \end{subfigure}
\end{figure}

\begin{figure}[H]
    \includegraphics[width=\linewidth]{\mapa/napakaPoisson.png}
    \caption{Napaka rekonstrukcij slike mesta v Frobeniusovi normi. Na abscisni osi so deleži znanih podatkov slik.}
\end{figure}

\begin{figure}[H]
    \includegraphics[width=\linewidth]{\mapa/casPoisson.png}
    \caption{Čas izvajanja rekonstrukcije slike mesta. Na abscisni osi so deleži znanih podatkov slik.}
\end{figure} 
