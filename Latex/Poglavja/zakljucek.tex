\chapter{Zaključek}\label{1407-1013}

Tekom diplomskega dela smo si ogledali 5 različnih algoritmov, za reševanje matričnih napolnitev. Komentirali smo njihovo delovanje ter se spoznali z definicijami, ki so pomembne in pogoste na tem področju. Nato smo pregledali in primerjali rezultate algoritmov, ter jih interpretirali, glede na delovanje algoritmov. Poskušali smo odgovoriti na vprašanja, ki so pri rekonstrukciji pomembna. Prav tako smo algoritme primerjali z drugim popularnim algoritmom za rekonstrukcijo, predstavili razliko v njegovem delovanju, ter podali primer, kjer algoritmi matričnih napolnitev vrnejo boljši rezultat.

Ker je področje matričnih napolnitev široko, bi lahko delo diplomske naloge še razširili. Algoritmi TNNM, LMaFit in ASD imajo še druge, bolj napredne različice implementacije, ki bi jih lahko preizkusili. Seveda pa obstajajo tudi drugi algoritmi, katerih se v tem delu nismo dotaknili. V razdelku \ref{1307-2251} smo omenili, da bi lahko algoritem LMaFit razširili tako, da bi začeli z več pari matrik $X$ in $Y$. \CG{Prav tako pa bi bilo algoritme smiselno preizkusiti še na drugih podatkih, ki niso slike. Ker imajo slike lokalne podatke podobne, na kar pa se pri problemu matričnih napolnitvah ne moremo zanašati, obstajajo boljši algoritmi za rekonstrukcijo slik. Literatura pogosto testira algoritme na naključno generiranih podatkih.}